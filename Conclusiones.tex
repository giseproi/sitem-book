\chapter*{Conclusiones}

El SITEM será el primer Sistema de Información de acceso público que gestiona información de la mayoría de los componentes fundamentales de la Telemedicina – entidades de salud, tecnologías de interconexión, protocolos, dispositivos, legislación, medicina – dotando a la comunidad de una herramienta única en su género la cual podrá ser utilizada en todas las etapas de desarrollo de proyectos en telemedicina.

El desarrollo mismo del SITEM se convierte en un modelo a seguir para proyectos de telemedicina o telesalud orientados a la web proponiendo una metodología –MUDS - interoperable, escalable y fundamentada en software libre. 
Lo anterior unido a la integración de tecnologías web en el SITEM brindará una plataforma robusta para la implementación de servicios web de telesalud y telemedicina básica para entidades de salud que carezcan de altos recursos financieros. Lo que a mediano plazo podrá generar un crecimiento de la demanda de servicios en telemedicina ya que aumenta la relación beneficio/costo a proyectos que por estar fundamentados en software comercial hasta el momento han sido inviables.
Con la puesta en marcha del proyecto SITEM se captará la atención del usuario en salud, para que conozca y entienda las posibilidades de la tecnología de Telemedicina y empiece a tomar un papel activo al presionar la implementación de estos servicios por parte de las entidades prestadoras del servicio de salud. Así mismo se pretende contribuir con el desmonte del monopolio informático que están tratando de imponer los operadores de telecomunicaciones y las empresas del sector privado.

Cuando se propuso el proyecto de Telemedicina Bogotá, se tuvieron en cuenta problemas puntuales para la implementación de servicios telemédicos en el Distrito Capital, como por ejemplo: a) En el momento no existe un diagnóstico real sobre los servicios requeridos en el área de telemedicina, razón suficiente para iniciar un trabajo de campo que establezca la situación actual de servicios médicos y la demanda real, así como la posibilidad de conocer a corto, mediano y largo plazo cuáles serían los costos de inversión que permitirían dar soluciones al problema  de cobertura se servicios de salud. B) La socialización del conocimiento alrededor de las tecnologías aplicadas al desarrollo de la medicina, es uno de los valores que lleva al éxito de soluciones efectivas en el sector salud, por tal motivo es necesario desarrollar un plan de alfabetización en el sector salud y en el sector gubernamental y académico. C) En el país no existen estrategias de investigación en esta área del conocimiento para llevar a cabo un estudio real que permita dar el paso a soluciones verdaderas sobre desarrollo tecnológico o experimental para poder implementar centros de investigación en Telemedicina.
Buscando suplir estas debilidades,  se emprendió la tarea de recopilar la información de los diferentes entes que conformarían una red telemédica. Se crea el proyecto SITEM y en las primeras fases I y II, se diseña y se hace una prueba de funcionalidad lo que lleva a generar la Fase III,  como proyecto para una investigación valiosa en el tema del desarrollo de software distribuido, interoperable, robusto y basado en software libre donde además se entra en fase piloto para depuración de la arquitectura, la depuración de los módulos estructurales y la inclusión de la base de información. La fase IV, se relaciona específicamente con la migración de la solución a plataformas diferentes a la usada en la etapa de prueba piloto con el ánimo de capturar un mayor ámbito del mercado en servicios de Telmedicina y Telesalud. 