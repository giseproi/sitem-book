\section{Portales de Información y Conocimiento}

Antes de la explosión de servicios a través de la Internet, los portales basados en aplicaciones web estaban recomendados solo para organizaciones que por su complejidad (en tamaño ó geografía) necesitaran de un sistema tecnológico para que todo su personal pudiese tener acceso a la información en forma compartida y simultánea. Sin embargo, fundamentado en el crecimiento del uso de Internet\footnote{El porcentaje de la población con acceso a Internet en Colombia a crecido de un 2,1\% en el año 2000 a un 15,8\% en el 2007, según la Comisión de Regulación de Telecomunicaciones.} surgue la necesidad de la sociedad por mantener cierto orden en la corriente de bytes y grupos de usuarios con intereses de información comunes empiezan a conglomerarse alrededor de portales temáticos no organizacionales.

Un portal de información no es, en esencia, una fuente nueva de información; es una vista de la información existente que dispuesta en una forma ordenada se convierte en una herramienta de conocimiento extraordinariamente poderosa permitiendo poner al descubierto información valiosa que se enmascaraba entre otra no menos interesante – Data Mining.
Dentro de las múltiples ventajas que ofrece el portal, es que proporciona la facilidad de obtener información actualizada a muy corto plazo, lo que es indispensable para la óptima toma de decisiones. Dicha información esta disponible, en condiciones óptimas,  veinticuatro horas al día, trescientos sesenta y cinco días del año, permitiendo así acceder a los datos que se necesitan de acuerdo con la disponibilidad singular de tiempo y apoyar la toma de decisiones bajo cualquier circunstancia y lugar.

Gran cantidad de organizaciones y grupos de usuarios están explotando el uso de los portales creando y transformando servicios y procesos tradicionales convirtiéndolos en servidores de autoservicio, pudiendo así dedicarse a aquellos de mayor valor agregado a la organización, el personal o el grupo de investigación. 

\subsection{Beneficios y obstáculos para la implementación de portales basados en aplicaciones web.}

Las facilidades que proporciona la tecnología, permite que el portal sea accedido a través de numerosas opciones, esto es a través de computadoras de escritorio y portátiles integradas a la red interna de la organización, a través de Internet por redes de banda ancha y estrecha y de los diversos medios inalámbricos como son las tecnologías celular, WiFi, WiMax, BlueTooh por intermedio de PDA, celulares y equipos de cómputo en redes WLAN.

Debido a la estructura del portal, se tiene una fuerte correlación entre diversas aplicaciones que nos permiten analizar interrelaciones que serían realmente complejas y tardadas si no se contara con ellos. Sin embargo, es importante recalcar más que los beneficios los problemas potenciales. De hecho en el éxito de un portal están enfocados factores clave que tienen beneficios y problemas asociados.

\begin{description}
 \item[Factor Humano] Los individuos adaptan los procesos de información en diferentes maneras. 
 \item[Factor Tecnológico] Intranets, pueden ser costosas y poco efectivas si la organización no tiene la tecnología necesaria para construirlas.
 \end{description} 

La principal ventaja obtenida al construir y mantener un portal basado en aplicaciones web es mejorar la eficiencia y efectividad en la comunicación de los miembros de una organización o un grupo de usuarios, lo que aumenta la objetividad en la toma de decisiones y la transferencia de conocimiento. Todo lo anterior se maximiza si el portal se concibe como fruto de un proceso de investigación en donde todos sus componentes y servicios se construyen, mantienen, distribuyen e integran de acuerdo a los requerimientos de los usuarios finales\cite{sarmento2005}.

Una de las características importantes de los portales es que en un sólo lugar - y con un mecanismo de acceso unificado, los usuarios pueden acceder a las aplicaciones. Esta integración con aplicaciones y servicios orientados al trabajo colaborativo hacen que trascienda los límites de un mero repositorio organizacional - que permite el autoservicio de requerimientos y extracción de información básica - y lo convierte en una herramienta de administración del conocimiento, útil para la toma de decisiones.

Las novedosas tecnologías que convergen en Internet permiten que la información sea personalizada y dirigida de tal forma que se potencian ciclos de creación, captura y diseminación de conocimiento necesarios para el crecimiento de los activos intangibles de los grupos y organizaciones. Así los portales convierten la información en valor, ya que eliminan las barreras de distancia y disponibilidad de información, reduciendo costos conectando a múltiples personas en diversos sitios al mismo tiempo.

La tabla \ref{beneficio} muestra los principales beneficios potenciales de desplegar los portales de información y conocimiento dentro del quehacer de las organizaciones, los grupos de trabajo y las comunidades de práctica.

\begin{table}
\begin{center}
\begin{tabular}{|l|}
\hline
\textbf{Beneficios Humanos (suaves)}\\
\hline
Provee estructura de soporte 24 hrs.\\
Servicio centrado en el usuario.\\
Medio ambiente amigable.\\
\hline
\textbf{Beneficios Físicos y capitales (beneficios fuertes)}\\
\hline
Creación medioambiente libre de papel.\\
Mejorar eficiencia y efectividad.\\
Reducción de costos.\\
Menores tiempos en consecución de información.\\
\hline
\textbf{Beneficios estratégicos}\\
\hline
Creación de herramientas innovadoras de apoyo.\\
Proveer información a tiempo real.\\
Apoyo a proceso de negocios de reingeniería\\
Apoyo a los ciclos de creación, captura y diseminación de conocimiento.\\
Aumento de Capital intangible.\\
Formalización del \textit{Know-How}.\\
\hline
\end{tabular}
\caption{Algunos Beneficios Potenciales al Implementar un Portal}
\label{beneficio} 
\end{center}
\end{table}

Los riesgos que se afrontan también son enormes y pueden llevar al traste cualquier política o proyecto de desarrollo; la tabla \ref{riesgos} muestra algunos de los más importantes que deben ser minimizados.

\begin{table}
\begin{center}
\begin{tabular}{|l|}
\hline
\textbf{De tipo Humano (Fuertes)}\\
\hline
Indiferencia de la administración.\\
Sobrevaloración del papel de las TIC\\
Dirección centrada en el capital financiero\\
Estructuras organizacionales conservadoras.\\
Micropoderes y feudos organizacionales autogestionados.\\
Ignorancia y/o resistencia respecto al uso de TIC.\\
Resistencia a estandarizar la información.\\
Resistencia a compartir información/conocimiento.\\
\hline
\textbf{Riesgos Físicos/Capital}\\
\hline
Procesos orientados a la técnica y no multidimensionales.\\
Carencia de capital(TIC marginales).\\
Dificultad de integración de tecnología nueva y la existente.\\
Ausencia de inter, trans y multidisciplinariedad.\\
\hline
\textbf{Riesgos tecnológicos (Débiles)}\\
\hline
Estándares propietarios.\\
Redes de interconexión da baja velocidad.\\
Alta relación Consumo/Adopción de tecnología.\\
\hline
\end{tabular}
\caption{Algunos riesgos Potenciales al Implementar un Portal}
\label{riesgos} 
\end{center}
\end{table}

\subsection{Ciclo de Vida de los Portales}
Los portales como representación sistemática del quehacer de un grupo humano evoluciona en la medida que dicho grupo mejora su conocimiento de las relaciones entre sus miembros y el entorno que los rodean. En general pueden determinarse cinco macro-etapas~\cite{egovernment} que aumentan gradualmente su funcionalidad basado en el  conocimiento organizacional y la interrelación de usuarios a través del Portal:

\begin{description}
\item[Presencia Emergente]
Esta es la etapa primaria por la que pasa un portal en donde su funcionalidad es la de distribuir información interesante para un grupo de usuarios, el cual es totalmente caracterizado por el grupo de personas que construye - en todas sus dimensiones, el portal. Dichos usuarios no intervienen directamente en la estructura del Portal el cual complementa sus servicios por medio de enlaces y dependencia a otros portales temáticamente relacionados.

\item[Presencia Mejorada] 
Los usuarios pueden determinar en cierto grado la navegación a través de búsquedas en el archivo del sitio. Un portal en esta etapa presentará gran cantidad de información que usualmente se agrupa por áreas temáticas. Los mapas del portal se distribuyen profusamente con el fin de guiar a los usuarios en su tránsito por el mismo y usualmente un sistema básico de ayuda prediseñada esta disponible.
 
\item[Interacción]
Los portales registran a sus usuarios. Se implementan herramientas en línea como las salas de charla - chat, las listas de correo y los foros; se realiza capacitación básica por medio de seminarios basados en contenidos y se hace uso extensivo de recursos multimediales. La ayuda es síncrona o asíncrona pero ágil lo que fomenta una depuración y actualización de la información contenida en el portal.

\item[Transacción] 
Los usuarios realizan operaciones a través del portal. El comercio electrónico, la gestión de contenidos, la personalización de los ambientes del portal, búsquedas semánticas y el despliegue de servicios avanzados - cursos, blogs, etc; caracterizan esta etapa.

\item[Transformación] 
La etapa más avanzada de los portales en donde se han estructurado comunidades de práctica sobre temas concretos que potencian los ciclos de conocimiento mediante las herramientas brindadas por el portal. Se observa una jerarquía \textit{ad hoc} de usuarios con base en su aporte. Ellos mismos generan contenido que es convalidado por la comunidad y los administradores técnicos limitan sus funciones a aquellas relacionadas con mantener operativa la plataforma tecnológica. Las transacciones y el contacto en tiempo real son rutinarios. Las aplicaciones son de conocimiento general y el nivel de inmersión en el portal es alto.
\end{description}


\subsection{Aplicaciones Web}

También conocidas como \textbf{WebApps} son, en su concepción más básica, aplicaciones que responden a peticiones realizadas por un usuario por medio de un navegador (cliente) y ejecutan la lógica del programa en un servidor. Las aplicaciones web usualmente interactúan con sistemas de bases de datos y distribuyen los resultados de sus operaciones en lenguajes estándar tales como HTML, SMIL, XML,  RDF, SVG, etc.\cite{jackson2005}. Las WebApps también se pueden encontrar en ambientes diferentes al modelo cliente - servidor\cite{bos2004}. 

Entre las características de las aplicaciones web se destacan\cite{bos2004}:

\begin{itemize}
\item \textbf{No requieren instalación.} En general las aplicaciones web no necesitan ejecutar rutinas de instalación en las máquinas cliente. Quizás en algunos lenguajes sea necesario la preparación de un ambiente específico de trabajo que en la mayoría de las veces es de acceso público.

\item \textbf{Accesibilidad.} Las aplicaciones web se despliegan desde de una página web. Los protocolos usados son estandarizados y abstraen fácilmente las capas de aplicación de las de diseño y datos.

\item \textbf{Facilidad en el Desarrollo.} Los lenguajes usados son de alto nivel, con un buen soporte para cadenas de caracteres, diferentes tipos de datos y con facilidades para la programación orientada a objetos. La mayoría de ellos con sintaxis similares y herramientas de desarrollo gratuitas de fácil adquisición. 

\item \textbf{Independencia de la Plataforma.} Las \textit{WebApps engines} implementan el modelo de capa intermedia lo que permite que las diferentes WebApps puedan ser desplegadas sobre diferentes plataformas sin detrimento de su funcionalidad. El uso de métodos genéricos definidos en interfaces de programación (API) ayuda en gran manera a garantizar esta característica.

\item \textbf{Seguridad.} No obstante la facilidad de acceso de la WebApp, estas pueden implementar rutinas avanzadas que brindan ambientes transaccionales seguros aislados del sistema de archivos y configuración del sistema en donde se alojan. El intercambio de información cifrada por la red y la integración con la seguridad de los servidores de bases de datos forman un contexto de alta seguridad.

\item \textbf{Privacidad.} Las WebApps pueden operar fácilmente sobre una Plataforma de Preferencias de Privacidad debido a que la mayoría de los motores están habilitados para soportar el protocolo \textbf{P3P}.

\item \textbf{Almacenamiento Persistente.} Tanto en el cliente - a través de archivos texto para el manejo de sesiones; como en el servidor de base de datos.

\item \textbf{Integración.} Las aplicaciones Web pueden brindar sus servicios - u obtener uno determinado, a través de interfaces claras y definidas en las denominadas redes de servicios Web.
\end{itemize}
