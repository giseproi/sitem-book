\section{Marco Legislativo y Normativo}

OpenSITEM es un sistema que gestiona información sobre diferentes áreas del saber de acuerdo a las categorías de los modelos que se definan. Cada aspecto (atributos, interfaces, interoperaciones) está definido conforme a un marco legislativo y normativo concreto, o a estándares y normas de  uso extendido y de facto aceptado en el mundo. La información de OpenSITEM que sea de acceso público no podrá incluir protección por derechos de autor\cite{congreso565},\cite{congreso23} que restrinjan su difusión. El código fuente de OpenSITEM es cubierto por una licencia abierta que se ciñe a la normatividad expresada en la Ley 565 de 2000: adopción del Tratado de la OMPI sobre Derechos de Autor y complementarias\cite{congreso565},\cite{congreso44},\cite{congreso1360} para garantizar que todos los aspectos tanto técnicos como conceptuales estén debidamente registrados. 

OpenSITEM es una plataforma para la definición de nodos y no es posible \textit{a priori} definir el marco legislativo que regirá cada nodo o categoría. Sin embargo, se describe a continuación un marco relacionado con las categorías base que se han definido en el primer modelo del sistema.

\subsection{En el Ámbito de los Servicios Médicos}

El Derecho a la Salud ha sido reconocido por normas y pactos internacionales contenidos en tratados sobre Derechos Humanos, Económicos, Sociales, y Culturales  (DHESC). Esos acuerdos han sido ratificados por Colombia para su cumplimiento como un derecho de los ciudadanos. “La Corte Constitucional; ha señalado que el inciso segundo del artículo 93 de la Carta Política confiere rango constitucional a todos los tratados de derechos humanos, económicos,  sociales y culturales, ratificados por Colombia y referidos a derechos que ya aparecen en la Carta” \cite{sentencia1319} como ocurre con el Derecho a la Salud. 

Al Ministerio de Salud y Protección Social, le corresponde expedir las normas técnicas y administrativas de obligatorio cumplimiento para las Entidades Promotoras de Salud del régimen contributivo, las Instituciones Prestadoras de Salud del Sistema General de Seguridad Social en Salud, las Administradoras del Régimen Subsidiado y para las Direcciones Seccionales, Distritales y Locales de Salud en cuanto al objetivo de cumplimiento en el desarrollo de actividades de protección específica, detección temprana y atención de enfermedades de interés en Salud Pública. 

A continuación se registra la normatividad que se tuvo en cuenta al momento de definir los componentes actuales del subsistema de servicios médicos en cuanto a la relevancia que se tiene tanto para la proyección de nuevas redes de eSalud, como para apoyar los sistemas básicos ya existentes. Es de anotar que lo contemplado en las leyes nacionales es, en su mayoría, derivado de normas internacionales que han sido objeto de detallados estudios y reconocidas técnicamente con base en las experiencias vividas por los profesionales de esta área. 

\begin{description}

\item[Ley 1751 de 2015]. Por medio de la cual se regula el derecho fundamental a la salud. Es una ley estatutaria que surge a partir de la debacle del proceso de reforma y tiene como efecto positivo el elevar a la salud como un derecho fundamental. Entró en rigor a partir del año 2017 y da lineamientos para reestructurar el sistema de salud a partir del desarrollo de \textit{redes de servicios} públicos, privados o mixtos. También declara la necesidad de establecer políticas relacionadas con la salud tales como la política para la información, la política de innovación, ciencia y tecnología; y la política farmacéutica nacional.

\item[Ley 1419 de 2010].  Por la cual se establecen los lineamientos para el desarrollo de la telesalud en Colombia. Define las redes de telesalud y el aprendizaje en telesalud como ejes principales de la gestión del conocimiento en salud. Si bien esta ley obliga a desarrollar el mapa de conectividad, aún en el 2017 no se encuentra uno que esté disponible para los ciudadanos.

\item[Resolución 2182 de julio 9 de 2004] Con esta resolución se definían las Condiciones de Habilitación para las instituciones que prestan servicios de salud bajo la modalidad de Telemedicina. Fue derogada por el artículo 11 de la \textbf{resolución 1043 de 2006}, con la cual se establecen las condiciones que deben cumplir los Prestadores de Servicios de Salud para habilitar sus servicios e implementar el componente de auditoría para el mejoramiento de la calidad de la atención y se dictan otras disposiciones. 

Posteriormente, con la \textbf{Resolución 1448 de 8 de Mayo de 2006} se regula la prestación de servicios de salud bajo la modalidad de telemedicina y establece las condiciones de habilitación de obligatorio cumplimiento para las instituciones que prestan servicios de salud. Esta resolución aclara que las actuaciones de los médicos en el ejercicio de la prestación de servicios bajo la modalidad de telemedicina se sujetarán a las disposiciones establecidas en la \textbf{Ley 23 de 1981} y demás normas que la reglamenten, modifiquen, adicionen o sustituyan.

\item[Resolución 4678 de noviembre 11 de 2015] Con esta resolución, modificada por la Resolución 1113 de 2017, el Ministerio de Salud y Protección Social adopta una Clasificación Única de Procedimientos en Salud (CUPS) la cual "...corresponde a un ordenamiento lógico y detallado de los procedimientos y servicios en salud que se realizan en al país, en cumplimiento de los principios de interoperabilidad y estandarización de datos utilizando, para tal efecto, la identificación por un código y una descripción validada por los expertos del país."\cite{minsalud4678}. La Clasificación Única de Procedimientos en Salud adaptación para Colombia, se implementa por \textbf{Resolución 365 de 1999}. Su primera publicación se presenta en un solo volumen que contiene la Lista Tabular y el Índice Alfabético. A partir de dicha resolución se realizó la primera actualización de la CUPS (1°A-CUPS) mediante la \textbf{Resolución 2333 de 2000}. En el año 2016, mediante la Resolución 3804, se establece el procedimiento para la actualización de la CUPS, con lo que el Ministerio espera darle una mayor agilidad al proceso.

\item[Resolución 1830 de junio 23 de 1999] adopta para Colombia, “Las codificaciones únicas de especialidades en salud, ocupaciones, actividades económicas y medicamentos esenciales" para el Sistema Integral de Información del SGSSS - SIIS 

\item[Resolución 1895 de noviembre 19 de 2001] por la cual se adopta para la codificación de morbilidad en Colombia, La Clasificación Estadística Internacional de Enfermedades y Problemas Relacionados con la Salud - Décima revisión. 

\begin{quote}
Considerando que en la 43a. Asamblea Mundial de la Salud llevada a cabo en 1990, fue aprobada por la Conferencia Internacional la Clasificación Estadística Internacional de Enfermedades y Problemas Relacionados con la Salud - Décima revisión -, (CIE-10) en la cual Colombia no expresó objeciones y adquirió el compromiso de implementarla. Resuelve Adoptar para la codificación de morbilidad en Colombia, la Clasificación Estadística Internacional de Enfermedades y Problemas Relacionados con la Salud -Décima revisión-, contenida en la publicación científica No.554 de la Organización Panamericana de la Salud, presentada en tres volúmenes: V1. Lista de Categorías; V2 Manual de Instrucciones; V3 Índice Alfabético.\end{quote} 

\item[Resolución 1995 de julio 8 de 1999] por la cual se establecen normas para el manejo de la Historia Clínica.

La Historia Clínica es un documento de vital importancia para la prestación de los servicios de atención en salud y para el desarrollo científico y cultural del sector, \textit{es un documento privado, obligatorio y sometido a reserva}, en el cual se registran cronológicamente las condiciones de salud del paciente, los actos médicos y los demás procedimientos ejecutados por el equipo de salud que interviene en su atención. Dicho documento únicamente puede ser conocido por terceros previa autorización del paciente o en los casos previstos por la ley. 


\item[Circular 015 de abril 4 de 2002] estándar de historias clínicas y registros, establece la obligatoriedad de definir procedimientos para utilizar una historia única institucional. Ello implica que la institución cuente con un mecanismo para unificar la información de cada paciente y su disponibilidad para el equipo de salud. No es restrictivo en cuanto al uso de medio magnético para su archivo, y sí es expreso en que debe garantizarse la confidencialidad y el carácter permanente de registrar en ella y en otros registros asistenciales.


\item [Decreto 2092 de 2 de Julio de 1986], Por el cual se reglamenta parcialmente los Títulos VI y XI de la \textbf{Ley 09 de 1979}, en cuanto a la elaboración, envase o empaque, almacenamiento, transporte y expendio de Medicamentos, Cosméticos y Similares. Se dan las Disposiciones Generales y Definiciones, el Registro Sanitario de Medicamentos, Cosméticos y Similares.
\end{description}

\subsection{En el Ámbito del Desarrollo de Redes Teleinformáticas}

\begin{description}
\item[Documento CONPES 3072] Aunque no es una norma regulatoria, es una declaración oficial del gobierno colombiano acerca de la necesidad de fomentar las Tecnologías de la Información para potenciar la absorción, creación y divulgación del conocimiento por medio del desarrollo sostenible en las infraestructuras física, de información y social. Según \cite{mincomunicaciones3072}:  “... para que el país pueda ofrecer un entorno económico atractivo y participar en la economía del Conocimiento, resulta indispensable desarrollar una sociedad en la que se fomente el uso y aplicación de las Tecnologías de la Información. A través de estas Tecnologías, se puede efectuar un salto en el desarrollo en un tiempo relativamente breve, mucho menor del que se necesita para superar el déficit de infraestructura física.".

El documento CONPES brinda un referente válido pues la mayoría de los objetivos estratégicos del SITEM contienen el espíritu expresado en diferentes partes del mismo.

\item[Documento CONPES 3582] Política Nacional de Ciencia, Tecnología e Innovación. En el cual se enfatiza el desarrollo de la salud y la tecnología como mecanismos de generación de valor social.

\item[Resolución 087 de 1997] “Por medio de la cual se regula en forma integral los servicios de Telefonía Pública Básica Conmutada (TPBC) en Colombia.” En donde claramente se expresa que: \begin{quote}
Los servicios de TPBC deberán ser utilizados como instrumento para impulsar el desarrollo político, económico y social del país con el objeto de elevar el nivel y la calidad de vida de los habitantes en Colombia. Los servicios de TPBC serán utilizados responsablemente para contribuir a la defensa de la democracia, a la promoción de la participación de los colombianos en la vida de la Nación y la garantía de la dignidad humana y de otros derechos fundamentales consagrados en la Constitución Política, y para asegurar la convivencia pacífica.\end{quote} 

Esta resolución presenta particular importancia para la extracción de elementos semánticos y algunos componentes necesarios en las redes de telecomunicaciones basadas en telefonía conmutada. Algunas heurísticas usadas en el subsistema de consultoría también se basan en apartes de esta resolución.

\item [Manual de Calidad de Servicio] “Con este Manual de calidad de servicio (QoS) se especifican los parámetros de calidad de servicio de red que permiten el suministro de servicios a los clientes y los usuarios, satisfaciendo sus expectativas de calidad de servicio. Estos parámetros tienen que ver tanto con la implementación del servicio como con su utilización continua. Asimismo, la calidad de servicio se relaciona con todos los aspectos relativos a la evaluación y gestión de las redes.”\cite{ITU2004}

Sus principales aspectos son recogidos en \cite{crtcondiciones} y \cite{crtindicadores}  las cuales sirven de base para el proyecto de resolución\cite{crtqos} de la Comisión de Regulación de Comunicaciones y que especifica entre otros: las definiciones relativas a la QoS, parámetros de medición, variables y propiedades técnicas de diferentes enlaces. Aunque el objetivo principal de este manual es el de garantizar la QoS en un sistema de telecomunicaciones dado es importante notar que sus indicaciones deben ser tenidas en cuenta al momento de proyectar los servicios de telecomunicaciones en un sistema de telemedicina dado. En estos aspectos también se considera dentro del modelado de ciertos componentes del SITEM las recomendaciones de la UIT en \cite{ITUG1000} y \cite{ITUG1010}, las cuales aún no tienen un equivalente en la normatividad colombiana pero son de uso extendido alrededor del orbe.

\end{description}

\subsection{En el ámbito del Tratamiento de Datos e Información}  

\begin{description}
 \item[Ley 1581 de 2014 y Ley 1266 de 2008]. Estas leyes están relacionadas con la protección de los datos personales y el aseguramiento de la privacidad de la información personal.
 
\end{description}