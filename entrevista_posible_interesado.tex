\chapter{Entrevista Base para Identificación de Interesados}
\label{appendix:entrevista_posible_interesado}

En la elaboración y validación de este instrumento de recolección de la información se siguió el método definido en la sección de \textit{Metodología para la Estructuración y Validación del Documento}, presentado en el capítulo de Estructura del Proyecto. Es necesario recalcar que el formato de entrevista es semi-estructurado y el entrevistador tendrá las preguntas solo como referencia.

\section{Metodología de Procesamiento}
El instrumento debe aplicarse en la plataforma de gestión de encuestas. Cada una de las respuestas se analiza de manera separada y se califican los atributos de claridad, pertinencia, independencia (que no tenga similitud con respuestas dadas por otro entrevistado), impacto y completitud.

Las preguntas que tengan mayores índices se procesarán primero y los usuarios que las respondieron se clasificaran como interesados del proyecto. Cada una de las preguntas viene acompañada de su categoría para facilitar la interpretación.

\section{Instrumento}

\subsection{Pregunta}
Categoría: Hechos.
Objetivo: Identificar si el entrevistado posee información específica.
Enunciado: ¿Cuáles son los procedimientos que desarrolla al interior del proyecto?

\subsection{Pregunta}
Categoría: Hechos.
Enunciado: ¿Qué datos produce al desarrollar su trabajo?

\subsection{Pregunta}
Categoría: Opinión.
Objetivo: Determinar la percepción del entrevistado respecto a un aspecto concreto.
Comentario: Se utiliza para obtener un criterio base para análisis multi-temporal.
Enunciado: ¿Cuáles son los procedimientos que desarrolla al interior del proyecto?

\subsection{Pregunta}
Categoría: Creencia.
Objetivo: Adquirir un conocimiento de la creencia, prejuicio o concepto que tiene el entrevistado respecto a un tema.
Comentario: Se utiliza como insumo para analizar posibles obstáculos o fortalezas que pueda tener una creencia en el desarrollo del proyecto.
Enunciado: ¿Cree que las tecnologías de información pueden contribuir a mejorar la productividad del grupo?

\subsection{Pregunta}
Categoría: Opinión.
Enunciado: ¿Las herramientas software que apoyan su trabajo al interior del grupo son adecuadas?

\subsection{Pregunta}
Categoría: Opinión.
Enunciado: ¿Las herramientas software que apoyan su trabajo al interior del grupo son suficientes?

\subsection{Pregunta}
Categoría: Intención.
Objetivo: Examinar los mecanismos de optimización de tareas que utilizaría el entrevistado.
Enunciado: ¿Cómo desarrollaría su trabajo si existiese una herramienta adecuada?











