\chapter{GNU General Public License}
\label{licencia_gnu}
\textit{Licencia Pública General\\
Versión 2, Junio de 1991\\
Copyright (C) 1989, 1991 Free Software Foundation, Inc.\\
59 Temple Place - Suite 330, Boston, MA 02111-1307, USA\\
Traducción No oficial}

Toda persona tiene permiso de copiar y distribuir copias fieles de este documento de licencia, pero no se permite hacer modificaciones.

\section{PREAMBULO}

Los contratos de licencia de la mayor parte del software están diseñados para quitarle su libertad de compartir y modificar dicho software. En contraste, la \textbf{GNU General Public License} pretende garantizar su libertad de compartir y modificar el software, esto para asegurar que el software es libre para todos sus usuarios.

Esta licencia pública general se aplica a la mayoría del software de la “FSF Free Software Foundation” (Fundación para el Software Libre) y a cualquier otro programa de software cuyos autores así lo establezcan. Algunos otros programas de software de la Free Software Foundation están cubiertos por la “LGPL Library General Public License” (Licencia Pública General para Librerías), la cual puede aplicar a sus programas también.

Cuando hablamos de software libre, nos referimos a libertad, no a precio. Esta licencia está diseñada para asegurar que: 

\begin{enumerate}
\item Usted tiene la libertad de distribuir copias del software libre (y cobrar por ese sencillo servicio si así lo desea) 
\item Recibir el código fuente (o tener la posibilidad de obtenerlo si así lo desea) 
\item Usted puede modificar el software o utilizar partes de el en nuevos programas de software libre 
\item Usted esté enterado de que tiene la posibilidad de hacer todas estas cosas. 
\end{enumerate}


Para proteger sus derechos, necesitamos hacer restricciones que prohíban a cualquiera denegarle estos derechos o a pedirle que renuncie a ellos. Estas restricciones se traducen en algunas responsabilidades para usted si distribuye copias del software, o si lo modifica. 

Por ejemplo, si usted distribuye copias de un programa, ya sea gratuitamente o por algún importe, usted debe dar al que recibe el software todos los derechos que usted tiene sobre el mismo. Debe asegurarse también que reciban el código fuente o bien que puedan obtenerlo si lo desean. Y por último debe mostrarle a esa persona estos términos para que conozca los derechos de que goza. 

Nosotros protegemos sus derechos en 2 pasos: (1) protegiendo los derechos de autor del software y (2) ofreciéndole este contrato de licencia que le otorga permiso legal para copiar, distribuir y modificar el software. 

Además, para la protección de los autores de software y la nuestra, queremos asegurarnos de que toda persona entienda que \textbf{no existe ninguna garantía del software libre}. Si el software es modificado por alguien y lo distribuye, queremos que quienes lo reciban sepan que la copia que obtuvieron no es la original, por lo que cualquier problema provocado por quien realizó la modificación no afectará la reputación del autor original. 

Finalmente, cualquier programa de software libre es constantemente amenazado por las patentes de software. Deseamos evadir el peligro de que los re-distribuidores de un programa de software libre obtenga individualmente los derechos de patente con el fin de volver dicho programa propietario. Para prevenir esto, hemos dejado en claro que cualquier patente deberá ser licenciada para el uso libre de toda persona o que no esté licenciada del todo. 

A continuación se describen con precisión los términos y condiciones para copiar, distribuir y modificar el software. 

\section{TERMINOS Y CONDICIONES PARA COPIA, MODIFICACION Y DISTRIBUCION }


0. Esta licencia aplica a cualquier programa o trabajo que contenga una nota puesta por el propietario de los derechos del trabajo estableciendo que su trabajo puede ser distribuido bajo los términos de esta “GPL General Public License”. El “Programa”, utilizado en lo subsecuente, se refiere a cualquier programa o trabajo original, y el “trabajo basado en el Programa” significa ya sea el Programa o cualquier trabajo derivado del mismo bajo la ley de derechos de autor: es decir, un trabajo que contenga el Programa o alguna porción de el, ya sea íntegra o con modificaciones y/o traducciones a otros idiomas. De aquí en adelante “traducción” estará incluida (pero no limitada a) en el término “modificación”, y la persona a la que se aplique esta licencia será llamado “usted”. 

Otras actividades que no sean copia, distribución o modificación no están cubiertas en esta licencia y están fuera de su alcance. El acto de ejecutar el programa no está restringido, y la salida de información del programa está cubierta sólo si su contenido constituye un trabajo basado en el Programa (es independiente de si fue resultado de ejecutar el programa). Si esto es cierto o no depende de la función del programa. 

\begin{enumerate}
\item Usted puede copiar y distribuir copias fieles del código fuente del programa tal como lo recibió, en cualquier medio, siempre que proporcione de manera consiente y apropiada una nota de derechos de autor y una declaración de no garantía, además de mantener intactas todas las notas que se refieran a esta licencia y a la ausencia de garantía, y que le proporcione a las demás personas que reciban el programa una copia de esta licencia junto con el Programa. 
Usted puede aplicar un cargo por el acto físico de transferir una copia, y ofrecer protección de garantía por una cuota, lo cual no compromete a que el autor original del Programa responda por tal efecto. 

\item Usted puede modificar su copia del Programa o de cualquier parte de él, formando así un trabajo basado en el Programa, y copiar y distribuir tales modificaciones o bien trabajar bajo los términos de la sección 1 arriba descrita, siempre que cumpla con las siguientes condiciones: 

A. Usted debe incluir en los archivos modificados notas declarando que modificó dichos archivos y la fecha de los cambios. 

B. Usted debe notificar que ese trabajo que distribuye contiene totalmente o en partes al Programa, y que debe ser licenciado como un conjunto sin cargo alguno a cualquier otra persona que reciba sus modificaciones bajo los términos de esta Licencia. 

C. Si el programa modificado lee normalmente comandos interactivamente cuando es ejecutado, usted debe presentar un aviso, cuando el programa inicie su ejecución en ese modo interactivo de la forma más ordinaria, que contenga una noticia de derechos de autor y un aviso de que no existe garantía alguna (o que sí existe si es que usted la proporciona) y que los usuarios pueden redistribuir el programa bajo esas condiciones, e informando al usuario como puede ver una copia de esta Licencia. (Excepción: si el programa en sí es interactivo pero normalmente no muestra notas, su trabajo basado en el Programa no tiene la obligación de mostrar tales notas) 

Estos requerimientos aplican al trabajo modificado como un todo. Si existen secciones identificables de tal trabajo que no son derivadas del Programa original, y pueden ser razonablemente consideradas trabajos separados e independientes como tal, entonces esta Licencia y sus términos no aplican a dichas secciones cuando usted las distribuye como trabajos separados. Pero cuando usted distribuye las mismas secciones como parte de un todo que es un trabajo basado en el Programa, la distribución del conjunto debe ser bajo los términos de esta Licencia, cuyos permisos para otras personas que obtengan el software se extienden para todo el software, así como para cada parte de el, independientemente de quién lo escribió. 

No es la intención de esta sección de reclamar derechos o pelear sus derechos sobre trabajos hechos enteramente por usted, en lugar de eso, la intención es ejercer el derecho de controlar la distribución de los trabajos derivados o colectivos basados en el Programa. 
Adicionalmente, el simple agregado de otro trabajo NO basado en el Programa al Programa en cuestión (o a un trabajo basado en el Programa) en algún medio de almacenamiento no pone el otro trabajo bajo el alcance de esta Licencia. 

\item Usted puede copiar y distribuir el Programa (o un trabajo basado en él, bajo la Sección 2) en código objeto o en forma de ejecutable bajo los términos de las secciones 1 y 2 arriba descritas siempre que cumpla los siguientes requisitos: 

A. Acompañarlo con el correspondiente código fuente legible por la máquina, que debe ser distribuido bajo los términos de las secciones 1 y 2 y en un medio comúnmente utilizado para el intercambio de software, o 

B. Acompañarlo con una oferta escrita, válida por al menos 3 años y para cualquier persona, por un cargo no mayor al costo que conlleve la distribución física del código fuente correspondiente en un medio comúnmente utilizado para el intercambio de software, o 

C. Acompañarlo con la información que usted recibió sobre la oferta de distribución del código fuente correspondiente. (Esta alternativa está permitida sólo para distribución no-comercial y sólo si usted recibió el Programa en código objeto o en forma de ejecutable con tal oferta de acuerdo a la subsección b anterior) 

El código fuente de un trabajo significa la forma preferida de hacer modificaciones al mismo. Para un trabajo ejecutable, un código fuente completo significa todo el código fuente de todos los módulos que contiene, mas cualquier archivo de definición de interfases, mas los programas utilizados para controlas la compilación y la instalación del ejecutable. 

Sin embargo, como excepción especial, no se requiere que el código fuente distribuido incluya cualquier cosa que no sea normalmente distribuida con las componentes mayores (compilador, kernel, etc.) del sistema operativo en el cual el ejecutable corre, a menos de que una componente en particular acompañe al ejecutable. 

Si la distribución del ejecutable o del código objeto se hace ofreciendo acceso a copiar desde un lugar designado, entonces el ofrecer acceso equivalente para copiar el código fuente desde el mismo lugar se considera distribución del código fuente, aunque las demás personas no copien el código fuente junto con el código objeto. 

\item Usted no puede copiar, modificar, sub-licenciar ni distribuir el Programa a menos que sea expresamente bajo esta Licencia, de otra forma cualquier intento de copiar, modificar, sub-licenciar o distribuir el programa es nulo, y automáticamente causará la pérdida de sus derechos bajo esta Licencia. Sin embargo, cualquier persona que haya recibido copias o derechos de usted bajo esta Licencia no verán terminadas sus Licencias ni sus derechos perdidos mientras ellas continúen cumpliendo los términos de esta Licencia. 

\item Usted no está obligado a aceptar esta Licencia, dado que no la ha firmado. Sin embargo, nada le otorga el permiso de modificar o distribuir el Programa ni sus trabajos derivados. Estas acciones están prohibidas por la ley si usted no acepta esta Licencia. Sin embargo, modificando o distribuyendo el Programa (o cualquier trabajo basado en el Programa) indica su aceptación de esta Licencia y de todos sus términos y condiciones para copiar, distribuir o modificar el Programa y/o trabajos basados en el. 

\item Cada vez que usted redistribuye el Programa (o cualquier trabajo basado en el Programa), la persona que lo recibe automáticamente recibe una licencia del autor original para copiar, distribuir o modificar el Programa sujeto a estos términos y condiciones. Usted no puede imponer ninguna restricción adicional a las personas que reciban el Programa sobre los derechos que en esta Licencia se les otorga. Usted no es responsable de forzar a terceras personas en el cumplimiento de esta Licencia. 

\item  Si como consecuencia de un veredicto de un juzgado o por el alegato de infringir una patente o por cualquier otra razón (no limitado solo a cuestiones de patentes) se imponen condiciones sobre usted que contradigan los términos y condiciones de esta Licencia, éstas no le excusan de los términos y condiciones aquí descritos. Si usted no puede distribuir el producto cumpliendo totalmente con las obligaciones concernientes a la resolución oficial y al mismo tiempo con las obligaciones que se describen en este contrato de Licencia, entonces no podrá distribuir más este producto. Por ejemplo, si una licencia de patente no permitirá la distribución del Programa de forma libre de regalías (sin pago de regalías) por parte de quienes lo reciban directa o indirectamente, entonces la única forma de cumplir con ambas obligaciones es renunciar a la distribución del mismo. 

Si cualquier parte de esta sección resulta inválida, inaplicable o no obligatoria bajo cualquier circunstancia en particular, la tendencia de esta es a aplicarse, y la sección completa se aplicará bajo otras circunstancias. 
La intención de esta sección no es la de inducirlo a infringir ninguna ley de patentes, ni tampoco infringir algún reclamo de derechos, ni discutir la validez de tales reclamos; esta sección tiene el único propósito de proteger la integridad del sistema de distribución del software libre, que está implementado por prácticas de licencia pública. Mucha gente ha hecho generosas contribuciones a la amplia gama de software distribuido bajo este sistema favoreciendo así la constante aplicación de este sistema de distribución; es decisión del autor/donador si su Programa será distribuido utilizando este u otro sistema de distribución, y la persona que recibe el software no puede obligarlo a hacer ninguna elección en particular. 

Esta sección pretende dejar muy en claro lo que se cree que será una consecuencia del resto de esta Licencia. 

\item  Si la distribución y/o el uso del Programa se restringe a algunos países ya sea por patentes, interfases protegidas por derechos de autor, el propietario original de los derechos de autor que ubica su Programa bajo esta Licencia puede agregar una restricción geográfica de distribución explícita excluyendo los países que aplique, dando como resultado que su distribución sólo se permita en los países no excluidos. En tal caso, esta Licencia incorpora la limitación como si hubiera sido escrita en el cuerpo de esta misma Licencia. 

\item La "FSF Free Software Foundation" puede publicar versiones nuevas o revisadas de la "GPL General Public License" de uno a otro momento. Estas nuevas versiones mantendrán el espíritu de la presente versión, pero pueden diferir en la inclusión de nuevos problemas o en la manera de tocar los problemas o aspectos ya presentes. 

Cada versión tendrá un número de versión que la distinga. Si el Programa especifica un número de versión para esta Licencia que aplique a él y "cualquier versión subsecuente", usted tiene la opción de seguir los términos y condiciones de dicha versión o de cualquiera de las posteriores versiones publicadas por la "FSF". Si el programa no especifica una versión en especial de esta Licencia, usted puede elegir entre cualquiera de las versiones que han sido publicadas por la "FSF". 

\item  Si usted desea incorporar partes del Programa en otros Programas de software libre cuyas condiciones de distribución sean distintas, deberá escribir al autor solicitando su autorización. Para programas de software protegidas por la "FSF Free Software Foundation", deberá escribir a la "FSF" solicitando autorización, en ocasiones hacemos excepciones. Nuestra decisión será guiada por dos metas principales: 

Mantener el estado de libertad de todos los derivados de nuestro software libre 

Promover el uso comunitario y compartido del software en general 


\textbf{INEXISTENCIA DE GARANTÍA}

\item  DEBIDO A QUE EL PROGRAMA SE OTORGA LIBRE DE CARGOS Y REGALÍAS, NO EXISTE NINGUNA GARANTÍA PARA EL MISMO HASTA DONDE LO PERMITA LA LEY APLICABLE. A EXCEPCIÓN DE QUE SE INDIQUE OTRA COSA, LOS PROPIETARIOS DE LOS DERECHOS DE AUTOR PROPORCIONAN EL PROGRAMA "COMO ES" SIN NINGUNA GARANTÍA DE NINGÚN TIPO, YA SEA EXPLICITA O IMPLÍCITA, INCLUYENDO, PERO NO LIMITADA A, LAS GARANTÍAS QUE IMPLICA EL MERCADEO Y EJERCICIO DE UN PROPÓSITO EN PARTICULAR. CUALQUIER RIESGO DEBIDO A LA CALIDAD Y DESEMPEÑO DEL PROGRAMA ES TOMADO COMPLETAMENTE POR USTED. SI EL SOFTWARE MUESTRA ALGÚN DEFECTO, USTED CUBRIRÁ LOS COSTOS DE CUALQUIER SERVICIO, REPARACIÓN O CORRECCIÓN DE SUS EQUIPOS Y/O SOFTWARE QUE REQUIERA. 
12. EN NINGÚN CASO NI BAJO NINGUNA CIRCUNSTANCIA EXCEPTO BAJO SOLICITUD DE LA LEY O DE COMÚN ACUERDO POR ESCRITO, NINGÚN PROPIETARIO DE LOS DERECHOS DE AUTOR NI TERCERAS PERSONAS QUE PUDIERAN MODIFICAR Y/O REDISTRIBUIR EL PROGRAMA COMO SE PERMITE ARRIBA, SERÁN RESPONSABLES DE LOS DAÑOS CORRESPONDIENTES AL USO O IMPOSIBILIDAD DE USAR EL PROGRAMA, SIN IMPORTAR SI SON DAÑOS GENERALES, ESPECIALES, INCIDENTALES O CONSEQUENTES CORRESPONDIENTES AL USO O IMPOSIBILIDAD DE USAR EL PROGRAMA (INCLUYENDO PERO NO LIMITADO A LA PERDIDA DE INFORMACIÓN O DETERIORO DE LA MISMA AFECTANDOLO A USTED, A TERCERAS PERSONAS QUE SEA POR FALLAS EN LA OPERACIÓN DEL PROGRAMA O SU INTERACCIÓN CON OTROS PROGRAMAS) INCLUSIVE SI TAL PROPIETARIO U OTRAS PERSONAS HAYAN SIDO NOTIFICADAS DE TALES FALLAS Y DE LA POSIBILIDAD DE TALES DAÑOS. 

\end{enumerate}

\textbf{FIN DE TÉRMINOS Y CONDICIONES} 

