\chapter{Declaración de Riesgos}
\label{anexo_riesgos}
El SITEM, como todo proyecto de objetivos ambiciosos, esta expuesto a una serie de contingencias que de no ser correctamente manejadas atentan la consecución de sus metas. Estos riesgos deben estar de la forma más óptima detectados, evaluados y medido su impacto. Cada vez que se avanza en el ciclo de desarrollo unos se vuelven más importantes que otros. A continuación se declaran los riesgos más importantes a los que se enfrenta el SITEM.

\begin{itemize}
\item \textbf{Baja disponibilidad de Tiempo.} 

La falta de tiempo por parte de los integrantes del proyecto debido al trabajo en paralelo de otros proyectos es un riesgo inminente que de seguro puede desencadenar en la no consecución de los objetivos. El reclutamiento de integrantes cuyo objetivo primario es el lleno de un requisito para grado pone en riesgo la continuidad en el desarrollo de algún hilo.

\textit{Métodos para la reducción del riesgo:} Es aquí cuando los métodos ágiles son de gran ayuda. El grupo de desarrollo fomenta la retroalimentación continua entre los diferentes hilos creando una especie de competencia sana en donde los componentes y prácticas más eficientes son detectadas y propagadas en las sesiones de capacitación. A partir de la versión 3.0, actualmente en desarrollo, los grupos adscritos no se les imponen un objetivo específico sino que las personas interesadas simplemente deciden que tarea es más relevante y su participación solo se tiene en cuenta a través de la producción - posicionamiento basado en el esfuerzo. 


\item \textbf{Pérdida de tiempo debido a procesos administrativos.} 

Las cargas administrativas inherentes a los procesos del proyecto consumen gran cantidad de tiempo y en ciertas circunstancias puede tomar más del esperado obstruyendo el desarrollo de las actividades en el proyecto. La dependencia de algunas actividades al formalismo burocrático desenfoca al grupo de trabajo.

\textit{Métodos para la reducción del riesgo:}  El trabajo se realiza en varias áreas y disciplinas de manera paralela brindando siempre la posibilidad de redistribuir recursos si se requiere esperar a que algún trámite administrativo se concrete.  El jefe del proyecto se encarga de realizar todos los trámites y generación de documentos necesarios con el fin de agilizar los procesos administrativos. 


\item \textbf{Falta de recursos para trabajo.} 

El mayor riesgo para la elaboración del proyecto y la consecución de los objetivos del mismo se encuentra en el no disponer de recursos para dichas tareas. Principalmente un equipo de computo capaz de soportar sistemas operativos Linux así como de  un espacio físico de trabajo dónde se encuentre dicho equipo y en dónde los integrantes del proyecto puedan realizar sus actividades y centralizar información.

\textit{Métodos para la reducción del riesgo:}  Cada uno de los integrantes del proyecto está en capacidad de realizar las tareas que le corresponden por su cuenta fomentando el teletrabajo. El jefe del proyecto se encargo de realizar la centralización del trabajo y la información por medio del préstamo al proyecto de recursos propios. En fases posteriores el grupo de investigación recibe apoyo del Centro de Investigaciones y Desarrollo Científico de la Universidad Distrital con lo que se adquiere un servidor con acceso público a Internet.



\item \textbf{Falta de información y demora en la consecución de la misma.}

El proyecto requiere y se basa en cierta información proveniente de diversas fuentes como publicaciones, manuales y estudios realizados dentro y fuera del grupo GITEM; la demora o imposibilidad de consecución de información calificada se convierte en un riesgo para el desarrollo normal del proyecto.

\textit{Métodos para la reducción de riesgo:} Se realiza desde el comienzo del proyecto un listado de información necesario e indispensable y se inicia la búsqueda y consecución de la misma desde el inicio.  El jefe de proyecto realiza los procedimientos requeridos con el fin de contar con los resultados de estudios previos en el interior del GITEM. Un cúmulo documental se mantiene disponible en todo momento para la comunidad de desarrollo, las jornadas de capacitación y el despliegue de presentaciones focalizadas es también de gran ayuda.

\item \textbf{Falla en la dirección del proyecto.}

Existe la posibilidad de que el director del proyecto o de alguno de sus ayudantes cometa algún error o falla atentando de ésta manera con el proceso natural de desarrollo.

\textit{Métodos para la reducción de riesgo:}  Se realiza un seguimiento continuo a todas las tareas asociadas a la dirección del proyecto. El concepto y objetivos del SITEM plasmados en su documento de Visión se han convertido en un artefacto de consulta continua y recurrente.

\item \textbf{Valoración nula o errónea de los resultados.} 

Al terminar una iteración dentro del proceso de desarrollo es posible que no se obtengan los resultados esperados, el exceso de planificación y la rigidez frente al cambio puede generar que se subvaloren los resultados obtenidos y, con la excusa de no cambiar un documento impreso y empastado, sacrificar nuevas ideas y mejoras.

\textit{Métodos para la reducción de riesgo:} Se ha difundido claramente en el grupo de desarrollo el hecho de que todos los artefactos están en construcción; nunca se valora uno de ellos como la versión definitiva. El uso de términos correctos como las versiones estables permite que los entes de control al interior del grupo puedan realizar métricas a corto plaza que en sumatoria pueden arrojar datos valiosos en cuanto al avance del proyecto. Los objetivos de los trabajos de grado pueden ser modificados respecto a los anteproyectos siempre y cuando se conserve el espíritu de los mismos. En la práctica se utiliza Cervisia, un sistema de CVS, que permite el seguimiento del proyecto, sus resultados y sus errores.

\item \textbf{Falta de motivación de los integrantes del proyecto.} 

Debido a la posibilidad de que los integrantes del grupo en ciertos momentos y etapas del desarrollo del proyecto carezcan de la motivación necesaria, se corre el riesgo de que las actividades asignadas no se realicen o se realicen de una forma no satisfactoria. Usualmente esto es fruto de reclutamientos forzosos de personal.

\textit{Métodos para la reducción de riesgo:} A integrantes de fases preliminares se les dicta capacitaciones en los métodos de desarrollo ágiles y se les descarga en gran medida la responsabilidad de documentar el desarrollo - enfoque 100\% centrado en el código. Para el cumplimiento de las tareas y actividades se asigna un tiempo adicional con el objetivo de que los integrantes del grupo resolvieran sus dudas y preocupaciones sobre el proyecto y así atacar una determinada tarea entre todos los integrantes.

En la actualidad el proyecto es de carácter abierto y cualquier persona puede; sin una intermediación directa del grupo de investigación, realizar trabajos sobre el mismo. Los desarrollos que deseen ser validados podrán solicitar dicha certificación al grupo de desarrollo principal el cuál sopesará los resultados y en algunos casos concretos el proceso de desarrollo. Es decir, a partir de la versión 3.0 del SITEM ningún nuevo integrante es regulado por las políticas del grupo principal de desarrollo, es una decisión individual el seguimiento o no de los lineamientos.

\item \textbf{Imposibilidad total de trabajo.} 

Posiblemente por motivos externos al proyecto como paros, vacaciones o cierres, es posible que el trabajo se detenga totalmente.

\textit{Métodos para la reducción de riesgo:} Teletrabajo, depreciación de la comunicación persona a persona y uso intensivo de herramientas de trabajo en grupo dentro de la plataforma del SITEM. Especialmente las de primer nivel como correo electrónico, canales IRC y Foros. Las de segundo nivel tales como chat y videoconferencia en tiempo real han sido poco utilizadas para minimizar el riesgo colateral de segregación tecnológica. 

\item \textbf{Capacitación deficiente.} 

Es posible que algún miembro del proyecto asignado a alguna tarea no posea todos los conocimientos necesarios para realizarla.

\textit{Métodos para la reducción de riesgo:} Al interior del SITEM se han creado cursos en línea con contenidos actualizados a los que los integrantes podrán acceder de forma asíncrona a través de Internet. En los foros se guarda información relevante a los problemas en el desarrollo de tareas con el objetivo de que todo el equipo de trabajo pueda encontrar soluciones basadas en la colaboración. El jefe del proyecto puede asignar recursos enfocados en la capacitación de ciertas áreas específicas.

\item \textbf{Mal funcionamiento de herramientas de software y hardware.} 

La posibilidad de fallo en los equipos de trabajo y de los recursos utilizados puede causar la para y perdida de tiempo.

\textit{Métodos para la reducción de riesgo:} En el grupo de desarrollo principal existe un plan de mantenimiento a corto plazo que contempla actividades para el aseguramiento de la integridad de la plataforma de hardware y de los datos asociados al proyecto. Con la autorización para desplegar el hilo principal en servidores de aplicaciones se tendrá un mayor respaldo.

\item \textit{No seguimiento del plan de trabajo.} 

Las libertades propias de cada integrante del grupo además del conjunto formado por todos los otros riesgos pueden causar un desarrollo no planeado. En sí este riesgo solo es evidente cuando se captan recursos que van enfocados al cumplimiento de objetivos específicos, a partir de la versión 3.0 el grupo principal será el encargado de dictaminar los alcances de los nuevos releases.

\textit{Método para la reducción de riesgo:}  Uno o más de los integrantes debe actuar como ingeniero de procesos y se encargará de verificar el cumplimiento del plan de trabajo.

\item \textbf{Desacuerdo entre los integrantes del grupo.} 

La posibilidad de desacuerdos de trabajo o personales puede darse al interior del proyecto. 

\textit{Método para la reducción de riesgo:} En el caso de desarrollo novedoso los integrantes en desacuerdo deberán seguir las reglas de contribución concurrente. Para todos los demás casos - procedimiento depreciado - el jefe de proyecto y el ingeniero de procesos deberán encargarse de tomar cualquier decisión con miras al cumplimiento de los objetivos.

\item \textbf{Desviación de los recursos de trabajo.} 

Es posible que los recursos obtenidos para el desarrollo del proyecto sean utilizados para otras actividades que no involucran la obtención de objetivos específicos.

\textit{Método para la reducción de riesgo:}  El ingeniero de procesos verifica el correcto cumplimiento del plan de trabajo así como la correcta utilización de los recursos.  Se realiza un historial de utilización de los recursos.

\item \textbf{Otros riesgos.} 

Existen otros riesgos como la renuncia al proyecto de algún miembro, el cambió de los objetivos del proyecto, cambió en las características del mercado y eventos externos que afecten el normal desarrollo del proyecto.

\textit{Método para la reducción de riesgo:}  El proceso de desarrollo ágil - con elementos claves del proceso unificado, la utilización de artefactos, el desarrollo centrado en la persona y el compromiso con la filosofía general de la libertad y el derecho a la información.
\end{itemize}

