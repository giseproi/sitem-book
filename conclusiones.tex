\chapter{Conclusiones y Recomendaciones}

La aproximación basada en nodos de las redes de eSalud permite que el grupo de investigación adopte los principios de la orientación a objetos para el diseño de redes de eSalud. La interpretación de los servicios médicos como agregados de nodos, ha puesto en evidencia la necesidad de definir las interfaces de cada componente para asegurar su pertenencia a una determinada red. Este enfoque basado en nodos con interfaces definidas ha permitido que el grupo pueda valorar el nivel de capacidad de los nodos y definir cuál de ellos presenta un mayor potencial. 
Con el desarrollo del proyecto se consolida una descripción de arquitectura para gestionar un  modelo de nodos potencialmente útiles en redes de eSalud. También se brinda un conjunto de prototipos que permite al grupo de investigación obtener un ecosistema emergente para la gestión de la información en sus equipos especializados.

La federación de aplicaciones de software libre y de código abierto es una solución óptima para alcanzar funcionalidad invirtiendo bajos recursos. Esto, unido a la decisión de centrar el trabajo en la arquitectura ha permitido que el desarrollo avance incrementalmente, aunque se tenga un bajo tiempo de retención de personal. 

El uso de un marco de desarrollo simple y específico ha demostrado beneficios frente a utilizar marcos genéricos. La continuación del uso del marco de trabajo por parte de ex-integrantes del equipo en proyectos diferentes, indica un adecuado nivel de madurez que potencia la colaboración entre programadores.S e debe complementar el empleo de un marco de desarrollo con el uso de servicios web y APIs para fomentar la interoperabilidad y adaptabilidad de la solución federada.

La completa inmersión de los interesados en el proceso de desarrollo es clave para garantizar un avance. La gestión de la expectativa es importante sobre todo en este proyecto en donde la evolución de la descripción de la arquitectura es mucho más rápida que el desarrollo y puesta en producción de la solución.

La validez de la arquitectura propuesta, así como los módulos prototipo han sido endógenas y se requiere realizar una fase de transición en donde se pueda alcanzar un mayor grupo de usuarios.

La plataforma que se requiere para desplegar el sistema no es compleja de administrar pues está basada en virtualización. No obstante, aunque en la actualidad grandes proveedores ofrecen infraestructura como servicio (IaaS), por cuestiones de presupuesto - y trámites internos, no se ha podido explotar esta opción. Por esta razón la solución aún se encuentra en una intranet. 

Desafortunadamente, tareas tan simples como configurar una entrada DNS, crear reglas en un firewall u obtener acceso vía SSH se han convertido en verdaderas pesadillas pues el proyecto depende de las unidades especializadas de la institución. Las estrategias de contingencia y recuperación de desastres han demostrado ser efectivas, pues por errores "involuntarios" de los administradores externos de la plataforma, en varias ocasiones ha tocado reconfigurar el sistema a partir de copias de seguridad. En este escenario es claro que la Universidad Distrital Francisco José de Caldas no cuenta con una organización que permita el despliegue de soluciones fruto de investigación en un ambiente de alta disponibilidad.
