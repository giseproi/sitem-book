\chapter{Análisis de Necesidad de un Bus de Servicio Empresarial}
\label{appendix:lista_de_chequeo_ESB}

Para determinar si el proyecto se beneficiaba de un Bus de servicio Empresarial, se respondió la Lista de Chequeo propuesta por Ross Mason del proyecto Mule:

\section{Capacidad de Integración}

¿Está integrando tres (3) o más aplicaciones o servicios?

Respuesta: SI.

¿Tiene más de diez (10) aplicaciones para integrar?

Respuesta: NO

\section{Protocolo de Comunicación}
¿Requiere usar más de un protocolo de comunicación?

Respuesta: NO.

\section{Capacidades especiales de mensajería}
¿Necesita capacidad para el enrutamiento de mensajes tales como bifurcación o agrupación de flujos de mensajes; o enrutamiento basado en el contenido?

Respuesta: NO.

\section{Orientación a Servicios}

¿Necesita publicar servicios que deban ser consumidos por otras aplicaciones?

Respuesta: SI

\section{Escalabilidad vs Simplificación}
¿Realmente necesita la escalabilidad provista por un ESB?

Respuesta: NO DETERMINADO

¿Conoce el alcance de su aplicación, puede iniciar sin requerir un ESB?

Respuesta: SI


\chapter {Aspectos Relativos al Despliegue}

El despliegue de la solución requiere una plataforma tecnológica adecuada tal como se muestra en el anexo \ref{modelo_despliegue} para garantizar un óptimo servicio a los potenciales usuarios \footnote{La versión actual esta desplegada en servidores de la intranet universitaria}. Las herramientas básicas se encuentran disponibles para que el grupo de investigación convoque a las entidades de salud, profesionales en el área de la tecnología, especialistas en medicina y fabricantes - distribuidores - de dispositivos médicos que participaron en el Estudio Red de Telemedicina Bogotá \cite{aparicio2000} para que de forma conjunta enriquezcan la base de información en el subproceso de prueba piloto (presupuestada como Fase IV).

Basado en el módulo de generación de herramientas para la recolección de información, anexo \ref{manual_usuario}, el grupo aplicará diferentes instrumentos entre los que se incluyen:

\begin{itemize}
\item Entrevistas: Con el objeto de acordar las directrices a seguir tanto con los operadores de telecomunicaciones, como los prestadores del servicio de salud que participaron de la fase I de recolección de información con el propósito de socializar los resultados del proyecto e invitarlos para que editen y actualicen sus datos obteniendo beneficios incrementalmente. Dichos servicios van desde la disponibilidad de un mapa de sedes, servicios y profesionales hasta la consolidación de información para la gestión de sus redes tecnológicas primarias.

\item Encuestas: Para identificar nuevos requerimientos de servicios que puedan ser desplegados en la plataforma propuesta. Además, estos instrumentos permitirán medir el impacto en la cobertura de los servicios y de participación de las instituciones objeto de la investigación de campo.
\end{itemize}

Las unidades de recolección de información mostradas en el anexo \ref{formulario_preliminar}, se han adaptado de aquellas propuestas por el proyecto europeo HERMES. \footnote{Proyecto de investigación a tres años, financiado por la Comunidad Económica Europea y que cumplió sus objetivos hacia principios del milenio dejando como resultado un conjunto de preguntas básicas que apoyan los procesos de implementación de soluciones médicas apoyadas en las TIC.} El modelo investigación evaluativa que continua en la fase IV del proyecto pretende medir la evolución del nivel de servicios de salud prestados con el apoyo de TIC comparando sucesivamente el modo de operación encontrado entre el año 2000 y 2005 con aquel encontrado entre el año 2008 y 2011 luego que algunas entidades interactúen con el SITEM.

\subsection {Fuentes de Información Primaria}

Para la carga de información inicial en el SITEM se utilizan los resultados del estudio de campo realizado en varias instituciones de carácter público y privado. Dicho resultados se encuentran consignados en sendas tesis en formato digital e impresos disponibles en la biblioteca de la Universidad Distrital y archivo del grupo de investigación:

\begin{itemize}
 \item Hospital Rafael Uribe Uribe.\cite{guarin2003}
 \item Hospital San Pedro Claver.\cite{ardila2001},\cite{rozo2002}
 \item Hospital Simón Bolívar. \cite{acero2002}
 \item Hospital El Tunal. \cite{ruiz2002}\cite{duque2002}
 \item Hospital La Victoria.\cite{barrero2000}
 \item Hospital San José.\cite{gonzalez2002}
\end{itemize}