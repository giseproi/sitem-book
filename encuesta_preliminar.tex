\chapter{Instrumento Preliminar para la Fase de Transición}
\label{formulario_preliminar}

\section{Presentación}

El presente instrumento agrupa de una manera lógica los aspectos necesarios para identificar el estado y potencialidad de los servicios de salud que hacen uso intensivo de las TIC. El contenido está basado en aquel utilizado por el proyecto HERMES, modelo completamente depurado que ha servido de base para la recolección de información de varios proyectos de telemedicina emprendidos en la Comunidad Económica Europea (CEE). Así mismo se le han adicionado algunos ítem aplicables al contexto colombiano.

\subsection{Indicaciones Generales}

El cuestionario deberá ser respondido como fase posterior a la etapa de divulgación del proyecto y una vez se haya conformado un comité, o en su defecto, se haya designado un encargado, por parte de la institución objetivo. Es necesario que en el proceso los miembros del proyecto sirvan de apoyo resolviendo de una manera exacta las dudas que surjan en el ámbito tecnológico (telesalud, estándares de telecomunicaciones y otros). De igual forma se deben realimentar de la información especializada que puedan capturar de los comités. Para apoyar estas tareas podrán hacer uso de las herramientas de trabajo en grupo que se encuentran en el Portal SITEM.

La profundidad con que se respondan las cuestiones garantiza que la institución evalúe su situación actual frente a la posible implementación de servicios de salud prestados en la modalidad de Telemédicina.

El cuestionario esta disponible para ser diligenciado en línea de tal forma que podrá ser resuelto en cualquier orden y anexando cualquier tipo de información que se considere conveniente para aclarar y/o complementar una respuesta. 

Las cuestiones deberán ser manejadas por un especialista en la materia aunque esto conduzca a que los encargados del proyecto jueguen un papel meramente coordinador de actividades. Todas las cuestiones deben ser tenidas en cuenta aunque se hayan tratado tácitamente en respuestas anteriores.

\subsection{Contexto Clínico}

Las preguntas que tienen que ver con los servicios clínicos están incluidos en este cuestionario dado que uno de sus objetivos es dar un conocimiento global del sistema de telemedicina y sus características. 

El cuestionario puede ser usado para ver el estado actual de los aspectos clínicos y para evaluar cuales mejoras deben ser hechas. Las preguntas no pretenden definir competencias clínicas y son por lo tanto abiertas y basadas en la familia de estándares de calidad ISO 9000. Estas características están en consonancia con el concepto de Administración con Calidad Total  (TQM):

\begin{enumerate}
\item Definición de necesidades
\item Planificación para cubrir necesidades.
\item Controles para cubrir los estándares.
\item Mejora continua
\end{enumerate}

Un gran número de factores deben ser tomados en mente en consideración a los tópicos relacionados en esta sección:

Los estándares médicos deben definir el contexto sobre el cual el sistema de telemedicina ha de ser introducido, no solamente las necesidades que han de ser atendidas por dicho servicio. Es decir, la definición de estándares médicos y de un conjunto de directrices para asegurar la más alta calidad en la atención debe ser una actividad independiente del sistema de telemedicina a implementar.

Cualquier servicio medico en la modalidad de telemedicina que se quiera implementar en un ambiente donde existe un conjunto aceptado y aplicado de estándares médicos y directrices, debe estar completamente estructurado para que sea compatible. Si no lo es debe estar lo suficientemente documentado y la necesidad de su aplicación claramente dilucidada.

Es necesario tener en cuenta las siguientes definiciones:

\begin{description}
\item[Evento de Atención en Salud]
Un contacto para el cuidado de la salud que requiere que un servicio clínico sea aplicado.

\item [Servicio Clínico]
Uno a más planes de salud autorizados e implementados para poder llenar los requerimientos del evento de atención en salud.

\item [Plan de salud]
Uno o más procedimientos clínicos asociados a estándares médicos y directrices que se utilizan para poder cubrir un servicio médico.

\item [Procedimiento clínico]
Una o más actividades de tratamiento en salud.

\item [Directriz médica]
Un informe sistemáticamente desarrollado, el cual asiste en la toma de decisiones acerca de los procedimientos a seguir para una condición clínica específica. 

\item [Estándar médico] 
Un documento, establecido por consenso y aprobado por un cuerpo consultor organizado, el cual provee para el uso común y repetido de las pautas o las características de actividades de cuidados clínicos (o sus resultados)  y su objetivo principal es el de alcanzar el grado óptimo de orden en un contexto de medicina dado.
\end{description}

\subsubsection{Unidades de Recolección de Información}


\begin{enumerate}
\item ¿Cuales son los servicios de salud por modalidad telemédicina que deben ser implementados?

Describa sus ideas/necesidades para el uso de la telemedicina en su ambiente de trabajo 
\begin{itemize}
\item ¿Cuales son los servicios clínicos que serán afectados o reemplazados por el nuevo sistema de telemedicina?
\item ¿Cuales organizaciones/departamentos clínicos están directamente involucrados en el servicio? 
\end{itemize}

\item ¿Cuales son los requerimientos de los usuarios?
\begin{itemize}
\item  ¿Quienes son los usuarios? 

Quien está involucrado.

\item ¿Qué hacen ellos?
roles/actividades 
Usuarios Cliente (pacientes) 
Usuarios participantes (doctores, enfermeros, paramédicos,  personal de soporte, etc.) 

\item  ¿Que método era usado para obtener los requerimientos clínicos de los usuarios? 

Ejemplo: ¿ Existe un grupo de usuarios, incluyendo un representante del paciente, donde se use un cuestionario o se realice una encuesta?

\item ¿Cuales son los eventos/episodios clínicos que serán dirigidos en el sistema de telemedicina? 

\item ¿Que señales, incluyendo signos vitales, necesitarán ser monitorizados? 

\item ¿Qué equipo es necesario para monitorizar dichas variables? 

\item ¿Qué procedimientos (incluyendo entrenamiento) son requeridos para implementar la monitoria de las variables? 

\item ¿Quién interpretará la información monitorizada? (Especificar el nivel profesional) 

\item ¿Que factores son requeridos para garantizar la seguridad del paciente, exactitud y funcionalidad de los equipos, y aceptación por parte del usuario en la monitoria y otros procedimientos?

\end{itemize}

\item  ¿Cuales son los procedimientos de servicios clínicos que podrán ser afectados por la implementación de la telemedicina?
 
Ejemplo: El acceso al servicio,  referencia, admisión, descarga y procedimientos de seguimiento. 

Describir los eventos y procedimientos existentes, además de las particularidades del ambiente local (específicas) 

Describir una aproximación al nuevo sistema de telemedicina (tomando en consideración condiciones y directrices locales) 

Describir la estrategia  con la cual se obtendrá la aprobación local y extendida de los cambios. 

\item ¿Cuales son los estándares/ pautas relacionadas con el sistema de telemedicina?

Pautas existentes que deben ser conservadas en el sistema de telemedicina. 

Pautas que deben ser reemplazadas. 

\begin{itemize}
\item ¿Existen nuevas directrices que deban ser creadas para implementar el sistema de telemedicina? 

\item ¿Que proceso de desarrollo de directrices debería ser usado?

\end{itemize}

\item ¿Cuales son los recursos humanos requeridos por el sistema de telemedicina? 

\begin{itemize}
\item ¿El sistema de telemedicina puede ser implementado y operado por el personal existente? 

\item ¿Nuevo personal debe ser entrenado?
\end{itemize}

\item ¿Cuales son las necesidades de entrenamiento para el sistema de telemedicina?

En telemática, telemedicina, procedimientos de calidad, análisis de requerimientos, indicadores de resultados y desempeño, etc. 

Tipo de entrenamiento requerido.

Tipo y grado del personal.
\end{enumerate}

\subsection{ Contexto de servicio / relación con el entorno}

Es necesario definir una estrategia de desarrollo para poder garantizar que la solución en Telemedicina encaje perfectamente en el ambiente existente de servicios en salud. Los servicios Telemédicos usualmente se montan sobre alguno de los siguientes escenarios: Creación de nuevos servicios Telemédicos ó servicio telemédico que reemplazará un servicio no telemédico existente.

\subsubsection{Unidades de Recolección de Información}

\begin{itemize}
\item Restricciones legales. 

La pregunta explora las restricciones legales, éticas y sociales sobre el servicio y aquellos que participan en él. Las restricciones éticas pueden estar enmarcadas en leyes pero no siempre sucede, ej. Los médicos tienen límites éticos promulgados por organismos internacionales pero razones culturales pueden flexibilizar o endurecer dichas directrices,


\begin{enumerate}
\item ¿Cuales son las restricciones legales de los servicios telemédicos?

\begin{itemize}
\item Con respecto a las trasmisiones de datos. 
\item Con respecto a la protección de datos.
\item Con respecto a la responsabilidad en los productos y el seguimiento de estándares.
\item Con respecto a la libertad de información / privacidad 
\item Con respecto a la responsabilidad personal y organizativa. 
\item Con respecto a las entidades en las cuales se siguen responsabilidades legales. 
\item ¿Que organismos son responsables de la dirección legal? 
\end{itemize}

\item ¿Cuales son las consideraciones éticas

\begin{itemize}
\item ¿Para quien están dirigidas las restricciones éticas? 
\item ¿Con respecto a qué se definen dichas restricciones? 
\item ¿Que organismos son los encargados de la dirección en cuanto a cuestiones éticas?
\end{itemize}


\item ¿Cuales son los factores culturales y sociales a ser considerados?

\begin{itemize}
\item Consideraciones relacionadas con las raza. 
\item Consideraciones relacionadas con creencias culturales. 
\item Consideraciones relacionadas con el nivel educativo de los usuarios. 
\item Consideraciones relacionadas con el grupo socio económico al cual pertenece el usuario. 
\item Consideraciones relacionados con la ubicación geográfica de los usuarios. 
\item Consideraciones relacionadas con la edad de los usuarios.
\end{itemize}
\end{enumerate}

\item Consecuencias organizativas del sistema de telemedicina

La siguiente pregunta tiene que ver con los efectos que tendrá la introducción de un servicio de salud por modalidad de Telemedicina en la organización. 


\begin{enumerate}
\item ¿Que consecuencias tendrá en su organización la introducción del sistema de telemedicina? 

\begin{itemize}
\item Sobre la practica actual del trabajo.
\item Sobre los estándares locales. 
\item Sobre las inversiones existentes.
\end{itemize}

\item ¿Que características técnicas u organizativas requieren adaptación? 

\begin{itemize}
\item Practica actual del trabajo.
\item Estándares locales.
\item Inversiones existentes.
\item Políticas administrativas.
\item Maquinaria consultiva.
\end{itemize}

\end{enumerate}

\item ¿Que organizaciones/ contratistas deben interactuar para proveer el servicio de salud por modalidad de Telemedicina? 

\begin{itemize}
\item Organizaciones gubernamentales 
\item Administración regional de salud. 
\item Organizaciones públicas. 
\item Organizaciones profesionales. 
\item Organizaciones voluntarias (cruz roja, defensa civil, etc) 
\item Compañías de telecomunicaciones (Proveedores de servicios de interconexión) 
\item Compañías de seguros.
\item Otras compañías.
\end{itemize}
\end{itemize}

\subsection{Consideraciones Tecnológicas}

Esta sección esta relacionada con la definición de los recursos técnicos y de infraestructura necesarios para la provisión del sistema de telemedicina.

\subsubsection{Unidades de Recolección de Información}

\begin{enumerate}
\item Requerimientos de hardware

\begin{itemize}
\item ¿ Cuales plataformas de hardware son requeridas? (tanto como para proveer el servicio de salud en la modalidad de Telemedicina como para la utilización del mismo)

Ej.  PC, UNIX,  
\item  ¿Que tipo de almacenamiento se requiere? 
\begin{itemize}
\item  De acuerdo a la cantidad de información.
\item  De acuerdo a las políticas de seguridad.
\item  De acuerdo a la velocidad de búsqueda de la información.
\end{itemize}

\item  ¿Que tecnología de visualización se requiere?

\item  ¿Cuales son los requerimientos de procesamiento y como puede ser medido?
 

Ejemplo: Velocidad, procesamiento en paralelo. Medición en MFlops, Mips, Whetstones, Dhrystones o otras medidas estándar. 

\item ¿Que requerimientos especiales deben tener los equipos del usuario final?

Ej. Unidades móviles u otros dispositivos de usuario específicos (para captura de información, análisis, etc)

\item ¿Que puertos de comunicaciones deberán estar disponibles? 

Ejemplo: Ethernet, V32, SCSI, RS 232, USB, conectores especiales a otros equipos.

\item ¿Que tecnologías de adquisición, entrada o salida de datos son requeridas? 

Ej. Dispositivos de monitoria, modems, impresores, OCR, etc.

\item ¿Existen en su institución dispositivos que cumplan los requerimientos?

\end{itemize}


\item  Software 

\begin{itemize}
\item ¿Cuales sistemas operativos se requiere? 
\item ¿ Que herramientas de desarrollo son requeridas para desarrollar aplicaciones específicas para el sistema de telemedicina? 

Ejemplo: Lenguajes de programación, compiladores, enlazadores, depuradores, editores, herramientas CASE.
\end{itemize}

\item  ¿ Que infraestructura de comunicaciones se requiere? 
Ejemplo: Hardware de red, Software de red, 

\item  ¿ Que enlaces de comunicaciones se requiere? 
Ejemplo: Hardware (modems, bridges, switch, router,etc) 

\begin{itemize}
\item ¿ Que formatos? 
\item ¿ Que velocidad en los tiempos de respuesta? 
\item ¿ Que estándares / protocolos?
\item ¿ Que ancho de banda?
\end{itemize}

\item  Software aplicativo. 

\begin{itemize}
\item ¿ Cuales son los requerimientos para KBS, DSS? 
\item ¿ Cuales son los requerimientos para los sistemas de bases de datos? 
\item ¿ Cuales son los requerimientos para otro tipo de software aplicativo?
Ejemplo: ¿Existe software comercial, GNU, GPL que cubra las necesidades? 
\end{itemize}

\item  Requerimientos en infraestructura. 

\begin{itemize}
\item ¿ Que mobiliario se requiere? 
\item ¿ Que requerimientos eléctricos deben ser cumplidos?

Voltaje, corrientes, regulación, aislamiento, sistemas de tierra, etc.

\item ¿ Que temperatura, iluminación o humedad se requiere? 
\item ¿ Que cambios deben realizarse a los sitios existentes?
Ejemplo: Cableado, requerimientos de espacio, etc.
\end{itemize}

\item  ¿ Cuales metodologías están implementadas (o deben ser adoptadas) para cumplir los requerimientos tecnológicos y de infraestructura del servicio?                                                                                          \end{enumerate}


\subsection{Consideraciones de Calidad}

Esta sección contiene preguntas que deben ser contestadas para garantizar un servicio que tome en cuenta los factores de calidad en todas las etapas del proyecto.

\subsubsection{Unidades de Recolección de Información}

\begin{enumerate}
\item Aseguramiento de la calidad.
\begin{itemize}
\item ¿Cuales son las consideraciones de calidad a tener en cuenta en el sistema tele médico?

Ejemplo:. Mejoramiento de los mecanismos de cuidados para el paciente, mejoramiento de la eficiencia, reducción de costos, etc.

\item ¿Cuales métodos de administración con calidad total (TQM) existen o deben ser implementados? 

\item ¿Quien es responsable de la supervisión de los servicios médicos / de enfermería,  servicios técnicos, servicios administrativos y cuales son sus roles en el servicio telemédico?

\item ¿Como podría ser validado el sistema de telemedicina? 

Definir factores, variables, criterios e indicadores.

\item ¿Existen manuales y políticas de calidad implementadas para el sistema de telemedicina?

\item ¿Como podría obtenerse aprobación de la autoridades? 

Ejemplo: Legislación Local, Nacional o certificación ISO 
\end{itemize}


\item Grupo de usuarios

\begin{itemize}
\item ¿Está o estará un grupo de usuarios establecido para el sistema de telemedicina? 
\item ¿Esta o estará disponible un manual de usuario para el sistema de telemedicina? 
\item ¿Existirá enlaces entre los diversos grupos de usuarios nacionales/mundiales de servicios de salud similares brindados por la modalidad de telemedicina?
\end{itemize}


\item Describa las estrategias que se seguirán para dar conocimiento al público del nuevo servicio telemédico

\begin{itemize}
\item Estrategias para promocionar los objetivos del servicio y la forma de acceder a él.
\item Estrategias para el material educativo que será usado en el servicio. 
\item Estrategia para la mejora, promoción de resultados e impacto del servicio.
\end{itemize}

\item ¿Como podrían evaluarse las variables de calidad?
\item ¿Cuales son los valores aceptables para las variables de calidad seleccionadas ?
\item ¿Quien es el responsable de definir las estrategias de calidad?
\item ¿ Que tipo de documentación y procedimientos son usados (o se necesitan) para asegurar que la calidad total esta implementada?
\item ¿Que criterios pueden ser utilizados para demostrar que la calidad en el servicio es máxima? 
\end{enumerate}


\subsection{Aceptación del servicio de salud brindado en la modalidad de telemedicina}

\subsubsection{Unidades de Recolección de Información}

\begin{enumerate}
\item ¿ Qué tipos de intereses (beneficios) son inherentes al establecimiento de un sistema de telemedicina y que variables deben ser consideradas? 

Tipos de interés cuyos valores deben ser considerados: 

\begin{itemize}
\item Clínicos : Doctores, enfermeras, etc.
\item Tecnológicos: Técnicos, investigadores, doctores, desarrolladores, etc.
\item Éticos.
\item Económicos.
\item Industriales: Empresas, institutos de investigación/universidades, sociedad en general.
\end{itemize}

\item  ¿Qué hace que el sistema de telemedicina sea aceptado por las personas envueltas en el proyecto?

\item ¿Cual es el costo proyectado para establecer el sistema de telemedicina?

\begin{itemize}
\item Costos Capitales Tangibles: Equipos, instalación/pruebas, Cambios estructurales (Edificios/cuartos), cambios en la organización (Procedimientos), depreciación. 
\item Costos operativos: Personal, Servicio de Comunicaciones, mantenimiento y servicio, control de calidad, capacitación de personal, jornadas de socialización.
\end{itemize}

\item ¿Quien debería pagar por el establecimiento / implementación del sistema de telemedicina? 

Entre otros puede estar:

\begin{itemize}
\item Hospital, centros atención de salud.
\item Universidad.
\item Empresas de Telecomunicaciones.
\item Fundaciones de investigación. 
\item Proveedores de servicios de salud bajo la modalidad de telemedicina, compañías de aseguramiento.
\end{itemize}


\item ¿Cómo debe ser evaluado el sistema de telemedicina?

\begin{itemize}
\item Evaluación económica.
\item Evaluación de impacto social.
\item Evaluación de impacto médico.
\item Evaluación clínica.
\item Evaluación a las comunidades de práctica.
\end{itemize}

\item ¿Que técnicas de evaluación están disponibles?

\begin{itemize}
\item Investigación Evaluativa.
\item Evaluación Integral.
\item Etnografía.
\item Análisis de costos.
\item Análisis costo/beneficio.
\item Ingeniería económica.
\end{itemize}

\end{enumerate}


\subsection{Aspectos de Construcción y Persistencia del sistema de telemedicina}

\subsubsection{Unidades de Recolección de Información}

\begin{enumerate}
\item Tiempo de vida del proyecto

\begin{itemize}
\item ¿ Cual es el tiempo de vida proyectado para el sistema de telemedicina?
\item ¿ Que etapas pueden identificadas en dicho tiempo de vida?
\item ¿Cuánto tiempo tardaría en implementarse el servicio en una nueva ubicación?
\item ¿ Cuando y cuales mejoras pueden ser anticipadas / planeadas?
\end{itemize}
\item  ¿ Que recursos son o deben estar disponibles para establecer el sistema de telemedicina? 

\begin{itemize}
\item Recursos de información ( manuales, educación, soporte)
\item Recursos financieros.
\item Personal (Habilidades, incentivos, educación, entrenamiento) 
\item Espaciales (arquitectónicos)
\end{itemize}

\item  ¿ Que recursos deben ser re – evaluados?

\begin{itemize}
\item Financieros.
\item De personal.
\item Espaciales.
\item De cultura en la organización.
\item De mecanismos de interacción con el entorno.
\end{itemize}


\item Disponibilidad de los recursos para la operación del sistema de telemedicina. 

\begin{itemize}
\item ¿En que sitios? 
\item ¿En que tiempo?.
\item ¿Quien los suple?
\item ¿A quien o que van dirigidos?
\end{itemize}

\end{enumerate}


\subsection{Políticas de la Organización}

Esta sección es relacionada con la formulación de políticas administrativas con respecto a la prestación de servicios Tele médicos.

Las políticas versan sobre los objetivos de la compañía, prácticas estandarizadas, regulaciones, estatutos, manuales de procedimientos, manuales de seguridad, códigos de ética, etc.

\begin{enumerate}
\item ¿ Que políticas / regulaciones administrativas pueden tener efecto sobre el sistema de telemedicina?

\begin{itemize}
\item Con relación con la disponibilidad y distribución de recursos.
\item Con relación a las estructuras de la organización.
\item Con relación a las personas que son afectadas
\item Con relación a la calidad (norma ISO 9000 o similares)                                                      \end{itemize}

\item ¿Que implicaciones pueden tener para el sistema de telemedicina?

\begin{itemize}
\item Con relación con la disponibilidad y distribución de recursos.
\item Con relación a las estructuras de la organización.
\item Con relación a las personas que son afectadas.
\end{itemize}

\item ¿ Cómo deben ser definidas las políticas de seguridad, manejo de información y calidad para el sistema de telemedicina?

\begin{itemize}
\item Con respecto a políticas administrativas.
\item Con respecto a la disponibilidad de fuentes de información.
\item Con respecto a la libertad de uso de la información, a la protección de datos y la confidencialidad.
\item Con respecto a la legislación.
\item Con respecto a consideraciones éticas y sociales.
\item Con respecto a la distribución de privilegios en el acceso al sistema de telemedicina. 
\item Con respecto a las políticas de calidad definidas.
\end{itemize}

\end{enumerate}

\subsection{Consideraciones acerca de la información}

Esta sección cubre la información que ha de ser parte del sistema de Telemedicina, incluye cuestiones sobre flujo de datos, estructura de datos y archivos, almacenamiento de datos, entre otras. 

Las necesidades de información del sistema y el procesamiento que se haga de ella tienen una gran implicación cuando se pretende elegir la tecnología a ser utilizada.

\subsubsection{Unidades de Recolección de Información}

\begin{enumerate}
\item ¿Que tipos de información son necesarios para el sistema telemédico? 
\begin{itemize}
\item texto, numérico, imagen, video.
\item Imágenes de rayos X, MRI.
\end{itemize}

\item Formatos de archivos y estándares de codificación. 

\begin{itemize}
\item ¿Cuales estándares de codificación deberán ser usados para tipos genéricos de datos? 

Ejemplo. MPEG, JPEG, MHEG, IA5, etc.
\item ¿Que estándares de codificación médica deberán ser usados para los elementos de información? 

Ej. Códigos READ, ICD9, ICD10, WHO.
\end{itemize}
\item ¿ Cuales son las actividades en las cuales la información está involucrada? ¿ Como y donde está siendo utilizada?
\item Calidad de la información 

\begin{itemize}
\item ¿ Con que calidad la información debe ser utilizada en las diferentes etapas del servicio. (recolección, procesamiento, transmisión y visualización)?
\item ¿Que tan relevante es la calidad en esas etapas para la exactitud de la conclusión final?
\item ¿Como puede ser maximizado el nivel de exactitud del diagnóstico final?
\end{itemize}

\item ¿Que procedimientos de control de calidad en la información son necesarios?
\item ¿ Como es el flujo de la información en el sistema de Telemedicina (dentro de las fronteras del sistema)?
\item ¿Cual es la información que fluye desde y hacia el sistema de Telemedicina?
\item ¿ que pasos deben ser tomados en cuenta para garantizar la seguridad de los datos?
\end{enumerate}
\subsection{Procesamiento de información}

\subsubsection{Unidades de Recolección de Información}

\begin{enumerate}
\item ¿Que conversiones en la información son (o deberían ser hechos) antes de su presentación ¿ (Si los datos originales son transmitidos entonces ninguna conversión deberá ser necesaria)

Ej. De-comprensión  de imágenes, decodificación, etc.

\item ¿Que procesamiento de señal es requerido?
\item ¿Cuales son las implicaciones de tales transformaciones / procesamientos?
\begin{itemize}
\item Con respecto a responsabilidades legales. 
\item Con respecto a consideraciones de calidad.
\end{itemize}

\item ¿ Cómo deberá ser presentada la información al usuario? 
\begin{itemize}
\item Requerimientos en interfase de usuario.\end{itemize}
\end{enumerate}
