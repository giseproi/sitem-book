\documentclass[11pt]{report}
\usepackage[spanish]{babel}	% Division de silabas en español.
\selectlanguage{spanish}	% En caso de error instalar texlive-collection o texlive-lang
\usepackage{ucs}		% Soporte para UNICODE
\usepackage[utf8x]{inputenc}	% Para poner acentos directamente
\usepackage[usenames]{color}    % Habilitar uso de colores por nombre
\usepackage{color}
\usepackage{colortbl}
\usepackage{sidecap}		% Capacidad de colocar leyendas al lado de las imágenes
\usepackage{appendix}		
\usepackage[letterpaper,left=3cm,top=3cm,right=3cm,bottom=3cm,nohead]{geometry} %Margenes
\usepackage[dvips]{graphicx}	% Convertir las gráficas a formato Postscripts
\parindent=0in			% Tamaño de la sangría 
\setlength{\parskip}{0.5cm}	%Espacio entre parrafos
\sffamily 			% Familia predeterminada de fuentes a utilizar en el documento
\title{Sistema de Información para la Caracterización de Proyectos de e-salud}
\author{Lilia Edith Aparicio - Paulo César Coronado}
%\includeonly{proceso_desarrollo}
\begin{document}
\DeclareGraphicsExtensions{.jpg, .png, .eps}
\renewcommand{\tablename}{Tabla}
\renewcommand{\listtablename}{Índice de tablas}
\renewcommand{\appendixname}{Anexo}
%\maketitle
\tableofcontents
\chapter*{\begin{Large}PRESENTACIÓN\end{Large}}
\addcontentsline{toc}{chapter}{PRESENTACIÓN}

En el marco del proyecto\textit{ Telemedicina Bogotá 2K}, el grupo de investigación en telemedicina de la Universidad Distrital Francisco José de Caldas (GITEM), realizó un estudio de campo para caracterizar las entidades de la Red Distrital de Salud. Este estudio permitió dar un diagnóstico del estado de la eSalud en la ciudad y refrendar que las acciones de investigación del grupo estaban alineadas con la propuesta que la Organización Mundial de la salud (OMS) y la Unión Internacional de Telecomunicaciones presentaron en las \textit{Herramientas para el Desarrollo de Estrategias Nacionales de eSalud} \cite{ituoms2012}. 

Como consecuencia de los hallazgos presentados en los informes del Ministerio de Salud y Seguridad Social (MINSALUD) \cite{minsalud2016}, de la Organización para la Cooperación y el Desarrollo Económico (OCDE) \cite{ocde2015} y el Observatorio Así Vamos en Salud \cite{oaves2017}; el grupo GITEM desarrolló la fase III del \textbf{Sistema de Información para la Caracterización de Nodos Potenciales de Redes de eSalud} con el que se propone un Portal de inteligencia analítica que apoye los procesos de definición de capacidad en proyectos de eSalud. 

OpenSITEM, como se ha denominado el sistema, es fruto del esfuerzo que realiza el grupo para sistematizar su experiencia de diagnóstico. Busca ofrecer una plataforma alternativa, independiente y emergente que colabore en la superación de las limitaciones que en materia de gestión y análisis de información tienen los equipos de trabajo encargados del diseño y desarrollo de redes de e-salud. 

En el dominio técnico, OpenSITEM es un sistema federado de aplicaciones de software libre o de código abierto que provee herramientas para analizar datos e información de los siguientes elementos que son de interés para la descripción - y definición de capacidad de, nodos potenciales de redes de e-salud: entidades de salud, servicios médicos, tecnologías de interconexión, operadores de telecomunicaciones, equipos médicos, organizaciones, profesionales, estándares, pacientes, enfermedades, medicamentos y proyectos. OpenSITEM \textit{no es un sistema de eSalud} sino de una plataforma para apoyar la definición de la capacidad que tiene un nodo para potencialmente hacer parte de una red de eSalud.

OpenSITEM propone un mecanismo para la integración de actores en el área de \textit{caracterización y análisis de capacidad} de nodos en redes de eSalud. Provee un escenario ubicuo, basado en tecnologías de la información y un modelo de trabajo colaborativo en red que propende por la construcción evolutiva de una base de información y conocimiento. Siguiendo los lineamientos de apertura en la información y el conocimiento (open philosophy), busca elaborar un \textit{recurso público} para ayudar al diseño de proyectos de e-Salud en nuestra ciudad. 

La arquitectura general del sistema y la experiencia en el desarrollo se presenta en este documento, organizado en las siguientes secciones:

\begin{itemize}
 \item \textbf{Antecedentes}: Enumera los aspectos que motivaron el proyecto y un resumen de las fases anteriores.
 \item \textbf{Estructura del Proyecto}: Muestra un resumen de los aspectos, que a consideración de los autores, son relevantes para medir el impacto presente y la potencialidad del proyecto.
 \item \textbf{Contexto Teórico}: Recopilación de aspectos teóricos que fueron de interés para abordar el desarrollo del trabajo de investigación . 
 \item \textbf{OpenSITEM}: Descripción de la arquitectura del sistema resultante haciendo énfasis en sus conceptos y propiedades fundamentales. El producto software que de manera emergente se está creando está centrado en esta descripción y guiado por los casos de uso que se presentan en este informe. Esta sección se complementa con los anexos de los modelos de requisitos, análisis y diseño, implementación, despliegue y datos.
 \item \textbf{Experiencia de Desarrollo}: Teniendo en cuenta que la plataforma de software que soporta OpenSITEM es de código abierto, se presenta una definición del método de trabajo empleado para que los grupos interesados en participar puedan integrarse a la comunidad de desarrollo. Esta sección hace especial énfasis en los aportes originales del grupo y describe las demás herramientas que se integran al modelo.
 \item \textbf{Conclusiones y Recomendaciones}: Aspectos y hallazgos que se deben considerar para la consolidación de OpenSITEM así como el desarrollo de nuevos proyectos de investigación.
\end{itemize}

\begin{figure}[!htpb]
\begin{minipage}[r]{0.8\textwidth}
\begin{flushright}
\textbf{Lilia Edith Aparicio Pico. PhD}
\\Directora GITEM+\\Grupo de Investigación en Telemedicina\\Universidad Distrital Francisco José de Caldas
\end{flushright}
\end{minipage}
\begin{minipage}[r]{0.15\textwidth}
 \includegraphics[width=20mm, height=26mm]{edith.png}
\end{minipage}
\end{figure}
\chapter{Antecedentes}

A principios del milenio, el grupo GITEM inició un estudio para abordar problemas puntuales en la implementación de servicios médicos prestados a través de medios teleinformáticos en el Distrito Capital\cite{aparicio2000}:

\begin{quote}
“En el momento no existe un diagnóstico real sobre los servicios requeridos en el área de telemedicina, razón suficiente para iniciar un trabajo de campo que establezca la situación actual de servicios médicos y la demanda real, así como la posibilidad de conocer a corto, mediano y largo plazo cuáles serían los costos de inversión que permitirían dar soluciones al problema  de cobertura.

La socialización del conocimiento alrededor de las tecnologías aplicadas al desarrollo de la medicina, es uno de los valores que lleva al éxito de soluciones efectivas en el sector salud, por tal motivo es necesario desarrollar un plan de alfabetización en el sector salud y en el sector gubernamental y académico.

En el país no existen estrategias de investigación en esta área del conocimiento para llevar a cabo un estudio real que permita dar el paso a soluciones verdaderas sobre desarrollo tecnológico o experimental para poder implementar centros de investigación en Telemedicina.

La Universidad Distrital tiene el recurso humano, el conocimiento y la experiencia científica y tecnológica, capaz de dar soluciones tangibles a estas necesidades; unida al conocimiento y experiencia de entidades como clínicas y hospitales  y con la participación de operadores de comunicaciones, puede desarrollar soluciones efectivas a los problemas de salud que afronta la sociedad colombiana.”
\end{quote}

Para facilitar el aprovechamiento del estudio, los resultados obtenidos fueron recopilados en extensos tomos en formato físico y digital. Aunque eficaz en primera instancia, los resultados obtenidos no tenían una estructura documental coherente ni un modelo de información que los caracterizara. Este hecho, a la par con el ingreso de entidades al estudio, aumentó la complejidad a la hora de generar estudios comparativos o de apoyo a la toma de decisiones, teniéndose en muchos casos información faltante, redundante e innecesaria para el proyecto. Dado que la muestra objeto de estudio es intrínsecamente dinámica, cualquier cambio de la condiciones iniciales es difícilmente reflejado, quedando en poco tiempo la información desactualizada. En ese escenario, la labor de articular las prestaciones de las organizaciones suponía un proceso lento y la estrategia de gestión de información empleada en el estudio comenzó a mostrar debilidades.

Respondiendo a este nuevo contexto problémico se creó al interior del grupo un proyecto denominado \textbf{Sistema de Información para la Caracterización de Proyectos de eSalud}, que en sus primeras fases de desarrollo dio solución parcial al marcar las pautas hacia la integración de información para el \textbf{GITEM}. En paralelo el grupo de investigación implementó, en convenio con \textit{Colciencias}, el Sistema de Referencia y Contrarreferencia para el Distrito Capital, utilizando herramientas de desarrollo propietarias específicamente el middleware .NET de Microsoft con lo que el grupo adquirió experiencia en la desarrollo de aplicaciones siguiendo metodologías efectivas para la construcción de software.

\section{Sistema de Información para la Caracterización de Proyectos de eSalud}

Problemas asociados con malas interpretaciones del concepto de \textit{licencia a perpetuidad} del software propietario, plantearon la necesidad de realizar investigación relacionada con el desarrollo de software distribuido, interoperable y robusto basado en la filosofía del software libre y del código abierto- es decir, que abordara sus potencialidades y encarara sus debilidades en un entorno de trabajo de corte industrial. Bajo este enfoque nació SITEM.

SITEM es un proyecto de investigación asociado a un holotipo proyectivo\cite{hurtado2000}, que tiene como impacto esperado el apoyo a labores de consultoría y diseño de redes de eSalud. Se desarrolla con un método de trabajo que surge de la metodología de OpenUP\cite{balduino2010} en donde se prioriza las disciplinas de Requisitos y Arquitectura, teniendo en cuenta el riesgo que se tiene en el grupo de investigación para consolidar equipos de trabajo por periodos de tiempo superiores a tres meses.

El proyecto tiene como particularidad el tener que cumplir con los requisitos empleando reducidos recursos técnicos y financieros, los cuales deben ser administrados dentro de un ambiente de alta regulación burocrática. El escenario común ha sido el bajo tiempo de permanencia de los integrantes, la ejecución constante de tareas de capacitación, el uso intensivo de tecnologías de la comunicación, el teletrabajo  y la escasa interacción persona a persona. Con nuestra experiencia comprobamos la máxima de Larman \cite{larman2003} : “Rápido, barato, bueno: elija dos cualquiera”; y dejando a un lado las fuertes esperanzas, sacrificamos el único aspectos que podíamos y la rapidez en que versiones estables del proyecto habrían de ver la luz se reduciría significativamente. No obstante, los conceptos fundamentales de desarrollo de aplicaciones de software libre \cite{raymond} han aportado varias claves para minimizar el impacto de este sacrificio.

\begin{figure}
 \centering
 \includegraphics[width=156mm, height=195mm]{modelo_fases.png}
 \caption{Aproximación incremental a un sistema de Gestión de Conocimiento}
 \label{modelo1}
\end{figure}

El grupo plantea un modelo general en donde los objetivos de producir datos, consumir información y compartir conocimiento en el área de interés, será alcanzado en varias fases\footnote{A través del presente documento se utilizan los términos de fase en el proceso investigativo y fases del Proceso de Desarrollo del Sistema Software. Las dos se suponen diferentes y su interpretación e importancia están asociados al contexto en el que se ubican.} de las cuales este documento describe aquellas que han culminado.

La figura \ref{modelo1} muestra el proceso de estructuración del sistema como una sucesión de fases que generan dimensiones de datos, información y conocimiento de un componente. El sistema visto como un \textit{holos} presenta al investigador una gran cantidad de datos que en la medida que se descubren, recolectan, observan y registran se vuelven susceptibles de ser descritos, analizados, integrados y comparados (información), acercándose a un conocimiento refinado. El carácter de discernible - el momento en que las dimensiones se solapan en grado sumo, se evidencia con el aumento colectivo de especialización en la materia y en nuestro caso, con el grado de inmersión que los usuarios del SITEM tengan a partir de la liberación de la versión 1.0 estable (diciembre de 2017).

En el transcurso de las fases \textit{solo se manejan ciertos aspectos} que incrementalmente refinan el modelo del sistema, con base en las diferentes \textit{mórulas} de datos, información y conocimiento generadas. Aunque en el gráfico se muestra un tanto discretos y exactos, los límites existentes entre las tres mórulas principales - datos, información, conocimiento; son en la realidad difusos. Si se consideran los aspectos meramente técnicos del SITEM se corre el riesgo inminente de diseminación en regiones poco profundas del sistema - dispersión en la mórula de datos - razón por la cual la herramienta software se ha convertido en un artefacto intermedio y no en el fin último de la investigación.

\subsection{SITEM – Fase I}
Con la primera fase  se definió un conjunto base de componentes de las redes de eSalud y sus interrelaciones, basado en una investigación exploratoria y descriptiva realizada por los integrantes del grupo. También se determinaron las características esenciales del portal esperado para el SITEM vislumbrando la necesidad de integrar un producto software adaptado a las necesidades del entorno distrital. Esto teniendo en cuenta que no existe una plataforma de acceso público que permita la gestión y análisis de información pertinente, confiable, actualizada y estructurada en torno a las redes de eSalud del Distrito Capital.

Esta fase se definió un modelo de negocios que lograba mostrar las interrelaciones que tendría el sistema con los proyectos del grupo y un conjunto representativo de portales relacionados temáticamente, encontrando \textit{un costo de oportunidad adecuado ya que en la actualidad ningún portal se especializa en el proceso de diagnóstico de capacidad para la implementación de eSalud.} No obstante, al momento de emprender el desarrollo, el grupo de investigación no contaba con los recursos mínimos requeridos por lo cual la propuesta y su implementación no pasó de ser un prototipo de baja funcionalidad. 

Los alcances y logros efectivos de esta fase fueron:

\begin{itemize}
\item Descripción del Modelo de Negocio.
\item Propuesta de Desarrollo.
\item Bosquejo de la Arquitectura general del Sistema.
\item Integración conceptual del SITEM dentro de los proyectos del grupo.
\item Estudios sobre filosofía de Software Libre y el movimiento del código abierto.
\item Construcción de un Prototipo de baja funcionalidad conocido como SITEM versión 0.1, bautizada internamente como \textbf{Kauil}.
\end{itemize}


\begin{figure}
 \centering
 \includegraphics[width=142mm, height=190mm]{fase_sitem.png}
 \caption{Fases transcurridas en el desarrollo del SITEM}
 \label{fase_sitem}
\end{figure}

\subsection{SITEM – Fase II}

El modelo de negocio y el modelos de requisitos, análisis y diseño sirvieron de base para la segunda fase cuyo objetivo era \textit{..describir una arquitectura general para el Sistema de Información para la Caracterización de Proyectos de eSalud}; y crear un portal prototipo que integrara componentes de software para concretar parte de dicha arquitectura. Para poder minimizar los riesgos asociados al proyecto se adaptó el \textit{OpenUP} a las especificidades de desarrollo del Sistema, lo que favoreció efectivamente su elaboración, implementación, mantenimiento y crecimiento.

En la segunda fase, debido a la naturaleza del SITEM, el grupo de investigación decide separar los hilos de desarrollo del Sistema y del proceso de estructuración del sitio web del grupo. El SITEM por primera vez se puede acceder desde \textit{Internet} gracias al despliegue que se realiza sobre la plataforma de hardware y software brindada por la \textit{Universidad Distrital}. 

En esta fase se realiza el modelado de datos y se esbozan las rutinas de manejo de seguridad. Para la integración de componentes se utilizan herramientas de software libre y el grupo de desarrollo aumenta a cinco integrantes. El trabajo se encuentra, con excepción del director del proyecto, soportado y ejecutado por estudiantes de pregrado del proyecto curricular de Ingeniería Electrónica convirtiéndose en el \textbf{primer proyecto de desarrollo de software libre realizado por el GITEM}.

Los alcances de esta fase fueron:
\begin{itemize}
\item Arquitectura Mejorada del Sistema
\item Modelo depurado de Requisitos
\item Modelo de Análisis - Segunda Versión
\item Modelo de Diseño
\item Construcción de Componentes software soportado en su totalidad por herramientas de software libre.
\item Versión 0.5 beta bautizada internamente como \textbf{Gucumatz}.
\end{itemize}


\subsection{SITEM – Fase III}

En esta fase el grupo toma una decisión arquitectónica importante y concentra su esfuerzo en el motor de integración de aplicaciones, cuyo objetivo es la federación de sistemas existentes de software libre o código abierto (FLOSS) con el objetivo de lograr un mayor nivel funcional del que se había alcanzado en las fases anteriores. Se trata de buscar el aseguramiento de la calidad en el desarrollo, la interoperabilidad, escalabilidad y el uso extensivo de FLOSS. Se documenta todas las etapas involucradas en la creación de la aplicación para que sirva de plantilla a sistemas relacionados y se formaliza las áreas de capacitación a partir de ciclos genéricos de transferencia de conocimiento apoyados en tecnologías de la información.

Es precisamente esta fase la que da vida a este documento, a una arquitectura emergente y una versión 1.0 del sistema denominada \textit{OpenSITEM}, con la que el grupo entrega un sistema complejo constituido por un motor de federación - basado en SARA un framework de desarrollo para aplicaciones web desarrollado por el grupo - que articula las herramientas externas Knowage, ERPNext, Alfresco Community, CAMUNDA y OpenProject, así como varias herramientas propias tales como:

\begin{itemize}
\item Sistema Integrado de Evaluación
\item Sistema de Gestión de Redes de Práctica y Colaboración
\item Sistema de agentes notificadores y de recomendación
\item Sistema de Información Geográfica 3D basado en Cesium.
\end{itemize}

De esta forma SITEM se convierte en una compleja solución que abarca diferentes dominios, con capacidad de adaptación para un propósito específico. En particular, la fase III se concentra en la configuración de OpenSITEM como \textit{Sistema de Información para la Caracterización de Proyectos de eSalud}.


\begin{figure}
 \centering
 \includegraphics[width=156mm, height=118mm]{pagina_principal.png}
 \caption{Sistema de Información para la Caracterización de Proyectos de eSalud. Interfaz de Usuario en la Fase III}
 \label{sitem_faseIII}
\end{figure}
\chapter{Contexto Teórico}
No son los aspectos técnicos para la creación de sistemas de información los que distinguen al SITEM sino las nuevas estructuras de información -y su correlación, las que brindan una novedad en el contexto de la proyección de soluciones en Telemedicina. El presente capítulo provee la base teórica que sustenta dichas estructuras y relaciones, abordando el marco legislativo y demás temáticas que son necesarias para caracterizar los componentes primordiales de los Sistemas de Telemedicina. 

Teniendo en cuenta el caracter evolutivo del proyecto software y como referencia para posteriores desarrollos soportados en el SITEM, también se tratan cuestiones relevantes al proceso de ingeniería que guió el análisis, diseño y elaboración del sistema.

\section{Telemedicina}

La Telemedicina \cite{aim}, \cite{bashshur77}, \cite{itu} se ha convertido rápidamente en un concepto que extiende sus raices etimológicas. El Ministerio de Protección Social colombiano con la \textit{Resolución 1448 del 8 de mayo de 2006} define a la Telemedicina como:
\begin{quote}
“la provisión de servicios de salud a distancia, en los componentes de promoción, prevención, diagnóstico, tratamiento o rehabilitación, por profesionales de la salud que utilizan tecnologías de la información y la comunicación, que les permiten intercambiar datos con el propósito de facilitar el acceso de la población a servicios que presentan limitaciones de oferta, de acceso a los servicios o de ambos en su área geográfica.”
\end{quote} 

Es por tanto un campo multidisciplinar que integra componentes de diferentes áreas del saber que incluye entre otros a la medicina, la ingeniería electrónica, la telemática, la informática, la ingeniería de sistemas, la inteligencia artificial, la biónica, la sicología, la sociología y la antropología. Las redes actuales de Telemedicina consideran elementos que van mucho más alla del simple despliegue de redes tecnológicas de intercomunicación y evidentemente se plantean como redes de interacción social cuyo objetivo primario - más no el único, es la prestación de servicios médicos apoyadas en las TIC.

Dependiendo el grado en que se presente cada uno de los elementos mostrados en la figura ~\ref{elementosred} y de la mayor o menor correlación entre ellos, se pueden crear sistemas de Telemedicina que se acerquen al ideal de proveer servicios de salud de alta calidad. Dichos sistemas, aunque dinámicos y evolutivos, deberán mantener una estructura nuclear cuyas propiedades generales son según \cite{bashshur95}: 
\begin{enumerate}
\item Separación geográfica entre el proveedor y el cliente durante un encuentro clínico o entre dos o más proveedores.
\item Utilización de tecnologías informáticas y de comunicaciones para realizar la interacción. 
\item Equipo humano/técnico de gestión del sistema. 
\item Desarrollo de una infraestructura organizacional. 
\item Desarrollo de protocolos clínicos para orientar a los pacientes hacia diagnósticos y fuentes de tratamiento apropiados. 
\item Normas de comportamiento que reemplazan las normas del comportamiento presenciales.
\end{enumerate}

\begin{figure}
 \centering
 \includegraphics[width=156mm, height=156mm]{sistema_telemedicina.png}
 \caption{Elementos genéricos y áreas del saber de una red de Telemedicina}
 \label{elementosred}
\end{figure}

La Telemedicina es catalogada siguiendo básicamente criterios de \cite{oas2002}: sincronización temporal entre el proveedor y el cliente -diferida o en tiempo real; servicios de salud que se presten \cite{aparicio2000} - consulta, diagnóstico, atención, seguimiento, educación, administración, etc;  o especialidad médica que se trate - radiología, patología, cardiología, dermatología, etc. La catalogación siempre es independiente de la red tecnológica y de comunicaciones sobre la cual se despliegue.

\subsection{Componentes Tecnológicos de los Sistemas de Telemedicina}
La fase actual del SITEM se centra en dos conjuntos básicos de componentes de los sistemas de Telemedicina, los médicos y los tecnológicos. Debido al perfil de profesionales que han participado en el desarrollo, los elementos tecnológicos han sido mejor caracterizados hasta el momento.

\begin{figure}
 \centering
 \includegraphics[width=156mm, height=156mm]{red_1.png}
 \caption{Subsistemas Tecnológicos Básicos en un Sistema de Telemedicina}
 \label{subsistemas}
\end{figure}


Para efectos de facilitar su análisis y modelado, en la dimensión puramente tecnológica, un sistema de Telemedicina puede reducirse a cuatro subsistemas \cite{aparicio2003}:

\begin{itemize}
 \item \textbf{Captura de datos:} Conformado por los dispositivos de hardware, los protocolos y aplicaciones software que trabajan conjuntamente para transformar información médica en datos susceptibles de ser administrados usando técnicas digitales.
 \item \textbf{Transmisión de Datos:} Hacen parte de este subsistema los dispositivos de hardware, las tecnologías de interconexión, los protocolos y aplicaciones que permiten estructurar redes de transmisión de datos digitales de una manera fiable en tiempos aceptables para un servicio específico.
 \item \textbf{Gestión de Información:} Dispositivos de hardware - computadores, sistemas de almacenamiento masivo, etc; y  sistemas de información que almacenan, procesan, distribuyen y analizan la información proveniente de los subsistemas de captura de datos.
 \item \textbf{Despliegue de información:} Elementos de hardware (pantallas, transductores, sistemas de audio, etc), aplicaciones software y protocolos asociados que permiten recibir y reproducir la información médica. 
\end{itemize}

Todos ellos necesariamente interrelacionados por medio de interfaces y protocolos definidos; El uso de estándares abiertos es de vital importancia para permitir que los sistemas de Telemedicina puedan ser interoperables, esto se ha logrado en gran medida en el subsistema de transmisión de datos pero aún se encuentran serios problemas en los demás subsistemas debido al sinnúmero de patentes y protocolos propietarios que las empresas fabricantes de dispositivos médicos aún ostentan.
\section{Marco Legislativo y Normativo}

El SITEM es un sistema que gestiona información sobre diferentes áreas del saber. Cada aspecto está definido conforme a un marco legislativo y normativo concreto, o a estándares y normas de  uso extendido y de facto aceptado en el mundo. La información del SITEM es de acceso público por lo que no podrá incluirse ningún tipo de información que esté protegida por derechos de autor\cite{congreso565},\cite{congreso23} que restrinjan su difusión. 

De forma análoga debido a que el SITEM es un proyecto de software libre ha de ceñirse en todo momento a la normatividad expresada en \cite{congreso565},\cite{congreso44},\cite{congreso1360} para garantizar que todos los aspectos tanto técnicos como conceptuales son de dominio público o, en el caso de aquellos inéditos, están debidamente registrados. En la actualidad el grupo GITEM cursa el requerimiento para obtener el registro correspondiente al núcleo de federación de aplicaciones, el cual es pieza fundamental para la articulación del producto software. 

La relación puntual de la normatividad trasciende el objetivo del presente documento, sin embargo, se describen a continuación aquellas que más incidencia han tenido en el modelado del sistema.

\subsection{En el Ámbito de los Servicios Médicos}

El Derecho a la Salud ha sido reconocido por normas y pactos internacionales contenidos en tratados sobre Derechos Humanos, Económicos, Sociales, y Culturales  (DHESC) . Esos acuerdos han sido ratificados por Colombia para su cumplimiento como un derecho de los ciudadanos. “La Corte Constitucional; ha señalado que el inciso segundo del artículo 93 de la Carta Política confiere rango constitucional a todos los tratados de derechos humanos, económicos,  sociales y culturales, ratificados por Colombia y referidos a derechos que ya aparecen en la Carta” \cite{sentencia1319} como ocurre con el Derecho a la Salud. 

Al Ministerio de Salud y Protección Social, le corresponde expedir las normas técnicas y administrativas de obligatorio cumplimiento para las Entidades Promotoras de Salud del régimen contributivo, las Instituciones Prestadoras de Salud del Sistema General de Seguridad Social en Salud, las Administradoras del Régimen Subsidiado y para las Direcciones Seccionales, Distritales y Locales de Salud en cuanto al objetivo de cumplimiento en el desarrollo de actividades de protección específica, detección temprana y atención de enfermedades de interés en Salud Pública. 

A continuación se registra la normatividad que se tuvo en cuenta al momento de definir los componentes actuales del subsistema de servicios médicos en cuanto a la relevancia que se tiene tanto para la proyección de nuevas redes de eSalud, como para apoyar los sistemas básicos ya existentes. Es de anotar que lo contemplado en las leyes nacionales es, en su mayoría, derivado de normas internacionales que han sido objeto de detallados estudios y reconocidas técnicamente con base en las experiencias vividas por los profesionales de esta área. 

\begin{description}

\item[Ley 1751 de 2015]. Por medio de la cual se regula el derecho fundamental a la salud. Es una ley estatutaria que surge a partir de la debacle del proceso de reforma y tiene como efecto positivo el elevar a la salud como un derecho fundamental. Entró en rigor a partir del año 2017 y da lineamientos para reestructurar el sistema de salud a partir del desarrollo de \textit{redes de servicios} públicos, privados o mixtos. También declara la necesidad de establecer políticas relacionadas con la salud tales como la política para la información, la política de innovación, ciencia y tecnología; y la política farmacéutica nacional.

\item[Ley 1419 de 2010].  Por la cual se establecen los lineamientos para el desarrollo de la telesalud en Colombia. Define las redes de telesalud y el aprendizaje en telesalud como ejes principales de la gestión del conocimiento en salud. Si bien esta ley obliga a desarrollar el mapa de conectividad, aún en el 2017 no se encuentra uno que esté disponible para los ciudadanos.

\item[Resolución 2182 de julio 9 de 2004] Con esta resolución se definían las Condiciones de Habilitación para las instituciones que prestan servicios de salud bajo la modalidad de Telemedicina. Fue derogada por el artículo 11 de la \textbf{resolución 1043 de 2006}, con la cual se establecen las condiciones que deben cumplir los Prestadores de Servicios de Salud para habilitar sus servicios e implementar el componente de auditoría para el mejoramiento de la calidad de la atención y se dictan otras disposiciones. 

Posteriormente, con la \textbf{Resolución 1448 de 8 de Mayo de 2006} se regula la prestación de servicios de salud bajo la modalidad de telemedicina y establece las condiciones de habilitación de obligatorio cumplimiento para las instituciones que prestan servicios de salud. Esta resolución aclara que las actuaciones de los médicos en el ejercicio de la prestación de servicios bajo la modalidad de telemedicina se sujetarán a las disposiciones establecidas en la \textbf{Ley 23 de 1981} y demás normas que la reglamenten, modifiquen, adicionen o sustituyan.

\item[Resolución 4678 de noviembre 11 de 2015] Con esta resolución, modificada por la Resolución 1113 de 2017, el Ministerio de Salud y Protección Social adopta una Clasificación Única de Procedimientos en Salud (CUPS) la cual "...corresponde a un ordenamiento lógico y detallado de los procedimientos y servicios en salud que se realizan en al país, en cumplimiento de los principios de interoperabilidad y estandarización de datos utilizando, para tal efecto, la identificación por un código y una descripción validada por los expertos del país."\cite{minsalud4678}. La Clasificación Única de Procedimientos en Salud adaptación para Colombia, se implementa por \textbf{Resolución 365 de 1999}. Su primera publicación se presenta en un solo volumen que contiene la Lista Tabular y el Índice Alfabético. A partir de dicha resolución se realizó la primera actualización de la CUPS (1°A-CUPS) mediante la \textbf{Resolución 2333 de 2000}. En el año 2016, mediante la Resolución 3804, se establece el procedimiento para la actualización de la CUPS, con lo que el Ministerio espera darle una mayor agilidad al proceso.

\item[Resolución 1830 de junio 23 de 1999] adopta para Colombia, “Las codificaciones únicas de especialidades en salud, ocupaciones, actividades económicas y medicamentos esenciales" para el Sistema Integral de Información del SGSSS - SIIS 

\item[Resolución 1895 de noviembre 19 de 2001] por la cual se adopta para la codificación de morbilidad en Colombia, La Clasificación Estadística Internacional de Enfermedades y Problemas Relacionados con la Salud - Décima revisión. 

\begin{quote}
Considerando que en la 43a. Asamblea Mundial de la Salud llevada a cabo en 1990, fue aprobada por la Conferencia Internacional la Clasificación Estadística Internacional de Enfermedades y Problemas Relacionados con la Salud - Décima revisión -, (CIE-10) en la cual Colombia no expresó objeciones y adquirió el compromiso de implementarla. Resuelve Adoptar para la codificación de morbilidad en Colombia, la Clasificación Estadística Internacional de Enfermedades y Problemas Relacionados con la Salud -Décima revisión-, contenida en la publicación científica No.554 de la Organización Panamericana de la Salud, presentada en tres volúmenes: V1. Lista de Categorías; V2 Manual de Instrucciones; V3 Índice Alfabético.\end{quote} 

\item[Resolución 1995 de julio 8 de 1999] por la cual se establecen normas para el manejo de la Historia Clínica.

La Historia Clínica es un documento de vital importancia para la prestación de los servicios de atención en salud y para el desarrollo científico y cultural del sector, \textit{es un documento privado, obligatorio y sometido a reserva}, en el cual se registran cronológicamente las condiciones de salud del paciente, los actos médicos y los demás procedimientos ejecutados por el equipo de salud que interviene en su atención. Dicho documento únicamente puede ser conocido por terceros previa autorización del paciente o en los casos previstos por la ley. 


\item[Circular 015 de abril 4 de 2002] estándar de historias clínicas y registros, establece la obligatoriedad de definir procedimientos para utilizar una historia única institucional. Ello implica que la institución cuente con un mecanismo para unificar la información de cada paciente y su disponibilidad para el equipo de salud. No es restrictivo en cuanto al uso de medio magnético para su archivo, y sí es expreso en que debe garantizarse la confidencialidad y el carácter permanente de registrar en ella y en otros registros asistenciales.


\item [Decreto 2092 de 2 de Julio de 1986], Por el cual se reglamenta parcialmente los Títulos VI y XI de la \textbf{Ley 09 de 1979}, en cuanto a la elaboración, envase o empaque, almacenamiento, transporte y expendio de Medicamentos, Cosméticos y Similares. Se dan las Disposiciones Generales y Definiciones, el Registro Sanitario de Medicamentos, Cosméticos y Similares.
\end{description}

\subsection{En el Ámbito del Desarrollo de Redes Teleinformáticas}

\begin{description}
\item[Documento CONPES 3072] Aunque no es una norma regulatoria, es una declaración oficial del gobierno colombiano acerca de la necesidad de fomentar las Tecnologías de la Información para potenciar la absorción, creación y divulgación del conocimiento por medio del desarrollo sostenible en las infraestructuras física, de información y social. Según \cite{mincomunicaciones3072}:  “... para que el país pueda ofrecer un entorno económico atractivo y participar en la economía del Conocimiento, resulta indispensable desarrollar una sociedad en la que se fomente el uso y aplicación de las Tecnologías de la Información. A través de estas Tecnologías, se puede efectuar un salto en el desarrollo en un tiempo relativamente breve, mucho menor del que se necesita para superar el déficit de infraestructura física.".

El documento CONPES brinda un referente válido pues la mayoría de los objetivos estratégicos del SITEM contienen el espíritu expresado en diferentes partes del mismo.

\item[Documento CONPES 3582] Política Nacional de Ciencia, Tecnología e Innovación. En el cual se enfatiza el desarrollo de la salud y la tecnología como mecanismos de generación de valor social.

\item[Resolución 087 de 1997] “Por medio de la cual se regula en forma integral los servicios de Telefonía Pública Básica Conmutada (TPBC) en Colombia.” En donde claramente se expresa que: \begin{quote}
Los servicios de TPBC deberán ser utilizados como instrumento para impulsar el desarrollo político, económico y social del país con el objeto de elevar el nivel y la calidad de vida de los habitantes en Colombia. Los servicios de TPBC serán utilizados responsablemente para contribuir a la defensa de la democracia, a la promoción de la participación de los colombianos en la vida de la Nación y la garantía de la dignidad humana y de otros derechos fundamentales consagrados en la Constitución Política, y para asegurar la convivencia pacífica.\end{quote} 

Esta resolución presenta particular importancia para la extracción de elementos semánticos y algunos componentes necesarios en las redes de telecomunicaciones basadas en telefonía conmutada. Algunas heurísticas usadas en el subsistema de consultoría también se basan en apartes de esta resolución.

\item [Manual de Calidad de Servicio] “Con este Manual de calidad de servicio (QoS) se especifican los parámetros de calidad de servicio de red que permiten el suministro de servicios a los clientes y los usuarios, satisfaciendo sus expectativas de calidad de servicio. Estos parámetros tienen que ver tanto con la implementación del servicio como con su utilización continua. Asimismo, la calidad de servicio se relaciona con todos los aspectos relativos a la evaluación y gestión de las redes.”\cite{ITU2004}

Sus principales aspectos son recogidos en \cite{crtcondiciones} y \cite{crtindicadores}  las cuales sirven de base para el proyecto de resolución\cite{crtqos} de la Comisión de Regulación de Comunicaciones y que especifica entre otros: las definiciones relativas a la QoS, parámetros de medición, variables y propiedades técnicas de diferentes enlaces. Aunque el objetivo principal de este manual es el de garantizar la QoS en un sistema de telecomunicaciones dado es importante notar que sus indicaciones deben ser tenidas en cuenta al momento de proyectar los servicios de telecomunicaciones en un sistema de telemedicina dado. En estos aspectos también se considera dentro del modelado de ciertos componentes del SITEM las recomendaciones de la UIT en \cite{ITUG1000} y \cite{ITUG1010}, las cuales aún no tienen un equivalente en la normatividad colombiana pero son de uso extendido alrededor del orbe.


\end{description}

\subsection{En el ámbito del Tratamiento de Datos e Información}  

\begin{description}
 \item[Ley 1581 de 2014 y Ley 1266 de 2008]. Estas leyes están relacionadas con la protección de los datos personales y el aseguramiento de la privacidad de la información personal.
 
\end{description}
\section{Ingeniería de Software}
A medida que el marco conceptual del OpenSITEM crece, el cúmulo de nuevos requerimientos - no vislumbrados en su planteamiento inicial, fomenta que los riesgos asociados al desarrollo de los componentes también crezcan. Lo que en un principio no era más que una "herramienta" para la administración del acervo documental fruto de una investigación, se convirtió en una propuesta de sistema de información y conocimiento que suponía un reto novedoso al interior del grupo de investigación. 

Es evidente que se deben sumar nuevos saberes para procurar manejar formalmente el proceso de elaboración del sistema: un punto inicial y obligado de estudio se centró en la ingeniería de software. 

Como rama de la ingeniería comparte la definición fundamental que de la misma brindó a mediados del siglo pasado el \textbf{Consejo de Ingenieros para el Desarrollo Profesional} - ECPD, por sus siglas en inglés; y que en general propone que: \begin{description}
\item[Ingeniería]
Es la aplicación creativa de principios científicos para el diseño o desarrollo de estructuras, máquinas, aparatos, procesos de manufactura o sistemas genéricos, para ser usados de forma independientemente o combinados; o la construcción y operación de los mismos con total conocimiento de su diseño; o el pronóstico de su comportamiento bajo ciertas condiciones de operación; todo aquello respecto a una funcionalidad esperada asegurando economía en el manejo de los recursos y con seguridad para la vida y la propiedad. \end{description}\footnote{Adaptación de la definición hecha por los autores.}

Se concibe entonces a la ingeniería de software como la aplicación de los principios de ingeniería a los sistemas de software con base a “un acercamiento sistemático, cuantificable y disciplinado del desarrollo. operación y mantenimiento de software”\cite{softwareengineering}; y ciertamente se fundamenta en actividades interrelacionadas, propias del ser humano cognosciente y creativo que en \cite{objectoriented} se identifican como: 

\begin{itemize}
\item \textbf{Actividades de Modelado:} Para abstraer la complejidad del dominio del problema en unidades factibles de ser objeto de estudio y análisis. En este contexto las nociones de contratos funcionales e independencia conceptual juegan un rol importante. Se pretende con estas actividades obtener modelos de análisis - como representaciones relativamente simples de la realidad y modelos de diseño - como representaciones del dominio de la solución de un problema dado, representado con un modelo de análisis. 
\item \textbf{Actividades de Resolución de problemas:} Siendo la ingeniería de software un proceso guiado de búsqueda de solución a un problema específico del ser humano que es viable de ser apoyado por sistemas software. Se concibe actualmente como un proceso investigativo que, de acuerdo a un acercamiento holístico, contempla flujos de trabajo continuos y evolutivos de exploración, descripción, análisis, comparación, explicación, predicción, proposición, modificación, confirmación y evaluación 
\item \textbf{Actividades de Adquisición de conocimiento:} Durante el desarrollo del sistema software el ingeniero, a partir de un modelo constructivista, recrea constantemente su conocimiento tácito a partir de las nuevas experiencias y el mayor conocimiento del dominio del problema así como de los diferentes paradigmas usados en la consecución de soluciones óptimas. En realidad el ingeniero, así como los demás actores que intervienen con el sistema, ven revalidados o reformados sus conocimientos a medida que los requerimientos son cumplidos y los riesgos minimizados.
\end{itemize}

También contempla actividades propias de trabajo colaborativo que producen integración de saberes en ambientes inter, trans y multidisciplinarios, lo que potencia efectivamente la creación de ciclos de conocimiento que contribuyen al refinamiento continuo del sistema - como objeto perfectible, y del conocimiento directo, que tanto del sistema como del proceso, tienen los actores vistos como sujetos perfectibles, racionales y cognoscientes; dentro de una dinámica de retroalimentación entre el sujeto que crea el sistema y el sistema mismo.

\subsection{Proceso de Desarrollo del Software}
Un proceso de desarrollo de software puede ser visto como el conjunto de actividades que deben realizar un grupo de personas para dar solución - mediante un sistema software- a un problema cuyas características y condiciones de resolución han sido especificadas. El producto final, el software, es un \textbf{sistema} o sea un conjunto de componentes funcionales que se relacionan por medio de interfaces definidas logran el objetivo común de solucionar los problemas determinados. 

Para determinar el proceso más apropiado según las necesidades y especificidades del proyecto se condujo una metodología \footnote{La metodología no fue exhaustiva y se limitó a un grupo muy reducido de elementos cuya caracterización se basó exclusivamente en indicadores de tipo cualitativo. Se recomienda remitirse a \cite{carty}, \cite{pressman}, \cite{jacobson2000}, \cite{koch} - entre otros, para detalles de los diferentes procesos.} centrada en diferentes modelos ampliamente aceptados en el campo de la ingeniería de software que al final dio lugar a un proceso consolidado de guía para el desarrollo del Sistema de Información para Proyectos de Telemedicina. Se aclara que este modelo, como el sistema y los actores, no es indiferente del proceso evolutivo de adaptación de conocimiento por lo que en realidad no se considera como una fórmula mágica sino simplemente como un caso específico que aporta unos lineamientos interesantes para otros proyectos de software similares y sirve de base para los ingenieros de proceso de fases posteriores en el ciclo de vida del macroproyecto SITEM. En últimas, un sistema software exitoso es aquél en el cual todos sus componentes se refinan constantemente y el proceso de desarrollo es un componentes nuclear que tiene mayor incidencia.

Existen tantos procesos de desarrollo de sistemas en el mundo que la mera enumeración taxativa podría cubrir cientos de páginas. El dilema de cual es el mejor de ellos es irresoluble, sin embargo se puede definir las características óptimas para un contexto en particular teniendo en cuenta múltiples indicadores a partir de aspectos tales como:

\begin{itemize}
 \item Tamaño del Grupo de Desarrollo
 \item Presupuesto
 \item Límites de tiempo.
 \ 
\end{itemize}

\subsection{Principios de Diseño y Desarrollo}

A través del tiempo se han decantado ciertas prácticas que son reconocidas como las más óptimas cuando se trata de construir un sistema software de gran magnitud - tanto en líneas de código, como en funcionalidad y recursos involucrados. Estos principios son tenidos en cuenta independientemente del proceso de desarrollo que se siga. Quizás los de mayor difusión son los \textit{patrones GRASP} - acrónimo de General Responsibility Assignment Software Patterns, que se basan en la asignación precisa de responsabilidades a cada uno de los componentes del sistema software.\footnote{Aún el Object Management Group declara el uso de ciertos principios de diseño en el desarrollo del metamodelo que especifica a UML}.

En el desarrollo del SITEM se recomienda, como estrategia para mantener la calidad del software, que los integrantes del grupo tengan en cuenta y adquieran competencias en el manejo de los siguientes patrones y principios: \footnote{Debido a que el proceso general adoptado contiene elementos del Desarrollo de Software de Código Abierto \cite{koch}, no siempre se obtiene un seguimiento preciso de los patrones por parte de todos los participantes. Refinamientos sucesivos y estrategias de capacitación se despliegan al interior del grupo para incrementalmente llegar a este objetivo.}

\begin{itemize}
\item \textbf{Modularidad}. Para facilitar las tareas de mantenimiento, depuración e incremento en la funcionalidad, se requiere que el sistema se implemente con base en componentes que presente propiedades de \textit{alta cohesión} funcional entre sus elementos y tengan \textit{bajo acoplamiento} entre sí. La alta cohesión funcional tiene en cuenta que el componente realiza solo tareas relacionadas y utiliza un conjunto de datos homogéneo. El bajo acoplamiento se refiere al hecho de que un componente se relaciona con otro a través de una interfaz estable y definida; dicha relación no está supeditada a la implementación interna de ninguno de los componentes, con bajo acoplamiento un cambio en un componente no requeriría ningún cambio en la implementación del componente asociado.

\item \textbf{Prueba continua.} Todos los módulos y sus componentes deben ser probados en cuanto su funcionalidad y el cumplimiento de los demás principios y patrones. Las actividades de prueba podrán ser automatizadas o realizadas manualmente pero en cualquier caso deben ser formalmente documentadas. Es una recomendación que en lo posible el personal de prueba sea diferente a aquel que ha diseñado, o construido, el componente.

\item \textbf{Codificar claramente.} La forma en que se ingresa el código o se agrupa un conjunto de elementos gráficos en un diagrama deben ser hecha de tal forma que se facilite su comprensión. Se recomienda el uso de comentarios para aquellas partes del código cuya funcionalidad no sea evidente o cuando se evite el tener que analizar piezas de código extensas.
 
\item \textbf{Abstracción Funcional por capas}. Los diferentes componentes del SITEM deberán centrar su funcionalidad en tres capas principales: datos, aplicación e interfaz. Los unidades que manejen cada una de las capas deben propender por conservar la modularidad.

\item \textbf{Reutilización}. Los diferentes componentes del SITEM - denominados bloques dentro del modelo de desarrollo, deben estar codificados de tal forma que puedan ser fácilmente adaptados en los diferentes módulos sistema.

\item \textbf{Re-creación de componentes}. Se debe conocer la estructura interna de un determinado componente para poder sugerir mejoras. Este principio no pretende desplazar a la reutilización sino que debe complementarlo. El contexto definirá cual de los dos deberá ser usado. El tiempo transcurrido desde la creación y la cantidad de uso del componente son indicadores a tener en cuenta.

\item \textbf{Controlar las versiones}. Debe mantenerse un repositorio que permita recuperar los estados anteriores de cualquier componente dentro del sistema. El incremento general en la funcionalidad, el refinamiento en el desempeño y la experiencia adquirida al desarrollar el software es información que permanece latente en los repositorios. Los repositorios integrados permiten mantener la sincronización de los grupos de trabajo y blinda el hilo estable - “oficial”- de los hilos secundarios en desarrollo o depuración. 

\item \textbf{Documentar}. Ya sea empotrado dentro del código, usando lenguajes de modelado o en artefactos independientes se deben documentar las actividades interesantes que se realicen en el desarrollo del sistema. La documentación debe usar estándares multiplataforma para que pueda ser transparentemente visualizados,editados y compartidos entre los integrantes del equipo de desarrollo.
\end{itemize}

\subsection{Proceso de Desarrollo de Software de Código Abierto}
Es indiscutible el papel preponderante que tiene la planificación en el desarrollo de cualquier tipo de sistema, sin embargo, no debe olvidarse que cuando se requiere solucionar un problema no basta con el mero seguimiento de una receta y es aquel “toque” único que brinda el ser humano el que hace que los sistemas de software se diferencien unos de otros. No es por casualidad que en nuestro medio el software se considera un producto que esta cubierto por la misma legislación que las obras literarias o musicales.

Apelando a ese recurso intangible llamado pasión, que aún hoy solo es característico de los seres humanos, se ha extendido en los últimos años el proceso de desarrollo de software de código abierto. Todos los paradigmas que las grandes empresas de desarrollo de software se han encargado de poner como las \textit{mejores prácticas}, se han reevaluado, trastocado, pisoteado y sin más ni más un gran cisma apareció en el horizonte. Este renacimiento moderno surge como respuesta humana al gran vacío de satisfacción de necesidades que brindaba el software a principios de los año noventa.

\begin{figure}
 \centering
 \includegraphics[width=156mm, height=137mm]{proceso_fs.png}
 \caption{Conceptos Básicos en el Desarrollo de Software de Código Abierto}
\label{proceso_fs} 
\end{figure}

Más que una forma de realizar sistemas, se trata de una visión revolucionaria en torno al software como patrimonio de la humanidad y se une a la filosofía del software libre que se expresa en \cite{stallman2002}:

\begin{quote}
“Software Libre” se refiere a la libertad de los usuarios para ejecutar, copiar, distribuir, estudiar, cambiar y mejorar el software. De modo más preciso, se refiere a cuatro libertades de los usuarios del software:

\begin{itemize}
\item La libertad de usar el programa, con cualquier propósito (libertad 0).
\item La libertad de estudiar cómo funciona el programa, y adaptarlo a tus necesidades (libertad 1). El acceso al código fuente es una condición previa para esto.
\item La libertad de distribuir copias, con lo que puedes ayudar a tu vecino (libertad 2).
\item La libertad de mejorar el programa y hacer públicas las mejoras a los demás, de modo que toda la comunidad se beneficie. (libertad 3). El acceso al código fuente es un requisito previo para esto.
\end{itemize}
\end{quote} 

Las aplicaciones más representativas del mundo del software libre como Apache, Mozilla, MySQL, PostgreSQL y el mismo sistema operativo Linux, luego de sus etapas primarias, adoptaron como proceso de desarrollo uno que contravenía en gran manera los fundamentos del control riguroso y ponía el futuro del sistema en manos de la anarquía \footnote{Definida en su sentido positivo como la situación humana en donde es innecesaria e indeseable la autoridad, lo que conlleva a una sociedad libre basada en el respeto mutuo de sus miembros y la cooperación voluntaria entre individuos.\cite{bce}}. Tal como lo propone Linus Torval “la idea es liberar versiones de prueba rápido a menudo, delegar cuanto sea posible, estar abierto hasta el punto de resultar promiscuo”.

Algunos principios fundamentales en este tipo de desarrollo los expone \cite{raymond}:

\begin{quote}
\begin{enumerate}
\item Todo buen trabajo de software comienza a partir de las necesidades personales del programador. (Todo buen trabajo empieza cuando uno tiene que rascarse su propia comezón). 

Esto podría sonar muy obvio: el viejo proverbio dice que "la necesidad es la madre de todos los inventos". Empero, hay muchos programadores de software que gastan sus días, a cambio de un salario, en programas que ni necesitan ni quieren. No ocurre lo mismo en el mundo Linux; lo que sirve para explicar por qué se da una calidad promedio de software tan alta en esa comunidad.
\item Los buenos programadores saben qué escribir. Los mejores, qué reescribir (y reutilizar).

... una importante característica de los grandes programadores es la meticulosidad con la que construyen. Saben que les pondrán diez no por el esfuerzo, sino por los resultados; y que casi siempre será más fácil partir de una buena solución parcial que de cero.
\item "Contemple desecharlo; de todos modos tendrá que hacerlo." cita encontrada en el capítulo 11 de libro The Mythical Man-Month escrito por el célebre Fred Brooks.

Diciéndolo de otro modo: no se entiende cabalmente un problema hasta que se implementa la primera solución. La siguiente vez quizás uno ya sepa lo suficiente para solucionarlo. Así que si quieres resolverlo, prepárate a empezar de nuevo al menos una vez.

\item Si tienes la actitud adecuada, encontrarás problemas interesantes.

\item Cuando se pierde el interés en un programa, el último deber es heredarlo a un sucesor competente.

\item. Tratar a los usuarios como colaboradores es la forma más apropiada de mejorar el código, y la más efectiva de depurarlo.

\item Libere rápido y a menudo, y escuche a sus clientes.

\item Dada una base suficiente de desarrolladores asistentes y beta-testers, casi cualquier problema puede ser caracterizado rápidamente, y su solución ser obvia al menos para alguien.

Dicho de manera menos formal, "con muchas miradas, todos los errores saltarán a la vista". A esto lo he bautizado como la Ley de Linus.

\item Las estructuras de datos inteligentes y el código burdo funcionan mucho mejor que en el caso inverso.

De nuevo Fred Brooks, Capítulo 11: “Muéstreme su código y esconda sus estructuras de datos, y continuaré intrigado. Muéstreme sus estructuras de datos y generalmente no necesitaré ver su código; resultará evidente.” 

\end{enumerate}
\end{quote} 
 
En general un proceso de desarrollo de software libre se basa en el hecho de que el programa puede ser instalado y el código fuente está disponible para cualquier persona. Es decir, la ausencia de barreras en cuanto a la limitación en el uso hace que muchas personas interesas en la funcionalidad que brinda el software lo descarguen y empiecen a utilizarlo. Dando inicio al siguiente ciclo:
\begin{enumerate}
\item El grupo inicial de programadores mantiene un sitio en la red para obtener retroalimentación de los usuarios los cuales reportan fallos, disafuncionalidades y solicitan nuevas características. 

\item Un desarrollador - que puede ser uno de los usuarios, revisa la lista de reportes y decide trabajar en uno específico; para tal efecto descarga la última versión del código fuente la modifica y la envía a un revisor para que este convalide la contribución.

\item La contribución se agrega al código fuente generando una nueva versión del sistema. Esto se realiza sincronizando el código fuente de desarrollo con aquél existente en la bodega de código fuente - repository, la cuál normalmente es un gestionada por un programa para el control de versiones. 

\item Si en algún momento dos programadores están realizando modificaciones a la misma porción de código y pretenden sincronizarlas ocurre un conflicto que deberá ser resuelto siguiendo reglas definidas que habitualmente contemplan el bloqueo de la versión más reciente, el aviso para resolución entre desarrolladores que causan el conflicto o el descarte de las contribuciones.
\end{enumerate}

De esta forma se va refinando el software siguiendo el ciclo mostrado en la figura \ref{proceso_fs}. El grupo de desarrollo se ve aumentado cuando usuarios expertos empiezan a proponer y realizar cambios directos en el código; cuando uno de ellos demuestra tener el suficiente interés y respeto hacia los intereses del software se le asigna el permiso para escribir directamente en la bodega de código.

La creación de la documentación así como de los modelos de requerimientos,análisis, diseño y despliegue siguen el mismo proceso.

\subsubsection{Métodos Ligeros}

Ha principios del milenio un grupo de experimentados desarrolladores, entre los que se encontraban Kent Beck, Alistair Cockburn, Martin Fowler y Dave Thomas, redactaron un manifiesto en el que consignaban los elementos de mayor importancia dentro del desarrollo de sistemas software \cite{beck1999}:
\begin{quote}
Nosotros estamos descubriendo mejores formas de desarrollar software dado que lo creamos y ayudamos a otros a realizar esta tarea. Por medio del trabajo de desarrollo hemos encontrado de gran valor elegir:

\begin{itemize}
\item \textbf{\textit{Individuos e interacciones}} sobre \textit{procesos y herramientas}
\item \textbf{\textit{Software ejecutable}} sobre \textit{documentación profusa}
\item \textbf{\textit{Colaboración del Cliente en el desarrollo}} sobre \textit{contrato de negocios}
\item \textbf{\textit{Respuesta al cambio}} sobre \textit{ceñirse a un plan.}
\end{itemize}

Mientras que existe valor en los elementos de la derecha nosotros valoramos más los elementos de la izquierda.\end{quote}

Con esto sentaban las bases para el despliegue de nuevos métodos de realizar software agrupados bajo el nombre genérico de “ágiles”\footnote{Siendo por definición un método caracterizado por ser liviano y ligero} que se contraponían a los métodos y procesos tradicionalmente rígidos y altamente planificados.

En \cite{koch} se expresan claramente las razones del porqué se desarrollan estos métodos y sus principales características. Los métodos ágiles nacen como respuesta del desarrollador puro al ambiente altamente industrializado y burocrático en el cual transcurren la mayoría de proyectos de desarrollo de software. En estos ambientes es típico el riguroso control que sobre el cumplimiento de cronogramas, planes de trabajo y presupuestos mantienen los denominados \textit{ingenieros de proceso}. El enfoque tradicional se basa en la planificación con la que se trata de \textit{predecir} desde las primeras etapas todos los pormenores del ciclo de desarrollo.

Debido a que los métodos tradicionales tienen fundamento en la ingeniería civil y mecánica, tratan de mitigar los riesgos poniendo un especial interés a las actividades de modelo en especial en las etapas de análisis y diseño; en general relegan a los desarrolladores a etapas de construcción erroneamente consideradas de \textit{cero esfuerzo} intelectual. Los requisitos del software se tratan de fijar desde los inicios del desarrollo, firmandose usualmente un contrato de aceptación de los mismos por parte del cliente. Estos métodos tradicionales siguen los lineamientos de aseguramiento de la calidad por la cual los procesos son eficientemente documentados, controlados, auditados, vigilados y mejorados. Todos esos aspectos hacen que el elemento clave sea el proceso y se relegue a segundo plano el crear productos que en realidad aporten un nuevo valor al cliente.

Para atacar la abrumadora complejidad que añade el proceso al sistema de software, los métodos ágiles proponen cambiar el paradigma \textit{predictivo} - rígido y resistente al cambio; por uno adaptativo que sea flexible y reaccione rápidamente antes cambios inesperados en los requisitos del software. Aquellos que han desarrollado un software de mediana o alta complejidad conocen de primera mano el hecho de que los requisitos no son estáticos, ellos cambian, evolucionan se transforman ya que en sí, no son sino abstracciones de necesidades del mundo real y este no es estático sino que se caracteriza por una fuerte dinámica.

El \textit{cliente también debe ser adaptable} en el sentido de que la mayoría de las veces los requerimientos del sistema los va descubriendo a medida que interactua con él. Una de las premisas de los procesos ágiles es el mantener un contacto permanente con el cliente e involucrarlo en todas las fases del desarrollo. Con esto se logra que el cliente obtenga un software que realmente cope sus intereses y (el cliente) sea consiente de los costos asociados al desarrollo del mismo.

Así como los requisitos cambian durante el desarrollo también lo hacen los recursos y el escenario en el cual se desenvuelve el equipo de trabajo. Para atacar esta característica de los sistemas software, se recomienda \textit{aferrarse a un presupuesto global} pero distribuyéndolo en pequeños presupuestos que solventen las tareas que ha corto plazo realizan los involucrados en el desarrollo.

Es claro con lo expuesto hasta ahora que el enfoque es considerar el desarrollo como una “\textit{carrera de 100 metros planos}” y no como una maratón. En tal sentido se deben gestar planes a corto plazo cuyo objetivo principal sea generar versiones del sistema que puedan ser probadas, corregidas e incrementalmente adicionadas en funcionalidad. Cada plan transcurre en lo que se denomina una iteración la cual usualmente no supera el mes de duración - algunos recomiendan una duración de dos semanas o ménos.\cite{beck1999}

Otro característica de los métodos clásicos, y que atacan los métodos ágiles, es aquella en la cual se considera a las personas como recursos intercambiables mediante la definición de \textit{roles} con funciones específicas y predictivas. Esto hace que las personas - cuyo comportamiento es poco predecible y no lineal \cite{cockburn1999}, tengan una moral baja y descienda su productividad; en el mejor de los casos trabajan con esfuerzo y, si sus condiciones son excelsas, rapidamente abandonan el grupo perdiéndose un activo intangible que repercute negativamente en la calidad global del sistema. Para los metodólogos ágiles el desarrollo se centra en las personas más que en los procesos, considera a cada miembro del grupo como un \textit{ser creativo e irreemplazable}, esto genera un gran cambio en cuanto al método: \textit{no es estático} ni recetario. Evoluciona, se recrea, se adapta y se concerta dentro el grupo de trabajo.

Evidentemente el proceso de desarrollo de software libre maneja los principios promulgados por los métodos ágiles los cuales encuentran quizás su máxima expresión en la Programación Extrema \cite{beck1999}.


\subsubsection{Proceso Unificado}
Según lo expresa \cite{alhir2003}:
\begin{quote}
El Proceso Unificado (UP) es un proceso de desarrollo de software basado en componentes dirigido por casos de uso, centrado en la arquitectura, iterativo e incremental...que utiliza la especificación UML dada por el Object Management Group (OMG) para preparar los esquemas del sistema. El Proceso Unificado es aplicable a diferentes tipos de sistemas de software, incluyendo proyectos de pequeña y larga escala; proyectos que tengan varios grados de complejidad técnica y administrativa, a través de diferentes dominios de aplicación y culturas organizacionales.

El PU nace de la unificación, en 1995, de la aproximación sugerida por Rational Software Corporation y el proceso orientado a objetos de la empresa Objetory AB. Se puede considerar al Proceso Unificado como un modelo de ciclo de vida del proyecto que incluye contexto, colaboraciones e interacciones. El UP es documentado totalmente en el libro “The Unified Software Development Process” escrito por Booch, Rumbaugh y Jacobson, y publicado por Addison- Wesley en 1999.\end{quote} 

Un sistema desde que nace hasta que muere repite el Proceso Unificado en ciclos de desarrollo constituidos por fases secuenciales cuyo objetivo es la producción incremental de liberaciones del sistema, llamadas comúnmente como generaciones del sistema. Cada una de las fases se convierte en un hito principal y esta constituido por pequeños microprocesos denominados \textit{iteraciones}. Habitualmente la numeración de las fases se hace de acuerdo a números enteros mientras que las iteraciones se hacen en números decimales. 

Las fases para el desarrollo de proyectos en el Proceso Unificado son cuatro \cite{jacobson2000}, a saber:

\begin{description}
\item [Fase de Concepción.] Tambien conocida como de inicio. Se centra en el establecimiento de las fronteras, ámbitos, riesgos asociados y visión del proyecto. Determina la viabilidad y los objetivos del proyecto. En esta fase podría tenerse una arquitectura general del sistema que esboze los subsistemas más importantes.

\item [Fase de Elaboración.] Se enfoca en la determinación de la arquitectura y requisitos del sistema; de esta forma se establece su viabilidad técnica. Durante esta fase se construyen los casos de uso críticos y se obtiene una arquitectura refinada del sistema. 

\item [Fase de Construcción] Es en la que se crea la mayor funcionalidad del producto, finaliza con cierta capacidad operativa. Se centra en la construcción del sistema y la arquitectura del sistema se considera estable.

\item [Fase de Transición]. Concluye con la liberación del producto, centrándose en la transición o distribución del sistema a la comunidad o usuario final. 

\end{description}

Dentro de las iteraciones el grupo de trabajo deberá distribuir sus esfuerzos en áreas estratégicas que conduzcan a la mitigación temprana de los riesgos, estás áreas son conocidas dentro del PU como disciplinas, figura \ref{proceso_unificado}:

\begin{itemize}
\item \textbf{Disciplina de administración de cambios en la configuración}, la cual se centra en la administración de la configuración del sistema y de las peticiones de cambios en la misma.

\item \textbf{Disciplina de administración de proyecto.}

\item \textbf{Disciplina de ambiente} que se centra en los ambientes de desarrollo del proyecto, incluyendo los procesos y las herramientas.

\item \textbf{Disciplina de modelado del negocio;} focalizado en la comprensión del negocio que esta siendo automatizado por el sistema capturando dicho conocimiento en un modelo del negocio.

\item \textbf{Disciplina de requerimientos,} necesaria para entender los requerimientos del sistema que automatiza el negocio y captura dichos requerimientos en un modelo de casos de uso.

\item \textbf{Disciplina de diseño y análisis,} centrada en analizar los requisitos y diseñar en sistema capturando tales  conocimientos en un modelo de análisis/diseño.

\item \textbf{Disciplina de implementación} para la implementación del sistema basado en el modelo de implementación.

\item \textbf{Disciplina de pruebas} que maneja las pruebas (evaluaciones) del sistema comparándolos con los requerimientos basándose primordialmente en el modelo de pruebas.

\item \textbf{Disciplina de distribución} encargada de la distribución del sistema basado en el modelo de distribución.
\end{itemize}

Durante la etapa de concepción la mayoría del esfuerzo está distribuido a través del modelo del negocio y la disciplina de requerimientos. 

Durante la fase de elaboración el esfuerzo se distribuye entre las disciplinas de implementación, diseño, análisis y requerimientos. Durante la etapa de construcción el esfuerzo se distribuye entre las disciplinas de análisis, diseño, implementación y pruebas. 

En la fase de transición el esfuerzo se distribuye a través de las disciplinas de prueba y distribución. Obviamente las disciplinas de soporte se distribuyen entre todas las cuatro fases. El objetivo general es producir el sistema, por lo tanto todas las disciplinas nucleares están comprometidas tanto como sea posible para no introducir riesgos en el proceso; esto es, los practicantes son los responsables de determinar cuales disciplinas comprometer y en que momento hacerlo.

En este punto es necesario definir varios conceptos: 
\begin{itemize}
\item \textbf{riesgo} en el Proceso Unificado se concibe como un obstáculo para alcanzar el éxito en la ejecución de una actividad, este riesgo puede estar determinado por características del negocio, humanas o técnicas.

\item \textbf{Iteración} es un paso o rama a través de un ruta hasta cierto destino. Dicho de otra forma es un movimiento planeado que puede ser evaluado para demostrar un progreso tangible dentro de una actividad o proceso, además, de acuerdo a lo citado por \cite{Zavala2000}: “Una iteración es iterativa en el aspecto de que es un acto repetitivo que propende la mejora continua del trabajo. Aditiva en el caso de que el resultado es siempre superior al alcanzado con un solo trabajo y paralela ya que el trabajo puede ser concurrente dentro de la iteración.”

\end{itemize}


\begin{figure}
 \centering
 \includegraphics[width=140mm, height=190mm]{pu.png}
 \caption{Ciclo de Desarrollo de un sistema según el Proceso Unificado}
 \label{proceso_unificado}
\end{figure}

Cuando se decantan los \textit{requerimientos}, aquello que el sistema debe cumplir, se está declarando explícitamente los casos de uso los cuales, dado que el Proceso Unificado está manejado por ellos, determinan las iteraciones. De la misma forma en que los casos evolucionan en el marco de las disciplinas regidas por un proceso iterativo, los sistemas evolucionan constantemente con base en iteraciones realizados en el marco de su arquitectura. Incluyendo en la arquitectura todos los elementos, sus colaboraciones e interacciones.  

Resulta pues obvio que la iteración, dado que es un avance demostrable, tiende en últimas a reducir los riesgos inherentes a cada una de las etapas del proceso de desarrollo. Esto ha sido definido por \cite{alhir2003} en su artículo:
\begin{quote}
.. de esta forma las iteraciones confrontan los riesgos derivados de los casos de uso y la arquitectura para alcanzar el éxito en el proyecto, buscando en todo momento reconciliar las fuerza técnicas y del negocio. Una iteración esta acotada en el tiempo con inicio y final fijos en donde una colección de colaboraciones son planeadas, ejecutadas y evaluadas del tal forma que en todo momento se pueda demostrar progreso en el proceso... un caso de uso evoluciona a través de un gran número de iteraciones y a través de cualquier número de disciplinas nucleares en una iteración. La experiencia y aprendizaje obtenido en una iteración evidentemente conduce la aplicación de las próximas iteraciones dentro del proceso..\end{quote} 

La iteración se convierte en el hito más importante para asegurar el crecimiento continuo y el aseguramiento de la calidad total dentro del sistema que se está desarrollando. Las iteraciones marcan totalmente el ciclo de desarrollo del SITEM que utiliza una aproximación iterativa propuesta por el Proceso Unificado.

\section{UML: Lenguaje de Modelado Unificado}

Independiente del modelo utilizado para la construcción y gestión del desarrollo del sistema se requiere que la comunicación entre los diferentes integrantes del grupo de desarrollo sea efectiva. Se hace indispensable que todo el equipo utilice y entienda un lenguaje consistente y unificado con el cual expresen claramente sus ideas y desde el cual puedan marcar claramente las directrices a seguir. El lenguaje de Modelado Unificado, UML por sus siglas en inglés, brinda las características tanto sintácticas como semánticas para lograr caracterizar lógicamente cualquier tipo de software permitiendo ser utilizado en cualquier etapa del diseño y es especialmente útil en aquellos desarrollos enfocados a objetos.

El Lenguaje de Modelado Unificado es definido por el \textit{Object Management Group}\footnote{La página del OMG (www.omg.org) describe la organización como un consorcio de la industria de la informática, sin ánimo de lucro, con caracter internacional y de membresía abierta. Los diferentes grupos de trabajo del OMG desarrollan estándares en un rango amplio de tecnologías.}:
\begin{quote}
UML es un lenguaje visual para la especificación, construcción, y documentación de los artefactos de un sistema. Es un lenguaje de modelado de propósito generalque puede ser usado con la mayoría de los métodos orientados a objetos y a componentes; que puede ser aplicado a todos los dominios de aplicación (p.e., salud, finanzas, telecomunicaciones, aeroespacial) y plataformas de implementación (p.e., J2EE, .NET). \end{quote} 

En \cite{jacobson2005} se recalca que UML es usado para entender, diseñar, buscar, configurar, mantener y controlar la información acerca de los sistemas.

Con UML se crean artefactos con información acerca de la estructura - o vista estática- y el comportamiento - o vista dinámica- de un sistema. Cada vista del sistema se modela como una colección de objetos que interactuan, es decir, que tienen interfaces y relaciones entre ellos perfectamente definidas. Fruto de tal relación entre objetos el sistema ofrece una funcionalidad o cumple un objetivo que es de interés.
\section{Portales de Información y Conocimiento}

Antes de la explosión de servicios a través de la Internet, los portales basados en aplicaciones web estaban recomendados solo para organizaciones que por su complejidad (en tamaño ó geografía) necesitaran de un sistema tecnológico para que todo su personal pudiese tener acceso a la información en forma compartida y simultánea. Sin embargo, fundamentado en el crecimiento del uso de Internet\footnote{El porcentaje de la población con acceso a Internet en Colombia a crecido de un 2,1\% en el año 2000 a un 15,8\% en el 2007, según la Comisión de Regulación de Telecomunicaciones.} surgue la necesidad de la sociedad por mantener cierto orden en la corriente de bytes y grupos de usuarios con intereses de información comunes empiezan a conglomerarse alrededor de portales temáticos no organizacionales.

Un portal de información no es, en esencia, una fuente nueva de información; es una vista de la información existente que dispuesta en una forma ordenada se convierte en una herramienta de conocimiento extraordinariamente poderosa permitiendo poner al descubierto información valiosa que se enmascaraba entre otra no menos interesante – Data Mining.
Dentro de las múltiples ventajas que ofrece el portal, es que proporciona la facilidad de obtener información actualizada a muy corto plazo, lo que es indispensable para la óptima toma de decisiones. Dicha información esta disponible, en condiciones óptimas,  veinticuatro horas al día, trescientos sesenta y cinco días del año, permitiendo así acceder a los datos que se necesitan de acuerdo con la disponibilidad singular de tiempo y apoyar la toma de decisiones bajo cualquier circunstancia y lugar.

Gran cantidad de organizaciones y grupos de usuarios están explotando el uso de los portales creando y transformando servicios y procesos tradicionales convirtiéndolos en servidores de autoservicio, pudiendo así dedicarse a aquellos de mayor valor agregado a la organización, el personal o el grupo de investigación. 

\subsection{Beneficios y obstáculos para la implementación de portales basados en aplicaciones web.}

Las facilidades que proporciona la tecnología, permite que el portal sea accedido a través de numerosas opciones, esto es a través de computadoras de escritorio y portátiles integradas a la red interna de la organización, a través de Internet por redes de banda ancha y estrecha y de los diversos medios inalámbricos como son las tecnologías celular, WiFi, WiMax, BlueTooh por intermedio de PDA, celulares y equipos de cómputo en redes WLAN.

Debido a la estructura del portal, se tiene una fuerte correlación entre diversas aplicaciones que nos permiten analizar interrelaciones que serían realmente complejas y tardadas si no se contara con ellos. Sin embargo, es importante recalcar más que los beneficios los problemas potenciales. De hecho en el éxito de un portal están enfocados factores clave que tienen beneficios y problemas asociados.

\begin{description}
 \item[Factor Humano] Los individuos adaptan los procesos de información en diferentes maneras. 
 \item[Factor Tecnológico] Intranets, pueden ser costosas y poco efectivas si la organización no tiene la tecnología necesaria para construirlas.
 \end{description} 

La principal ventaja obtenida al construir y mantener un portal basado en aplicaciones web es mejorar la eficiencia y efectividad en la comunicación de los miembros de una organización o un grupo de usuarios, lo que aumenta la objetividad en la toma de decisiones y la transferencia de conocimiento. Todo lo anterior se maximiza si el portal se concibe como fruto de un proceso de investigación en donde todos sus componentes y servicios se construyen, mantienen, distribuyen e integran de acuerdo a los requerimientos de los usuarios finales\cite{sarmento2005}.

Una de las características importantes de los portales es que en un sólo lugar - y con un mecanismo de acceso unificado, los usuarios pueden acceder a las aplicaciones. Esta integración con aplicaciones y servicios orientados al trabajo colaborativo hacen que trascienda los límites de un mero repositorio organizacional - que permite el autoservicio de requerimientos y extracción de información básica - y lo convierte en una herramienta de administración del conocimiento, útil para la toma de decisiones.

Las novedosas tecnologías que convergen en Internet permiten que la información sea personalizada y dirigida de tal forma que se potencian ciclos de creación, captura y diseminación de conocimiento necesarios para el crecimiento de los activos intangibles de los grupos y organizaciones. Así los portales convierten la información en valor, ya que eliminan las barreras de distancia y disponibilidad de información, reduciendo costos conectando a múltiples personas en diversos sitios al mismo tiempo.

La tabla \ref{beneficio} muestra los principales beneficios potenciales de desplegar los portales de información y conocimiento dentro del quehacer de las organizaciones, los grupos de trabajo y las comunidades de práctica.

\begin{table}
\begin{center}
\begin{tabular}{|l|}
\hline
\textbf{Beneficios Humanos (suaves)}\\
\hline
Provee estructura de soporte 24 hrs.\\
Servicio centrado en el usuario.\\
Medio ambiente amigable.\\
\hline
\textbf{Beneficios Físicos y capitales (beneficios fuertes)}\\
\hline
Creación medioambiente libre de papel.\\
Mejorar eficiencia y efectividad.\\
Reducción de costos.\\
Menores tiempos en consecución de información.\\
\hline
\textbf{Beneficios estratégicos}\\
\hline
Creación de herramientas innovadoras de apoyo.\\
Proveer información a tiempo real.\\
Apoyo a proceso de negocios de reingeniería\\
Apoyo a los ciclos de creación, captura y diseminación de conocimiento.\\
Aumento de Capital intangible.\\
Formalización del \textit{Know-How}.\\
\hline
\end{tabular}
\caption{Algunos Beneficios Potenciales al Implementar un Portal}
\label{beneficio} 
\end{center}
\end{table}

Los riesgos que se afrontan también son enormes y pueden llevar al traste cualquier política o proyecto de desarrollo; la tabla \ref{riesgos} muestra algunos de los más importantes que deben ser minimizados.

\begin{table}
\begin{center}
\begin{tabular}{|l|}
\hline
\textbf{De tipo Humano (Fuertes)}\\
\hline
Indiferencia de la administración.\\
Sobrevaloración del papel de las TIC\\
Dirección centrada en el capital financiero\\
Estructuras organizacionales conservadoras.\\
Micropoderes y feudos organizacionales autogestionados.\\
Ignorancia y/o resistencia respecto al uso de TIC.\\
Resistencia a estandarizar la información.\\
Resistencia a compartir información/conocimiento.\\
\hline
\textbf{Riesgos Físicos/Capital}\\
\hline
Procesos orientados a la técnica y no multidimensionales.\\
Carencia de capital(TIC marginales).\\
Dificultad de integración de tecnología nueva y la existente.\\
Ausencia de inter, trans y multidisciplinariedad.\\
\hline
\textbf{Riesgos tecnológicos (Débiles)}\\
\hline
Estándares propietarios.\\
Redes de interconexión da baja velocidad.\\
Alta relación Consumo/Adopción de tecnología.\\
\hline
\end{tabular}
\caption{Algunos riesgos Potenciales al Implementar un Portal}
\label{riesgos} 
\end{center}
\end{table}

\subsection{Ciclo de Vida de los Portales}
Los portales como representación sistemática del quehacer de un grupo humano evoluciona en la medida que dicho grupo mejora su conocimiento de las relaciones entre sus miembros y el entorno que los rodean. En general pueden determinarse cinco macro-etapas~\cite{egovernment} que aumentan gradualmente su funcionalidad basado en el  conocimiento organizacional y la interrelación de usuarios a través del Portal:

\begin{description}
\item[Presencia Emergente]
Esta es la etapa primaria por la que pasa un portal en donde su funcionalidad es la de distribuir información interesante para un grupo de usuarios, el cual es totalmente caracterizado por el grupo de personas que construye - en todas sus dimensiones, el portal. Dichos usuarios no intervienen directamente en la estructura del Portal el cual complementa sus servicios por medio de enlaces y dependencia a otros portales temáticamente relacionados.

\item[Presencia Mejorada] 
Los usuarios pueden determinar en cierto grado la navegación a través de búsquedas en el archivo del sitio. Un portal en esta etapa presentará gran cantidad de información que usualmente se agrupa por áreas temáticas. Los mapas del portal se distribuyen profusamente con el fin de guiar a los usuarios en su tránsito por el mismo y usualmente un sistema básico de ayuda prediseñada esta disponible.
 
\item[Interacción]
Los portales registran a sus usuarios. Se implementan herramientas en línea como las salas de charla - chat, las listas de correo y los foros; se realiza capacitación básica por medio de seminarios basados en contenidos y se hace uso extensivo de recursos multimediales. La ayuda es síncrona o asíncrona pero ágil lo que fomenta una depuración y actualización de la información contenida en el portal.

\item[Transacción] 
Los usuarios realizan operaciones a través del portal. El comercio electrónico, la gestión de contenidos, la personalización de los ambientes del portal, búsquedas semánticas y el despliegue de servicios avanzados - cursos, blogs, etc; caracterizan esta etapa.

\item[Transformación] 
La etapa más avanzada de los portales en donde se han estructurado comunidades de práctica sobre temas concretos que potencian los ciclos de conocimiento mediante las herramientas brindadas por el portal. Se observa una jerarquía \textit{ad hoc} de usuarios con base en su aporte. Ellos mismos generan contenido que es convalidado por la comunidad y los administradores técnicos limitan sus funciones a aquellas relacionadas con mantener operativa la plataforma tecnológica. Las transacciones y el contacto en tiempo real son rutinarios. Las aplicaciones son de conocimiento general y el nivel de inmersión en el portal es alto.
\end{description}


\subsection{Aplicaciones Web}

También conocidas como \textbf{WebApps} son, en su concepción más básica, aplicaciones que responden a peticiones realizadas por un usuario por medio de un navegador (cliente) y ejecutan la lógica del programa en un servidor. Las aplicaciones web usualmente interactúan con sistemas de bases de datos y distribuyen los resultados de sus operaciones en lenguajes estándar tales como HTML, SMIL, XML,  RDF, SVG, etc.\cite{jackson2005}. Las WebApps también se pueden encontrar en ambientes diferentes al modelo cliente - servidor\cite{bos2004}. 

Entre las características de las aplicaciones web se destacan\cite{bos2004}:

\begin{itemize}
\item \textbf{No requieren instalación.} En general las aplicaciones web no necesitan ejecutar rutinas de instalación en las máquinas cliente. Quizás en algunos lenguajes sea necesario la preparación de un ambiente específico de trabajo que en la mayoría de las veces es de acceso público.

\item \textbf{Accesibilidad.} Las aplicaciones web se despliegan desde de una página web. Los protocolos usados son estandarizados y abstraen fácilmente las capas de aplicación de las de diseño y datos.

\item \textbf{Facilidad en el Desarrollo.} Los lenguajes usados son de alto nivel, con un buen soporte para cadenas de caracteres, diferentes tipos de datos y con facilidades para la programación orientada a objetos. La mayoría de ellos con sintaxis similares y herramientas de desarrollo gratuitas de fácil adquisición. 

\item \textbf{Independencia de la Plataforma.} Las \textit{WebApps engines} implementan el modelo de capa intermedia lo que permite que las diferentes WebApps puedan ser desplegadas sobre diferentes plataformas sin detrimento de su funcionalidad. El uso de métodos genéricos definidos en interfaces de programación (API) ayuda en gran manera a garantizar esta característica.

\item \textbf{Seguridad.} No obstante la facilidad de acceso de la WebApp, estas pueden implementar rutinas avanzadas que brindan ambientes transaccionales seguros aislados del sistema de archivos y configuración del sistema en donde se alojan. El intercambio de información cifrada por la red y la integración con la seguridad de los servidores de bases de datos forman un contexto de alta seguridad.

\item \textbf{Privacidad.} Las WebApps pueden operar fácilmente sobre una Plataforma de Preferencias de Privacidad debido a que la mayoría de los motores están habilitados para soportar el protocolo \textbf{P3P}.

\item \textbf{Almacenamiento Persistente.} Tanto en el cliente - a través de archivos texto para el manejo de sesiones; como en el servidor de base de datos.

\item \textbf{Integración.} Las aplicaciones Web pueden brindar sus servicios - u obtener uno determinado, a través de interfaces claras y definidas en las denominadas redes de servicios Web.
\end{itemize}

\section{Aspectos Claves en la Gestión y Dirección del Conocimiento}

Como se aclaró anteriormente, OpenSITEM es una federación de diferentes tipos de sistemas software y sirve de soporte para el trabajo de los equipos de diseño de redes de eSalud del grupo de investigación GITEM (o similares). En esa línea, parte de su funcionalidad está enfocada en brindar apoyo a los flujos de trabajo de comunidades de práctica yu por tanto es necesario que se aborden aspectos relacionados con la gestión del conocimiento. A continuación se presenta la aproximación conceptual de está área.

\subsection{Acerca del conocimiento}

En la presente investigación conocimiento es entendido como un concepto que encierra, entre otras, las siguientes definiciones y se usarán de acuerdo a su contexto como partes complementarias de una misma meta – definición.

Partiendo de la definición clásica compilada por Gunter Dueck\cite{dueck2001}, el conocimiento es un concepto que puede tener componentes en una o varias de las siguientes dimensiones:

\begin{itemize}
\item  \textbf{Episteme} -\textit{ Dimensión abstracta – o metafísica}, en forma de generalizaciones, bases, leyes y principios científicos.

\item \textbf{Phronesis }– \textit{Dimensión práctica}, relativa al conocimiento pragmático discernido a través de las practicas aceptadas por la sociedad.

\item \textbf{Techne} – \textit{Dimensión técnica}, relativa a la forma de hacer las cosas, de la realización de actividades concretada en la forma de manuales, procedimientos y comunidades de práctica.

\item \textbf{Metis} – \textit{Dimensión objetiva}, como forma de volver corpóreo, real y sustancial la conjugación de los otros tipos de conocimiento.
\end{itemize}

Esta concepción multidimensional extiende y explica la taxonomía del conocimiento dada por los griegos:

“El conocimiento incluye restricciones implícitas y explícitas entre objetos (entidades), operaciones y relaciones, que permiten recoger heurísticas generales y específicas así como los procedimientos de inferencias relacionados con la situación a modelar”.\cite{sowa1984}

Otros autores despojan del sentido filosófico y colocan su definición en un plano simple y utilitario: “El conocimiento es información organizada y analizada para hacerla comprensible y aplicable a la resolución de problemas y toma de decisiones”.\cite{turban1992}. Si bien esta es una definición reduccionista, sirve de base para las propuestas de representación de conocimiento en documentos XML - comúnmente denominadas \textit{ontologías}. 


\subsection{Ciclo de Conocimiento}

Múltiples factores deben ser considerados cuando se trata de capturar, crear y diseminar el conocimiento dentro de un grupo de personas. La no homogeneidad en los medios de almacenamiento de conocimiento es uno de ellos. El conocimiento - en cualquiera de sus formas, puede estar guardado en diferentes partes que van desde entidades biológicas - mente humana, genes; a repositorios de conocimiento estructurado tales como ontologías, grafos de relación o mapas mentales.

Otro factor importante es la capacidad de acceso al conocimiento que es muy diferente al mero hecho de acceder a una fuente de conocimiento dada. En \cite{nokata1995} se considera que el conocimiento puede estar en dos estadios con propiedades diferentes: tácito y explícito. 

\begin{itemize}
\item \textbf{Conocimiento tácito:} Este conocimiento se corresponde con el conocimiento obtenido a través de la experiencia, conocimiento simultáneo  y conocimiento análogo. Esta forma de conocimiento usualmente se encuentra en medios de almacenamiento biológicos como la mente humana.

\item \textbf{Conocimiento explícito:} Se corresponde con el conocimiento racional, conocimiento secuencial, y conocimiento digital y se encuentra almacenado en documentos, bases de conocimiento, ontologías o cualquier otro medio abstracto de representación.
\end{itemize}

En \cite{liebowitz1998} se establece un tercer estadio llamado conocimiento implícito.

\begin{itemize}
\item \textit{Conocimiento Implícito:} Acceso directo mediante consulta y discusión. Requiere la localización y comunicación previa de conocimiento informal.
\end{itemize}

Un \textit{Ciclo de Conocimiento} es el proceso por el cual el conocimiento se transforma de tácito a explícito y viceversa por medio de las siguientes actividades:

\begin{itemize}
\item \textbf{Socialización:} Compartir conocimiento tácito entre individuos. El conocimiento permanece siendo tácito sin ser transformado en explícito. Este tipo de patrón no es muy interesante debido a su naturaleza tácita (Tácito - Tácito).

\item \textbf{Articulación:} Alguien transforma el conocimiento tácito en explícito (Tácito - Explícito).

\item \textbf{Síntesis:} Combinación de conocimiento explícito para crear nuevo conocimiento explícito  (Explícito - Explícito).
 
\item \textbf{Interiorización:} Proceso de transformar conocimiento explícito en tácito (Explícito - Tácito).
\end{itemize}

\begin{figure}
 \centering
 \includegraphics[width=156mm, height=156mm]{Ciclo_Conocimiento.png}
 \caption{Ciclo de Conocimiento}
 \label{ciclo_conocimiento}
\end{figure}

El flujo de conocimiento organizacional más importante es la transformación del conocimiento tácito en explícito \cite{davies2011}, esto es, la \textit{articulación} que se apoya en procesos avanzados de socialización. Esto permite acumular conocimiento explícito que puede ser compartido y accedido por los miembros de la organización. Por el contrario, la interiorización es el proceso natural llevado a cabo a través del aprendizaje individual por parte de los integrantes de la organización, esto es, la asimilación de conocimiento. 

El tercer flujo de conocimiento relacionado con la transformación de conocimiento es la \textit{combinación} o síntesis. En este caso se transfiere conocimiento a otra forma explícita de conocimiento. Un ejemplo sería el cambiar el formato de una base de conocimiento, agrupar ontologías o refinar heurísticas. Este tipo de flujo de conocimiento es importante para seleccionar, combinar y distribuir el conocimiento existente con diferentes fines. Por ser quizás el flujo de conocimiento más formal en el SITEM algunos componentes específicos implementan flujos de síntesis de conocimiento.

El cuarto flujo del ciclo de vida del conocimiento permite transformar conocimiento tácito en otras formas de conocimiento tácito mediante procesos de socialización. Un ejemplo de esto es cuando se transfiere conocimiento tácito de un experto a un ingeniero de conocimiento en una entrevista personal.

\subsection{Tecnologías del Conocimiento}

En la actualidad el conocimiento se considera un activo fundamental y como lo expresa \cite{gang2007}, tiene dos propiedades de vital importancia para contribuir al desarrollo de las organizaciones o comunidades de práctica:

\begin{itemize}
\item \textbf{Es explicable.} Cuando no se evidencia esta propiedad el conocimiento permanece tácito y en ninguna medida puede considerarse como perteneciente a la comunidad o la organización. Es entonces una tésis que solo aquel conocimiento que se ha convertido en explícito es el que posee una organización; de otra manera es propiedad exclusiva del individuo.

\item \textbf{Se puede comunicar y compartir.} Cuando no se toman medidas que formalicen estas actividades el conocimiento se pierde. Es decir, la organización o comunidad debe tener procesos conocidos que garanticen flujos de conocimiento basados en la síntesis, interiorización y socialización.
\end{itemize}

En general para que el conocimiento pueda generar ventajas competitivas debe ser gestionado de alguna manera. Esta necesidad dio surgimiento a dos áreas de investigacion comunmente agrupadas bajo el concepto de Tecnologías de Conocimento: Ingeniería de Conocimiento y Gestión de Conocimiento. 

Uno de los primeros problemas que deben atacar las tecnologías de conocimiento es el asociado con el Modelado de Conocimiento. En \cite{benafia2016} se expresan algunos principios a tener en cuenta cuando se modela el conocimiento:

\begin{itemize}
\item \textbf{Definición de roles de conocimiento}. El conocimiento se puede dividir en unidades atómicas que tienen propiedades irreductibles y que se asocian para lograr funciones que identifican dicha unidad.

\item \textbf{Identificación de tipos de conocimiento.} Cada unidad de conocimiento debe enmarcarse en uno o varios de los siguientes tipos de conocimiento: de tareas, inferencial, del dominio, ontologías del dominio, modelos del dominio. 

\item \textbf{Capacidad de ser compartido y reutilizado.} Las unidades de conocimiento pueden ser expresadas usando lenguajes y reglas formales. Lo que permite que pueda ser entendido por entidades con roles de conocimiento definidos.

\item \textbf{Uso de modelos gráficos.} La unidades de conocimiento pueden ser representadas mediante grafos tipo red en los cuales tanto los nodos como las rutas de interconexión son unidades atómicas de conocimiento.
\end{itemize}


\subsection{Gestión de conocimiento}

La gestión del conocimiento es de esos conceptos \textit{polisémicos} que no pocos autores \cite{girard2015}, \cite{firestone2001}, \cite{bergeron2003}, \cite{davenport1998}; entre muchos otros, pretenden erróneamente consolidar una definición, sin considerarla a partir de una visión sistémica que lo transcienda a lo que en investigación holística se comprende como sintagma\cite{hurtado2000}. La \textit{gestión de conocimiento}, como se percibe en la presente investigación, es “una unión sintagmática de diversos paradigmas” \cite{hurtado2000}. Fue \textit{Karl Wiig}, quien usó el término de \textit{gestión de conocimiento} por primera vez durante una conferencia en Suiza y a partir de ese momento diversos autores han conceptualizado el término surgiendo definiciones parciales tales como:

“La Gestión de Conocimiento es la construcción y aplicación sistemática, explícita y deliberada de conocimiento para maximizar la efectividad organizacional con respecto al conocimiento, por lo que usa sus activos de conocimiento”\cite{wiig1993}

“La Gestión de Conocimiento es el proceso de capturar experiencia colectiva organizacional donde ésta resida (por ejemplo, bases de datos, documentos, mentes humanas) y su distribución allá donde pueda ayudar a mejorar los resultados”\cite{hibbard1997}

“La Gestión de Conocimiento es la gestión y control explícito del conocimiento en una organización para lograr los objetivos de la organización” (van der Spek and Spijkervet, 1997)

Consecuente con la concepción de la gestión de conocimiento como un sintagma se puede concebir un paradigma asociado a un proceso con ciertas actividades implicadas: 

\begin{itemize}
\item Identificación y mapeo de bienes intelectuales de la organización.
\item Generación de conocimiento nuevo que permita obtener una ventaja competitiva. 
\item Recopilación accesible de información organizacional. 
\item Compartir buenas prácticas y tecnología, incluyendo técnicas de trabajo en grupo.
\end{itemize}

Este paradigma explora el conocimiento técnico, pragmático y objetivo considerando que la conjugación de leyes rígida e epistemológicas van en contravía de la dinámica misma de los sistemas. La articulación de este conocimiento debe ser mantenido de alguna forma en la organización, y de ahí surge el concepto de Memoria Corporativa. El saber hacer está completamente diseminado en la organización y debe ser integrado de forma coherente para facilitar el acceso al mismo y su reutilización, esto es, expresarlo en forma de memoria corporativa. 

Las memorias corporativas se consideran un elemento clave para gestionar el conocimiento porque facilitan su conservación, distribución y reutilización. Van Heijst define la memoria corporativa como una “representación de conocimiento e información organizacional explícita y persistente”, mientras que en (Nagenda y Plaza, 1996) se define como “los recursos colectivos de datos y conocimiento de una compañía, incluyendo experiencias en proyectos, experiencia en resolución de problemas, etc”. En (Abecker, 1998), una memoria corporativa es referida como “un contenedor que integra información contextual, documentos e información no estructurada, que facilita su uso y reutilización”.

%\section {Ontologias}


1.5 ONTOLOGÍAS

Tal como ocurre con el conocimiento, el concepto de ontología ha recibido múltiples definiciones a lo largo de la historia y es necesario aclarar que, una vez más y debido a sus múltiples usos, el término trasciende sus raíces etimológicas y filosóficas para convertirse en si mismo en un concepto con semántica de tipo contextual. Inicialmente para la filosofía una ontología es una 
“Parte de la metafísica que trata del ser en general y de sus propiedades trascendentales.” Diccionario de la Real Academia de la Lengua

“Ciencia o estudio del ser: específicamente, una rama de la metafísica relacionada con la naturaleza y las relaciones del ser; un sistema particular según el cual se investigan los problemas de la naturaleza del ser; esto es, filosofía fundamental”.

“Teoría relativa a los tipos de entidades y específicamente los tipos de entidades abstractas que se admiten en el lenguaje de un sistema” 

Q!ueda claro que el término ha sido tomado prestado de ls escuales filosóficas y es por ende allí en donde se puede obtener un contexto más apropiado que pèrmita concretar lo que, en el campo de la gestión de conocimiento, una ontología pretende englobar.

La “Oxford Companion of Philosophy” define ontología de la siguiente forma: 

 “Ontología, entendida como una rama de la metafísica, es la ciencia del ser en general,       abarcando aspectos como la naturaleza de la existencia y la estructura categórica de la       realidad. El término ontología tiene algunos usos adicionales en filosofía. En un sentido       derivativo se usa para referirse a un conjunto de cosas cuya existencia queda reconocida por una teoría o sistema de pensamiento. En este sentido se habla de la ontología de una teoría o de un sistema metafísico definido por tal ontología”

Y es por esto que la ontología puede ser concebida como una forma de reconocer – formalizar, la existencia de supuestos metafísicos tales como el conocimiento. Es esta formulación la que comúnmente es aceptada en el área de la inteligencia artificial sobre todo la expresada por Quine (Quine, 1961), quien dijo que todo lo que puede ser cuantificado existe.

La primera definición de ontología en Inteligencia Artificial apareció en (Neches et al, 1991):

“Una ontología define los términos básicos y relaciones que conforman el vocabulario de un área específica, así como las reglas para combinar dichos términos y las  relaciones para definir extensiones de vocabularios”

Una de las definiciones más extendidas es la dada por Tom Gruber (Gruber, 1993): 

 "Una ontología es una especificación explícita de una conceptualización. El término       proviene de la filosofía, donde una ontología es un recuento sistemático de la existencia. En sistemas de Inteligencia Artificial, lo que existe es lo que puede ser representado. Cuando el conocimiento de un dominio se representa mediante un formalismo declarativo, el conjunto de objetos que puede ser representado se llama universo del discurso. Esos conjuntos de objetos, y las relaciones que se establecen entre ellos, son reflejados en un vocabulario con el cual representamos el conocimiento en un sistema basado en conocimiento. Así, en el contexto de IA, podemos describir la ontología de un programa como un conjunto de términos. En tal ontología, las definiciones asocian nombres de entidades del universo del discurso con textos comprensibles por los humanos que describen el significado de los nombres, y axiomas formales que limitan la interpretación y buen uso de dichos términos. Formalmente, una ontología es una teoría lógica”
     
Gruber entiende por conceptualización “una interpretación estructurada de una parte del mundo que usan los seres humanos para pensar y comunicar sobre ella. Para un informático, una conceptualización podría ser la clasificación de sistemas informáticos atendiendo a su naturaleza física en sistemas hardware, sistemas software y sistemas firmware.” (Fernández, 2003). Además dicha conceptualización debe ser, de alguna forma, factible de ser compartida y entendida.

Nicola Guarino (Guarino, 1995) tratando de crear una definición concertada  expresó:

“Un punto de inicio en este esfuerzo clarificador será el cuidadoso análisis de la       interpretación dada por Gruber. El problema principal de dicha interpretación es que se basa en la noción de conceptualización. Una conceptualización es un conjunto de relaciones extensionales que describen un estado particular, mientras que la noción que tenemos en mente es intensional, esto es, algo como una rejilla conceptual al que le imponemos varios posibles estados ...En el sentido filosófico, podemos referirnos a una ontología como un sistema particular de categorías que representa una cierta visión del mundo. Como tal, este sistema no        depende de un lenguaje particular: la ontología de Aristóteles es siempre la misma,       independientemente del lenguaje usado para describirla. Por otro lado, en su uso más típico en IA, una ontología es un artefacto ingenieril constituido por un vocabulario específico para describir una cierta realidad, más un conjunto de supuestos explícitos concernientes al significado pretendido de las palabras del vocabulario. Este conjunto de supuestos tiene generalmente la forma de teorías lógicas de primer orden, donde las palabras del vocabulario aparecen como predicados unarios o binarios, respectivamente llamados conceptos y relaciones. En el caso más simple, una ontología describe una jerarquía de conceptos relacionados por relaciones de subsunción; en los casos más sofisticados, se añaden axiomas para expresar otras relaciones entre conceptos y restringir la posible interpretación.”

Más tarde Guarino modificaría su definición para abarcar conceptos más globales para la interpretación afirmando que: 

“Una ontología puede especificar una conceptualización en una forma muy indirecta, puesto que i) solo puede aproximar un conjunto de modelos pretendidos; y ii) tal conjunto de modelos pretendidos sólo es una caracterización débil de una conceptualización.”. (Guarino, 1998)

Borst (Borst, 1997), redefine la ontología de Gruber:

“Una ontología es una especificación formal de una conceptualización compartida.” 

A la par que (Studer et al, 1998) explica que:
“Conceptualización se refiere a un modelo abstracto de algún fenómeno en el mundo a través de la identificación de los conceptos relevantes de dicho fenómeno. Explícita significa que el tipo de conceptos y restricciones usados se definen explícitamente. Formal representa el hecho de que la ontología debería ser entendible por las máquinas. Compartida refleja la noción de que una ontología captura conocimiento consensual, esto es, que no es de un individuo, sino que es aceptado por un grupo” 

Con lo anterior se puede tener una visión adecuada de las ontologías – quizás no completa pero práctica, que permite asociar técnicas a la definición propia de estructuras que modelen y sirvan para expresar – socializar, el estado de conceptos abstractos.

1.5.1 Tipos de ontologías
      
En general las ontologías, referidas como medio para modelar sistemas abstractos, pueden ser clasificadas de acuerdo a ciertos criterios como son: (a) el tipo de conocimiento contenido; y (b) la motivación de la ontología.

1.5.1.1. Ontologías según el conocimiento contenido

Este es el criterio donde existe mayor diversidad, la cual puede ser ilustrada por las dos siguientes clasificaciones de ontologías. La primera de ellas fue propuesta en (Van Heijst et al, 1997), donde se distinguen tres tipos de ontologías: 
Ontologías terminológicas, lingüísticas: Especifican los términos usados para representar conocimiento en el dominio. Un ejemplo de este tipo de ontologías es la red semántica UMLS (Unified Medical Language System) (Lindberg et al, 1993).

Ontologías de información: Especifican la estructura de los registros de la base de datos. Los esquemas de bases de datos serían un ejemplo. 

Ontologías para modelar conocimiento: Especifican conceptualizaciones de conocimiento. Estas ontologías tienen una estructura interna mucho más rica que los anteriores tipos de ontologías, y éstas son las ontologías que interesan a los desarrolladores de sistemas basados en conocimiento.      

Una clasificación alternativa fue propuesta en (Mizoguchi et al, 1995), donde también se proponen tres categorías:

Ontologías del dominio: Contienen todos los conceptos asociados a un dominio particular.
Ontologías de tarea: Establecen la forma en la cual se puede usar el conocimiento del dominio para realizar tareas específicas. De esta forma, una aplicación podría realizar búsquedas de información mientras otra podría gestionar la asignación de bloques libre de memoria.
Ontologías generales: Contienen descripciones generales sobre objetos, eventos, relaciones    temporales,   relaciones    causales, modelos de  comportamiento y funcionalidades.

1.5.1.2. Ontologías por motivación para su creación

Inicialmente se distinguen cuatro tipos de ontologías:

Ontologías para la representación de conocimiento: Permiten explicar las conceptualizaciones que subyacen de los formalismos de representación de conocimiento (Davis et al, 1993).

Ontologías genéricas: Definen conceptos considerados genéricos en diferentes áreas. Ejemplos de tales conceptos serían componente, subclase, proceso, estado, etc. Estas ontologías son reutilizables en diferentes dominios. Se llaman también ontologías abstractas o superteorías porque permiten definir conceptos abstractos, y dichas ontologías pueden ser usadas para definir conceptos de forma más específica en diferentes dominios. Como ejemplos podemos ver la taxonomía, la mereología, la topología y la teoría general de sistemas.

Ontologías del dominio: Definen conceptualizaciones específicas del dominio. Las metodologías actuales de adquisición de conocimiento distinguen entre ontologías y conocimiento del dominio, porque el último describe situaciones factuales del dominio, mientras que las ontologías imponen descripciones sobre la estructura y contenido del conocimiento del dominio.

Ontologías de aplicación: Están ligadas al desarrollo de una aplicación concreta. Tales ontologías cubren los aspectos relacionados con aplicaciones particulares. Típicamente, estas ontologías toman conceptos de ontologías del dominio y genéricas, así como métodos específicos para realizar la tarea, por lo que no son muy adecuadas para ser reutilizadas.

Una clasificación alternativa fue propuesta por Poli (Poli, 2000). En dicha clasificación se identifican los siguientes tipos de ontologías: 

Ontologías generales: Tienen que ver con las categorías fundamentales y sus conexiones de dependencia. Con respecto a las categorías fundamentales, los investigadores se dan cada vez más cuenta de la dificultad de manejar este nivel supremo. Por ello, es de máxima importancia emplear una organización de categorías principales que sea lo más transparente posible. Existen categorías fundamentales que se aplican a todos los niveles ontológicos. Sin embargo, muchas de las categorías top- level pueden tener diferentes valores en niveles diferentes de la ontología, aunque deben  tener algo en común. 

Ontologías categóricas: Estudian las diversas formas en las que una categoría se da cuenta de los diversos niveles ontológicos, determinando la posible presencia de una teoría general que subsume sus concretizaciones. Mientras que la ontología general está más relacionada con la arquitectura de la teoría, la ontología categórica es más sensible a los detalles de las categorías individuales. Sin embargo, es obvio que ambas son necesarias.

Ontologías del dominio: Se refieren a la estructuración detallada de un contexto de análisis con respecto a los subdominios que lo componen. 

Ontologías genéricas: Parecen ligadas a corpus lingüísticos y léxicos conceptuales. De hecho, se pueden clasificar los términos en varios niveles. Esto significa que cada término debería ser accesible por defecto únicamente en su sentido genérico, mientas que sus significados especializados quedan para cuando se active una ontología del dominio específica. Por otro lado, la ontología del dominio contiene términos que no tienen correspondencias analíticas en ontologías genéricas. El conocimiento del dominio “satura” el conocimiento genérico.

Ontología regional: Analiza las categorías y sus conexiones de interdependencia para cada nivel ontológico (estrato o capa). 

Ontología aplicada: Estas ontologías son la aplicación concreta de entorno ontológico a un objeto específico (por ejemplo, un proyecto).

 1.5.2. Ontologías de tipo Formal y Descriptiva

Una tercera clasificación se basa en el grado de formalidad de la ontología. Según este criterio se distinguen tres tipos de ontologías en (Poli, 2002): 

Ontología descriptiva, relacionada con la recolección de información sobre los ítems del dominio analizado. La unidad y variedad del mundo es la salida de las conexiones de dependencia y formas de independencia entre los ítems. Cosas materiales, plantas y animales, así como los productos de los talentos y actividades de animales y humanos, son ítems del mundo. 

En otras palabras, el mundo no solo contiene cosas, animadas o no, sino también actividades y procesos, así como los productos derivados de los mismos. Es difícil negar que existen pensamientos, sensaciones y decisiones, así como el completo espectro de actividades mentales, así como uno está obligado a admitir la existencia de reglas, lenguajes, sociedades y costumbres (Poli,2001a).        

Ontología formal, que destila, filtra, codifica y organiza los resultados de una ontología descriptiva. Según esta interpretación, la ontología formal es formal en el sentido usado por Husserl es sus “Logical Investigations”. Ser formal en este sentido implica tratar con categorías como cosa, proceso, materia, forma, todo, parte, etc. Estas categorías caracterizan aspectos y tipos de realidad que todavía no han sido utilizados bajo ningún formalismo.        

La codificación formal en sentido estricto se da al nivel de ontología formalizada. El objetivo es encontrar la codificación formal apropiada para los constructores adquiridos de forma descriptiva y purificarlos formalmente como se indica. El nivel de construcciones formalizadas también está relacionado con la evaluación de la adecuación (expresiva, computacional, cognitiva) de los distintos formalismos, y con el problema de las traducciones recíprocas. La fuerte similaridad entre los términos “formal” y “formalizada” es un contratiempo. Una forma de evitar la confusión es utilizar “categórica” en vez de formal. La mayor parte de las teorías contemporáneas sólo reconocen dos niveles de análisis y suelen unir las categorías formales con el análisis formalizado. Como consecuencia, se suele negar la relevancia específica de los análisis categóricos. 

Los tres niveles ontológicos son diferentes pero no están separados, puesto que están relacionados en muchos aspectos. El conocimiento descriptivo puede referirse a categorías formales, y las salidas formalizadas a los otros dos niveles. Por otro lado, es más delicado establecer las diferencias y conexiones entre varias facetas ontológicas como se muestra en (Poli, 2002a).        

La aplicación de métodos lógico-formales a una ontología la transforma en ontología formal. Los primeros ontólogos formales creían que la tarea de construcción podía ser llevada a cabo de forma sistemática y está completamente basada en la resolución de problemas lógicos, esto es, en la gramática lógica de lenguajes particulares. En contraste, la antigua tradición ontológica se ha quedado en un almacén de intuiciones ontológicas, constituyendo argumentos informales e incluso retóricos sobre esas intuiciones como base. Como se establece en (Gangemi et al, 1999),       las relaciones formales implican entidades de todas las esferas materiales, de forma que son comprensibles per se como nociones universales. Por el contrario, las relaciones materiales son específicas de una o más esferas materiales. Esto presupone una división a priori del dominio en esferas materiales: primero se debe realizar una distinción entre relaciones formales y materiales con base a su comportamiento con respecto a tales subdominios. De esta forma, las relaciones formales establecen las conexiones y las diferencias entre subdominios primitivos, mientras que las relaciones materiales caracterizan las propiedades de un subdominio específico. Si se asume un dominio plano, sin estructura a priori, entonces no sería válida la distinción entre relaciones formales y materiales. 

1.5.3 Ingeniería Ontológica

Las ontologías proporcionan un vocabulario común de un área y definen, a diferentes niveles de formalismo, el significado de los términos y relaciones entre ellos. El conocimiento en ontologías se formaliza principalmente usando cinco tipos de componentes: clases, relaciones, funciones, axiomas e instancias (Gruber, 93).

Las clases en la ontología se suelen organizar en taxonomías. Algunas veces, la noción de ontología se diluye en el sentido que las taxonomías se consideran ontologías completas [Studer et al.; 98]. Se suele usar tanto el término clases como conceptos. Un concepto puede ser algo sobre lo que se dice algo y, por lo tanto, también podría ser la descripción de una tarea, función, acción, estrategia, proceso de razonamiento, etc. 

Las relaciones representan un tipo de interacción entre los conceptos del dominio. Se definen formalmente como cualquier subconjunto de un producto de n conjuntos, esto es: 
R: C1 x C2 x ... x Cn.
Como ejemplos de relaciones binarios incluimos: “subclase de” y “conectado a”. Las funciones son un tipo especial de relaciones en las que el n-ésimo elemento de la relación es único para los n-1 precedentes. Formalmente, definimos las funciones F como: 
F: C1 x C2 x ... x Cn-1 x  Cn. 
Como ejemplos podemos mencionar las funciones “madre de” y “precio de un coche usado”. 

Los axiomas son expresiones que son siempre ciertas. Pueden ser incluidas en una ontología con muchos propósitos, tales como definir el significado de los componentes ontológicos, definir restricciones complejas sobre los valores de los atributos, argumentos de relaciones, etc verificando la corrección de la información especificada en la ontología o deduciendo nueva información. Tales ontologías son llamadas ontologías pesadas, en contraste con las ontologías ligeras que no incluyen axiomas. 

Las instancias se usan para representar elementos específicos.

1.5.3.1. Relaciones

En (Gómez-Pérez et al, 2000) se enumeran las relaciones más comunes en dominios reales, a saber: equivalencia, taxonómica, partonómica, dependencia, topológica, causal, funcional, cronológica, similaridad, condicional y propósito. Sin embargo, no todas las relaciones tienen la misma relevancia ni imponen el mismo tipo de propiedades jerárquicas a la ontología. Entre este conjunto de relaciones podemos subrayar tres de ellas: taxonomía, mereología, y topología. Taxonomía.

La palabra taxonomía tiene su origen en dos términos griegos, a saber, taxis (orden) y nomos (tratado) y esta palabra proviene de la Filosofía. Taxonomía es la ciencia que estudia la división en grupos ordenados o categorías. Desde un punto de vista ontológico, una taxonomía es una organización ontológica basada en una relación de orden parcial llamada IS-A, a través de la cual se agrupan las entidades y son subsumidas por clases de más ato nivel. En general, las taxonomías han sido importantes para modelar esquemas de bases de datos, sistemas basadas en
conocimiento y vocabularios semánticos (Guarino and Welty, 2001).

A continuación se presentan las propiedades satisfechas por las relaciones taxonómicas. Con este propósito, se usará la notación empleada en (Guarino and Welty, 2001). De esta forma, se dice que un individuo x perteneciente a una clase OJO::::

Asimetría: Esta propiedad significa que la inclusión de una clase de individuos, X, en una clase Y implica la no inclusión de Y en X. Formalmente, esta propiedad garantiza que: (X es a Y) si y solo si no ocurre que (Y es a X).

Transitividad: Sea X incluido en una clase Y, que a su vez está incluido en una clase Z, ambas inclusiones a través de relaciones.

Irreflexividad: Admitir la reflexividad en relaciones taxonómicas solo tendría sentido para modelar tautologías. Una tautología es la expresión de un mismo hecho de distintas maneras. La relación taxonómica se considera no reflexiva.

Existen otras propiedades taxonómicas que están relacionadas con los atributos de los conceptos a través de la taxonomía:

Redefinición: Esta propiedad consiste en cambiar el nombre de una propiedad común a dos conceptos, padre e hijo, y se asigna un nombre diferente al atributo en el hijo. 

Herencia múltiple: Esta propiedad está asociada con atributos conceptuales. Un concepto puede tener diferentes padres taxonómicos, así que este concepto heredaría propiedades de todos sus padres. 

Además de las propiedades taxonómicas básicas, existen otras condiciones basadas en cuestiones filosóficas relacionadas con taxonomías. Algunas de estas condiciones se señalan en (Guarino and Welty, 2001):

Identidad
Unidad
Esencia
Dependencia
Rigidez

Las dos primeras condiciones se enlazan al concepto filosófico de “ser”. Según Guarino, las intuiciones tras ambos conceptos requieren, con la finalidad de comprenderlos, hacer una distinción entre ellos. Así, la condición de identidad se relaciona al problema de distinguir una instancia de su clase específica de instancias de la misma clase, por medio de lo que llamamos “propiedad característica”, la cual es única para cada instancia.


\subsubsection{Sistemas de Gestión de Conocimiento}

En IA, las bases de conocimiento son generadas para ser consumidas en sistemas expertos y basados en conocimiento, donde las computadoras usan inferencias para responder a cuestiones de usuario. Aunque es importante la adquisición de conocimiento para inferencias computacionales, en los desarrollos más recientes en Gestión del Conocimiento, el conocimiento queda disponible para consumo humano directo o para desarrollar software que procese dicho conocimiento. 

Históricamente, la Gestión de Conocimiento se ha centrado en un único grupo a través de lo que generalmente se ha conocido como sistema de información ejecutiva (EIS), que contiene un conjunto de herramientas para acceder a bases de datos, generar alertas, etc para apoyar el proceso de toma de decisiones. Más recientemente, se ha comenzado a diseñar sistemas de Gestión de Conocimiento para organizaciones completas. Si los ejecutivos necesitan acceder a la información y al conocimiento, es probable que sus empleados tengan interés en esa información.
 
De acuerdo con (O´Leary, 1999) citado por (Valencia, 2005), las principales funciones de un sistema de gestión de conocimiento son facilitar:

\begin{itemize}
\item La conversión de datos y texto en conocimiento;
\item La conversión de conocimiento individual y de grupo en conocimiento explícito;
\item La conexión de individuos y conocimiento a otros individuos y conocimientos;
\item La comunicación de información entre diferentes grupos;
\item La creación de nuevo conocimiento útil para la organización;
\end{itemize}

\subsubsection{Sistemas Integrados para el Soporte de Desempeño}

Sistemas que integran múltiples fuentes y herramientas de gestión de conocimiento en un único ambiente de trabajo para apoyar de una manera más efectiva las tareas relacionadas con: (Winslow and Bramer, 1994)

\begin{itemize}
\item \textbf{Infraestructura:} Organización y estructura del entorno de trabajo. 
\item \textbf{Control:} Monitorización, coordinación y control.
\item \textbf{Navegación:} Interacción hombre-máquina.
\item \textbf{Presentación:} Posibilidad para personalizar datos y servicios.
\item \textbf{Adquisición:} Captura conocimiento, casos, opiniones, aprendizaje y datos sensoriales en diferentes medios y su transformación en formato interno.
\item \textbf{Consultoría:} Consultar servicios, asistencia y recordatorios.
\item \textbf{Instrucción:} Ayuda y entrenamiento.
\item \textbf{Aprendizaje:} Aplicación de técnicas de descubrimiento de conocimiento y minería de datos.
\item \textbf{Evaluación:} Valoración y certificación basada en medidas del rendimiento y la calidad.
\item \textbf{Referencia:} Constituyen fuentes de conocimiento y experiencia para la organización.
\end{itemize}

Estos sistemas se han convertido en la actualidad en una necesidad debido al crecimiento desmesurado de aplicaciones incompatibles y protocolos no estandarizados. 


\chapter{OpenSITEM: Sistema Federado de Aplicaciones para la Caracterización de Nodos Potenciales de Redes de e-salud}

\textbf{OpenSITEM }es un sistema federado de aplicaciones de software libre o de código abierto que provee herramientas para analizar datos e información de los siguientes elementos que son de interés para la descripción - y definición de capacidad de, nodos potenciales de redes de e-salud: entidades de salud, servicios médicos, tecnologías de interconexión, operadores de telecomunicaciones, equipos médicos, organizaciones, profesionales, estándares, pacientes, enfermedades, medicamentos y proyectos. Provee un ambiente para apoyar las tareas de las comunidades de práctica involucradas en la investigación, el diseño, mantenimiento, desarrollo e implementación de redes de eSalud. Tuvo su génesis conceptual en la primera fase del Proyecto Telemedicina Bogotá como solución a la necesidad de administrar los resultados del estudio de campo realizado a las entidades e instituciones de salud y los operadores de Telecomunicaciones en la ciudad de Bogotá.

Su principal objetivo es: Implementar un sistema que permita la definición, categorización y caracterización de nodos potenciales de redes de eSalud, para apoyar las actividades básicas de los \textit{trabajadores del conocimiento} en el área de la telemedicina del grupo GITEM\footnote{Conformado por profesionales y estudiantes de la Universidad Distrital así como por profesionales de las diferentes instituciones que han participado en los diferentes estudios de campo.}  ofreciéndoles, además de un repositorio de datos, herramientas que facilitan las tareas de capturar, extraer, organizar, analizar, encontrar, sintetizar, distribuir y compartir información y conocimiento de nodos potenciales de las redes de eSalud. OpenSITEM cuenta con herramientas para poder  actualizar el estado de los nodos definods en el modelo base así como para la definición de nodos y categorías inéditas del Sistema de Salud de Bogotá Distrito Capital. El modelo base de nodos y categorías se construye a partir del estudio de campo realizado por el grupo de investigación, haciendo especial énfasis en los requerimientos de eSalud que se definieron en esa época.Con OpenSITEM se sistematiza los resultados de los estudios de campo emprendidos por el grupo de investigación GITEM y se constituye una plataforma para gestionar los datos de nodos potenciales de redes de eSalud. Contribuye a disminuir el tiempo de adquisición, análisis y despliegue de la información.  
\begin{figure}
 \centering
 \includegraphics[width=142mm, height=190mm]{sitem_principal.png}
 \caption{Imagen en el año 2017 de la página principal de OpenSITEM}
 \label{pantalla_sitem}
\end{figure}

El proceso de integración del sistema\footnote{OpenSITEM es un sistema federado de aplicaciones.} está basado en métodos ampliamente conocidos \cite {balduino2010}, \cite{koch},\cite{jacobson2000},\cite{larman2004}. Para el diseño de los módulos inéditos se ha tenido en consideración principios, patrones y antipatrones, tratando de minimizar los riesgos asociados a la mala calidad del software.

OpenSITEM hace uso extensivo de aplicaciones existentes, marcos de trabajo, bibliotecas, APIs, servicios web y plantillas, lo que ha permitido lograr un alto grado de funcionalidad específica. Los módulos propios\footnote{En referencia a la autoría, no al carácter de código abierto.} - aquellos que hacen parte de la suite desarrollada por GITEM - se implementan sobre el framework OpenSARA \footnote{Diseñado y construido por GITEM. OpenSARA es un producto de este proyecto y se ha usado en otros dominios tanto en la Universidad Distrital (sistema de gestión de inventarios, sistema de consultas a comunidades, sistema de evaluación para acreditación, entre otros), como por algunas empresas del sector TI (OpenKyOS).}.

\section{Módulos Funcionales}

OpenSITEM- incluyendo las aplicaciones de terceros y las desarrolladas por GITEM, ofrece los siguientes módulos base:

\begin{table}[]
\centering
\caption{My caption}
\label{my-label}
\begin{tabular}{lcl}
\rowcolor[HTML]{C0C0C0} 
\multicolumn{1}{c}{\cellcolor[HTML]{C0C0C0}\textbf{Módulo}} & \textbf{Elaboración} & \multicolumn{1}{c}{\cellcolor[HTML]{C0C0C0}\textbf{Nombre}}            \\
Motor de Federación de Aplicaciones                         & Propia               & OpenSITEM-AF                                                           \\
Gestión de Nodos                                            & Propia               & OpenSITEM-NM                                                           \\
Motor de recomendación                                      & Propia               & OpenSITEM-RS                                                           \\
Gestión de Encuestas                                        & Propia               & OpenSITEM-PM                                                           \\
Inteligencia de Negocio                                     & Tercero              & Knowage                                                                \\
Sistema de Información Geográfica                           & Tercero/Propia       & \begin{tabular}[c]{@{}l@{}}Cesium\\ qGIS\\ Leaflet\end{tabular}        \\
Gestión Documental                                          & Tercero              & Alfresco                                                               \\
Gestión de Reportes                                         & Tercero/Propio       & \begin{tabular}[c]{@{}l@{}}Reportico\\ Varias bibliotecas\end{tabular}
\end{tabular}
\end{table}

\subsection{Características innovadoras de OpenSITEM}


OpenSITEM presenta innovaciones en diferentes dominios:

\begin{itemize}
 \item Dominio de Aplicación: Desarrolla un motor y un proceso para la federación de aplicaciones. 
 \item Dominio de arquitectura: Propone un modelo de arquitectura para nodos en redes de eSalud.
 \item Dominio de utilidad: Implementa una plataforma para soportar flujos de trabajo de investigadores del GITEM. En el mercado no existe una herramienta que de manera unificada cumpliera con el modelo de requerimientos definido.
 \item Dominio social: Tanto el framework (OpenSARA) como los módulos de OpenSITEM están disponibles en repositorios públicos y cobijados por licencia de código abierto. El framework ya se ha utilizado por equipos de trabajo externos al grupo GITEM. Se tiene presupuestado que la fase IV del proyecto (transición) permita que los datos también sean abiertos. 
\end{itemize}


\section{Descripción de la Arquitectura de OpenSITEM}

OpenSITEM es una aplicación federada de arquitectura orientada a servicios, con características tales como las descritas en {earl2017}. La integración es realizada por un motor que permite el intercambio de mensajes y la sincronización de sesiones entre los diferentes aplicativos federados sin llegar a considerar una arquitectura de Bus de Servicio Empresarial (ESB por sus siglas en inglés)\footnote{Aunque existen ESB de corte empresarial tales como Mule, WSO2, Apache Service Mix o similares, el análisis de utilidad mostró que la mayoría de funcionalidades que ofrecen nunca serían utilizadas y entonces se optó por la simplicidad arquitectónica y evitar la tarea recurrente de configuración y administración del ESB. Ver el anexo \ref{appendix:lista_de_chequeo_ESB}.} 

El presente informe describe la arquitectura general de OpenSITEM y de los módulos inéditos realizados en el marco del proyecto. Las aplicaciones federadas construidas por terceros se abstraen, visibilizando únicamente las interfaces provenientes de las API o de los servicios web de adaptación. 

\subsection{Arquitectura General}






\subsection{Motor de Integración}

Un primer análisis de la arquitectura general lleva a considerar la capa de integración como un Bus de Servicio Empresarial. \footnote{En el presente documento un Bus de Servicio Empresarial es considerado un estilo arquitectónico}









\subsection{Gestión de nodos}

Un requisito no funcional de OpenSITEM es la capacidad de mantenimiento


\subsection{Motor de Recomendación}


\subsection{Gestión de Encuestas}






\section{Modelo de Nodos}

Como se presentó anteriormente, el motor de gestión de nodos permite la definición de cualquier tipo de nodo. No obstante, el equipo de trabajo ha definido un modelo base de nodos que se consideran el conjunto reducido que permite describir los elementos primordiales \footnote{En es te caso el carácter de \textit{primordial} fue definido por el grupo de interesados del GITEM.}

















\subsubsection{Subsistema Entidades de Salud} 
Este subsistema, figura \ref{entidades}, se creó para gestionar los datos recopilados en el estudio de campo realizado por el grupo GITEM en el marco del proyecto del Sistema de Gestión de Salud para el Distrito Capital fases I y II. Es por ende el subsistema base para el SITEM y su objetivo principal es la gestión de información referente a las entidades de salud en el entorno colombiano enfocándose en tres redes principales: la red de especialidades médicas, la red de comunicaciones y la red de atención.

\begin{figure}
 \centering
 \includegraphics[width=80mm, height=52mm]{entidades.png}
 \caption{Arquitectura Básica del Subsistema Entidades}
\label{entidades}
\end{figure}

La información disponible en el subsistema puede ser administrada por cada una de las entidades prestadoras de servicios de salud, de tal forma que se cree gradualmente un catálogo flexible para conocer el estado actual de las entidades y su potencialidad para ser parte en redes que presten servicios de salud a distancia usando TIC. Por defecto, las entidades se asocian a la arquitectura de red de la secretaría de Salud de Bogotá pero por medio del módulo denominado \textit{redes de atención} se puede crear fácilmente cualquier prototipo de red jerárquica de atención \cite{yellowlees}.

\subsubsection{Subsistema Tecnologías de Interconexión} 
Administra información relacionada con las tecnologías y protocolos de interconexión disponibles en las redes de acceso y transporte. Estas tecnologías se ordenan principalmente sobre el modelo de referencia OSI pudiéndose crear  - desde el módulo de arquitecturas, cualquier otro tipo de modelo. En la actualidad se tiene como alternativa de clasificación el modelo de TCP/IP. Es una guía técnica que muestra la información de las capas físicas, de enlace y de red en formatos básicos- o de características generales; e intermedios - o de características técnicas.

El objetivo principal que se persigue con la implementación de este subsistema es proveer a los analistas información para la revisión sistemática de las diferentes opciones que brindan los fabricantes de dispositivos y así proyectar redes que sean técnicamente viables. La información se estructura de acuerdo a indicadores cualitativos y cuantitativos que permiten evidenciar el carácter de interoperabilidad, impacto y permanencia de la tecnología en el mercado. 

\subsection{Subsistema Equipos y Tecnologías}
Información técnica sobre los diversos equipos y tecnologías usadas en Telemedicina. Posee secciones para la gestión de Información especializada de proveedores y fabricantes, así como la gestión de especificaciones técnicas, funcionales y físicas de los equipos.

Combinado con los demás subsistemas provee un mecanismo eficiente para realizar auditorías \textit{ex-ante} en redes tecnológicas, valorando alternativas de intercambio y reposición en los nodos.

\subsubsection{Subsistema Operadores de Telecomunicaciones}

\begin{figure}
 \centering
 \includegraphics[width=80mm, height=52mm]{operadores.png}
 \caption{Arquitectura Básica del Subsistema Operadores de Telecomunicaciones}
 \label{operadores}
\end{figure}

Contiene información relacionada con operadores de telecomunicaciones, figura \ref{operadores}, con énfasis en las características técnicas de los servicios que ofrecen, su cobertura y tarifas. Brinda a los analistas información comparativa entre operadores lo que permite determinar las ventajas y desventajas entre diferentes opciones de interconexión, de acuerdo a los servicios médicos que se quieren implementar. Este módulo se complementa con la información de los subsistemas de tecnologías de interconexión y servicios médicos, así como de la información de dominio público que muestra el \textit{Sistema de Información Unificado para el sector de las Telecomunicaciones} mantenido por la Comisión Reguladora de Telecomunicaciones.

\subsection{Subsistema Organizaciones y Proyectos} 

Posee herramientas informáticas para la gestión de información de proyectos nacionales y de la región en el ámbito de la Telemedicina, la Telesalud y la Tele - educación en medicina. 

Incluye secciones para la gestión de los datos correspondientes a organizaciones y grupos de investigación que trabajen en el área de la Telemedicina.  Este subsistema es uno de los pilares del SITEM pues en él se despliegan aplicaciones que fomentan el trabajo en grupo y se realiza la captura de las experiencias adquiridas en los diferentes proyectos desarrollados en el área. Así mismo el módulo posee herramientas de edición que permiten la sincronización del trabajo en la realización de estudios de factibilidad y estudios técnicos.

En la actualidad se está desarrollando la opción de generar automáticamente matrices comparativas sobre criterios predefinidos, lo que brinda una fuente de información para el seguimiento de buenas prácticas en el desarrollo de proyectos en el área.

\subsection{Subsistema Servicios Médicos} 
Este subsistema cuya arquitectura se muestra en la figura \ref{servicios}, contiene una guía catalogada de los diferentes servicios y especialidades médicas disponibles. Se pone especial atención en la descripción detallada de los requerimientos técnicos y tecnológicos que requiere cada especialidad así como el perfil de los profesionales y entidades educativas de formación de especialistas. Algunos componentes de este subsistema permiten la gestión de información - con base en la normatividad colombiana e internacional, relativa a procedimientos, medicamentos, enfermedades, laboratorios, etc. Esto lo convierte en un elemento de apoyo de una plataforma de servicios como puede ser la de consulta remota, el telediagnóstico o la tele - educación médica.

\begin{figure}
 \centering
 \includegraphics[width=80mm, height=60mm]{servicios.png}
 \caption{Subsistema Servicios Médicos}
 \label{servicios}
\end{figure}

\subsection{Subsistemas Entornos de Aprendizaje} 
Siguiendo la filosofía de integración del SITEM a proyectos de software libre, el subsistema de entornos de aprendizaje incorpora y adapta los elementos de la plataforma \textbf{Moodle} y provee un ambiente en línea para la estructuración, mantenimiento y distribución de conocimiento a través de cursos y seminarios. 

El subsistema permite libre acceso de usuarios a un conjunto de cursos y seminarios que apoyan el desarrollo de competencias en el área de la teleinformática, la telesalud, la telemedicina, las tecnologías de la información y las comunicaciones. Con un enfoque constructivista se procura la retroalimentación de los contenidos por parte de la comunidad. 

Como patrón de desarrollo el SITEM integra a su arquitectura, figura \ref{aplicaciones_sitem}, soluciones exitosas y robustas en el mundo del software libre, de esta forma reutiliza gran cantidad de aplicaciones, las adapta para proveer un ambiente integrado y aumenta sus prestaciones para implementar nuevos casos de uso. Entre las aplicaciones, mostradas en la figura \ref{aplicaciones_sitem}, que contribuyen enormemente en el SITEM se pueden citar:

\begin{description}
 \item[Moodle] Es un ambiente integrado de aplicaciones para la creación, organización, mantenimiento y seguimiento de cursos en línea. El fin primordial de sus herramientas es dar soporte a un marco de educación social constructivista.

\item[PHPBB] Conjunto de módulos escritos en PHP y Python para la gestión de foros en línea; aunque su línea funcional base hace parte de Moodle, en el SITEM los foros de carácter general - aquellos que no pertenecen a un entorno de aprendizaje específico, se implementan usando PHPBB. El manejo de sesiones, la gestión de usuario y la autenticación se han recodificado para que sean compatibles con los subsistemas principales del SITEM.

\item[MediaWiki] Herramienta para la construcción de sitios web tipo Wiki. Un Wiki es un neologismo basado en un término hawaiano y hace referencia a un sitio en el cual el contenido se construye de forma colaborativa usando para ello un lenguaje de etiquetado intermedio que permite la aplicación de cierta plantillas a porciones de texto para poder darles un formato específico. Aunque la información de una página Wiki no esta estructurada, el sistema lleva un historial de cambios por lo que será posible reconstruir el estado de dicha página en cualquier momento de su vida. Este aplicativo esta siendo usado en el SITEM para la construcción de su enciclopedia, los manuales de usuario y en ciertos módulos que requieren la construcción colaborativa de contenidos.

\item[MapServer] Es una aplicación desarrollada en Python que proporciona al SITEM los servicios básicos necesarios para la georeferenciación de la información en el subsistema de consultoría. La plataforma de MapServer define un conjunto de bibliotecas que permiten el manejo de información geográfica. Es software libre y soporta entre otros los formatos: ESRI shapefiles, PostGIS, ESRI ArcSDE, GML; utilizando la librería OGR (http://www.gdal.org/ogr/, 2007).

\item[Google] Su servicio web de motor de búsqueda se usa en el SITEM. Esta es una solución parcial que a corto plazo será desplazada por herramientas de software libre tales como DataparkSearch (http://www.dataparksearch.org/, 2007), ASPSeek (http://www.aspseek.com) y algunas bibliotecas en desarrollo por los integrantes del GITEM.

\item[Sistema de Información Unificado del Sector de Telecomunicaciones] Conocido como SIUST, es un aplicativo Web desarrollado por la Comisión Reguladora de Telecomunicaciones que contiene información del sector de las telecomunicaciones en Colombia: “...información técnica de infraestructura, normatividad del sector, estadísticas comerciales e índices financieros de los prestadores de servicios y los indicadores de gestión del sector entre otros.” \footnote{Tomado del sitio web del SIUST. http://www.siust.gov.co/siust/}

La información, que es de acceso público, permite que el SITEM se nutra de ella para complementar y validar sus propias bases de datos en algunos subsistemas.

\item[Wikipedia] Quizás la fuente de información colaborativa más grande en Internet, debido a que sus contenidos son de uso libre, el SITEM se nutre de ellos y a su vez los complementa. A partir de información disponible se han editado más de 50 artículos en Wikipedia que tienen relación temática con el SITEM.

 \end{description}

\begin{figure}
 \centering
 \includegraphics[width=156mm, height=156mm]{sitem_aplicaciones.png}
 \caption{Aplicaciones de Software Libre o Uso libre que complementan al SITEM}
 \label{aplicaciones_sitem}
\end{figure}

Es un objetivo claro del proyecto la integración con otros productos de software libre promoviendo el uso de tales herramientas, el desplazamiento del mercado hacia soluciones basadas en productos de código abierto como estrategia válida para el despliegue de aplicaciones en Telemedicina en entidades que tengan bajos recursos para el montaje y mantenimiento de redes de atención soportas en TIC.

\section {Aspectos Relativos a la Fase de Transición}

El actual despliegue de la solución en una plataforma tecnológica adecuada tal como se muestra en el anexo \ref{modelo_despliegue}, garantiza un óptimo servicio a los potenciales usuarios \footnote{La versión actual esta disponible en Internet en la dirección http://gitem.udistrital.edu.co/sitem/} y marca el paso de la versión 3.0 del producto a la fase de transición. Las herramientas básicas se encuentran disponibles para que el grupo de investigación convoque a las entidades de salud, profesionales en el área de la tecnología, especialistas en medicina y fabricantes - distribuidores - de dispositivos médicos que participaron en la primera fase del Estudio Red de Telemedicina Bogotá \cite{aparicio2000} para que de forma conjunta enriquezcan la base de información en el subproceso de prueba piloto.

Basado en el módulo de generación de herramientas para la recolección de información, anexo \ref{manual_usuario}, el grupo aplicará diferentes instrumentos entre los que se incluyen:

\begin{itemize}
\item Entrevistas: Con el objeto de acordar las directrices a seguir tanto con los operadores de telecomunicaciones, como los prestadores del servicio de salud que participaron de la fase I de recolección de información con el propósito de socializar los resultados del proyecto e invitarlos para que editen y actualicen sus datos obteniendo beneficios incrementalmente. Dichos servicios van desde la disponibilidad de un mapa de sedes, servicios y profesionales hasta la consolidación de información para la gestión de sus redes tecnológicas primarias.

\item Encuestas: Para identificar nuevos requerimientos de servicios que puedan ser desplegados en la plataforma propuesta. Además, estos instrumentos permitirán medir el impacto en la cobertura de los servicios y de participación de las instituciones objeto de la investigación de campo.
\end{itemize}

Las unidades de recolección de información mostradas en el anexo \ref{formulario_preliminar}, se han adaptado de aquellas propuestas por el proyecto europeo HERMES. \footnote{Proyecto de investigación a tres años, financiado por la Comunidad Económica Europea y que cumplió sus objetivos hacia principios del milenio dejando como resultado un conjunto de preguntas básicas que apoyan los procesos de implementación de soluciones médicas apoyadas en las TIC.} El modelo investigación evaluativa que continua en la fase IV del proyecto pretende medir la evolución del nivel de servicios de salud prestados con el apoyo de TIC comparando sucesivamente el modo de operación encontrado entre el año 2000 y 2005 con aquel encontrado entre el año 2008 y 2011 luego que algunas entidades interactúen con el SITEM.

\subsection {Fuentes de Información Primaria}

Para la carga de información inicial en el SITEM se utilizan los resultados del estudio de campo realizado en varias instituciones de carácter público y privado. Dicho resultados se encuentran consignados en sendas tesis en formato digital e impresos disponibles en la biblioteca de la Universidad Distrital y archivo del grupo de investigación:

\begin{itemize}
 \item Hospital Rafael Uribe Uribe.\cite{guarin2003}
 \item Hospital San Pedro Claver.\cite{ardila2001},\cite{rozo2002}
 \item Hospital Simón Bolívar. \cite{acero2002}
 \item Hospital El Tunal. \cite{ruiz2002}\cite{duque2002}
 \item Hospital La Victoria.\cite{barrero2000}
 \item Hospital San José.\cite{gonzalez2002}
\end{itemize}
\chapter{Experiencia de Desarrollo de openSITEM}

El presente capítulo describe la experiencia de desarrollo en el equipo base de openSITEM y junto con la sección de \textit{proceso de desarrollo} se considera que define los lineamientos generales. Como se ha mencionado, openSITEM es centrado en la arquitectura y guiado por casos de uso; y tanto el desarrollador principal como los equipos de trabajo temporales, están comprometidos con la tarea de generar un aplicativo funcional que esté alineado - de manera emergente, con las vistas arquitectónicas planteadas.

Los artefactos se construyen de manera iterativa e incremental, y el ``orden'' en el que aquí se muestra no implica una secuencia de actividades. No en pocas ocasiones abordar la disciplina de \textit{Modelado de Dominio} se realiza a continuación de - o en paralelo a, un taller de requisitos y dicho taller surge de una nueva restricción detectada al momento de despliegue. Esta capacidad de adaptación al cambio es lo que ha permitido que openSITEM vaya realizando la transición desde un aplicativo específico a una suite escalable - potenciada por la arquitectura orientada a servicios que está implementando.

Aclarado lo anterior, en la implementación de cada uno de los módulos de openSITEM se siguen las fases contempladas en el Proceso Unificado y se desarrollan flujos de trabajo en las disciplinas básicas de:
\begin{itemize}
\item Requisitos.
\item Análisis.
\item Diseño.
\item Elaboración.
\item Pruebas.
\item Despliegue.
\end{itemize}

En cada fase se presentan resultados que permiten medir constantemente los avances en el proyecto, así como comprobar los niveles de calidad y la validez del método empleado. A partir de iteraciones continuas por las diferentes disciplinas se refinan constantemente los modelos.

\subsection{Modelo de Requisitos}

El equipo base de openSITEM debe declarar explícitamente el requerimiento (o requerimientos) que da génesis a los elementos que se desean abordar. Dicha declaración debe ser coherente con los objetivos de openSITEM y la filosofía FLOSS, respetando siempre los principios constitucionales, la propiedad intelectual y el respeto al trabajo de los demás. Como puede deducirse, las contribuciones a la rama principal de openSITEM son de carácter burocrático basado en el análisis de los requerimientos de actualizaciones que realicen colaboradores que no pertenezcan el equipo base.

\subsubsection{Gestión del Contexto Problémico}

El contexto problémico de openSITEM está en constante redefinición. La línea base que se presenta a continuación marca el final de la caracterización que dio lugar a la tercera fase del proyecto y se constituye en el punto de partida para posteriores versiones.


\subsection{Línea base del Contexto Problémico}

Durante más de tres décadas se han realizado estudios relacionados con la telesalud, sus características, requerimientos, implicaciones, componentes y normatividad \cite{currie2014} \cite{aparicio2003} \cite{ross2016} \cite{bashshur95} \cite{oas2002} \cite{lewis2013}. Sin embargo, la mayoría de ellos se encuentran disgregados y en idiomas diferentes al español por lo cual su consulta es compleja y no existe un mapa de navegación especializado que guíe al investigador hacia las fuentes de información y este termina reinventando la rueda o, en el mejor de los escenarios, realizando un filtro interminable de información en donde lee, analiza y descarta datos – todo este trabajo tirado a la basura cuando otro investigador entra en escena y por falta de medios de socialización (y representación de conocimiento) desconoce el trabajo del primero, iniciando un nuevo proceso.









Datos básicos de enlaces, grupos de investigación, proyectos e información técnica en telemedicina ha sido consignada en sendas tesis que, aunque completas en el estudio, no ofrecen la pertinencia y eficacia necesaria para el ambiente investigativo. 

La idea de integración de una \textit{comunidad de práctica investigativa} en el área de la telemedicina, perseguida durante varios años por la comunidad académica, ha sido torpedeada constantemente por los intereses comerciales de las empresas que ven en esta tecnología un beneficio meramente económico y no conciben un sistema de información integral abierto de caracter social por lo que promuevan la cultura del “ocultismo” de información propia de los procesos capitales. Estudios abandonados, e inconclusos, junto con la complejidad innecesaria del proceso de determinación del estado del arte, están abocando a los grupos universitarios a competir codo a codo - a pesar de todas sus limitaciones - contra grandes empresas multinacionales interesadas en “sacar del camino” a estos facilitadores de procesos.

Ante esta perspectiva la academia, cuya arma más tenaz es el conocimiento filosófico guiado por el método científico, debe propender por colectivizar eficaz y eficientemente todos sus avances en el campo de la telemedicina brindando alternativas públicas y de alta calidad. Es en esta tarea en la cual las tecnologías de la información entran a jugar un papel importante y más aún el medio de difusión más democrático, o anárquico si se quiere, que haya concebido la humanidad: \textit{Internet}. El Sistema de Información en Telemedicina debe hacer uso de dichas tecnologías y nacer como respuesta a las necesidades de información confiable, pertinente, eficaz y eficiente de la comunidad investigativa así como de mecanismos rápidos de socialización y actualización de resultados.

\subsubsection{Responsabilidades del Sistema}
El Sistema de Información en Telemedicina debe ser una herramienta de apoyo al investigador en el área de desarrollo de proyectos de Telemedicina. Para esto deberá estar, como mínimo, en capacidad de:

\begin{itemize}
\item \textbf{Gestionar la información y conocimiento de diferentes nodos declarados en una red de telemedicina.} se hace especial énfasis en los siguientes componentes ya caracterizados: entidades de salud, servicios médicos - incluyendo medicamentos y enfermedades, grupos de investigación en telemedicina, tecnologías de interconexión y operadores de telecomunicaciones.

\item \textbf{Administrar en forma segura los diferentes actores del sistema} para brindarles una experiencia enriquecida al interactuar con el Portal. La administración debe ser transparente y con miras a la estructuración avanzada de contenidos de acuerdo al perfil del usuario.

\item \textbf{Proveer instrumentos} para implementar metodologías de captura de información.

\item \textbf{Presentar la información} requerida por los actores enriqueciéndola con contenidos relacionados y haciendo uso de una GUI multimedia basada en tecnologías web.

\item \textbf{Integrar aplicaciones de software libre} que brinden funcionalidad de apoyo a las comunidades de práctica que se generen y fomente las tareas de capturar, extraer, organizar, analizar, encontrar, sintetizar, distribuir y compartir información y conocimiento. 

\item \textbf{Funcionar sobre una plataforma tecnológica basada en aplicaciones de software libre.}

\end{itemize}


\subsubsection{Definición de Alcances}

El openSITEM es principalmente \textit{un concepto}, su estado actual es una representación del potencial real del sistema que debe ser socializado y entregado a la comunidad. La base de desarrollo principal es el grupo GITEM y será responsable de la versión oficial del producto. Sin embargo, dada la dinámica en el mundo del software libre, el grupo GITEM no limitará el trabajo independiente que sobre su desarrollo realice cualquier persona o grupo de personas. En este sentido la funcionalidad original del sistema podrá ser modificada pero no avalada directamente por el grupo.\footnote{Salvo en casos en que no se trasgredan directamente los objetivos primarios del desarrollo. En tales casos las contribuciones serán asociadas al hilo oficial de desarrollo.}. 

El openSITEM ha sido creado con el fin de apoyar a los grupos de trabajo que realizan labores en el área de proyección de sistemas de telemedicina. La información que en él se encuentra debe ser ingresada por personas autorizadas para asegurar en un alto grado la veracidad e idoneidad de la misma. Sin embargo no se puede garantizar, y no se garantiza, la exactitud, disponibilidad, integridad y oportunidad de dicha información: \textit{la información contenida en el sitem no es una fuente oficial de datos}. El uso de la misma es responsabilidad de quien lo realiza. La información que se encuentre en el openSITEM no ha sido necesariamente revisada por expertos profesionales. Todos los contenidos que se ingresen al openSITEM deben ser de licencia pública o de libre uso; los contenidos que no cumplan estos criterios serán eliminados.

\subsubsection{Definición de actores}

El openSITEM por su complejidad debe gestionar varios tipos de usuarios, cada uno con características específicas. Este tratamiento especial tiene que ver con el mantenimiento de la integridad de la información, la cual solo puede ser gestionada por un grupo selecto de usuarios. 

Algunos actores esperados en el openSITEM son:

\begin{itemize}
\item Administrador.
\item Consultor.
\item Especialista Médico.
\item Profesional TIC.
\item Usuario General.
\end{itemize}

Cada uno de ellos con las características mostradas en el anexo \ref{modelo_requisitos}.

\subsubsection{Casos de uso}

Los requisitos funcionales del openSITEM son declarados en diferente nivel de detalle por medio de una especificación de casos de uso, diagramas de comportamiento y de interacción. Como mínimo, al caracterizar un flujo de eventos por medio de un caso de uso se espera tener información acerca de:

\begin{itemize}
\item Nombre del Caso de Uso
\item Objetivo que se logra al ejecutarse el caso de uso.
\item Código que lo identifique unívocamente dentro del banco de artefactos.
\item Actores que intervienen al desarrollarse el caso de uso.
\item Casos de uso con los que está relacionado.
\item Precondiciones.  El estado del sistema que debe asegurarse antes de que el caso de uso inicie. Debido a que es responsabilidad del sistema no se verifica en el caso de uso.
\item Postcondiciones. Las características y estado del sistema una vez se haya terminado el caso de uso.
\item Flujo de Tareas. Flujo principal y alternativos de las tareas que se suceden al ejecutarse el caso de uso. Siempre se debe propender por mantener claridad en el modelo por lo que se recomienda utilizar diferentes artefactos para los flujos alternativos cuando esto lo amerite.
\end{itemize}

La tabla \ref{casouso}, muestra el caso de uso de diseño correspondiente al flujo principal del registro de un usuario en el sistema.

\begin{table}
\begin{center}
\begin{tabular}{|l|p{10cm}|}
\hline
\textbf{Caso de Uso}&\\
\hline
Nombre & Registrarse en el openSITEM\\
\hline
Objetivo & El actor logra crear una cuenta en el openSITEM con un rol específico para poder trabajar en un subsistema dado.\\
\hline
Código Interno & UC-GENERAL-001 \\
\hline
Actores & Usuario General\\
\hline
Precondiciones & Ninguna.\\
\hline
Flujo Básico & 1. El usuario general selecciona la opción de nuevo usuario desde la página principal del openSITEM.\\
& 2. El openSITEM muestra un formulario con los campos:\\
& Nombres\\
& Apellidos\\
& Correo Electrónico\\
& Teléfono\\
& Nombre de Usuario\\
& Clave\\
& Reescriba la clave\\
& Acceso Requerido\\
& 3. El usuario diligencia uno a uno los campos requeridos y opcionales.\\
& 4. El usuario envía los datos al openSITEM.\\
& 5. El openSITEM verifica que los datos tengan los formatos esperados.\\
& 6. El openSITEM ingresa el registro a la base de datos colocando el campo de estado en 1 - registrado sin autorización.\\
& 7. El openSITEM redirecciona a la página de registro exitoso.\\
& 8. El usuario acepta el mensaje.\\
\hline
Postcondiciones & Se agregó un registro en la base de datos con el campo de estado en 1.\\
\hline
Casos de uso relacionados&Seleccionar Rol en el openSITEM\\
\hline
\end{tabular}
\caption{Caso de Uso Registrarse en el openSITEM}
\label{casouso} 
\end{center}
\end{table}

El modelo de casos de uso consta en la actualidad con 180 elementos principales definidos\footnote{No se incluyen casos de uso implementados en aplicaciones conexas.} y más de 230 flujos alternativos - los más relevantes incluidos en el anexo \ref{modelo_requisitos}. Con esto se concreta los requerimientos de autenticación, gestión de información - incluyendo generaciones de informes, trabajo en grupo, consultoría y georeferenciación. Muchos de ellos aún no se han desarrollado en su totalidad por lo que se preve que este modelo sufra modificaciones.

\subsection{Modelo de Análisis y Diseño}

El modelo de requisitos es insumo para modelar los componentes del sistema con los modelos de análisis - en donde se especifica en más detalle cada caso de uso; y el de diseño, en donde se modela el comportamiento y la estructura de los diferentes componentes que implementarán la funcionalidad. Deliberadamente se unen estas dos disciplinas en un solo modelo para evitar un excesiva concentración de documentación en el análisis.

El equipo de desarrollo se ha apoyado en diferentes diagramas de estructura, de comportamiento e interacción. El anexo \ref{modelo_analisis} corresponde al modelo de análisis y diseño conteniendo los elementos más interesantes para diferentes componentes del sistema. En este punto cobra importancia la aplicación de ciertos patrones GRASP \cite{larman2003} especialmente se presta atención a mantener los principios de:

\begin{itemize}
\item Alta Cohesión
\item Bajo Acoplamiento
\end{itemize}

Asignando responsabilidades teniendo en cuenta:
\begin{itemize}
\item Experto en Información.
\item Controlador
\end{itemize}

En la figura \ref{secuencia}, se muestra el diagrama de interacción correspondiente a la realización del caso de uso registrarse en el sistema.

\begin{figure}
 \centering
 \includegraphics[width=156mm, height=156mm]{secuencia.png}
 \caption{Realización del Caso de Uso registrarse en el Sistema}
 \label{secuencia}
\end{figure}


\subsection{Modelo de Implementación}

El conjunto de diagramas del Anexo \ref{modelo_analisis} brinda la información fundamental para el modelo de implementación. En el openSITEM se agrupan los diferentes componentes en la jerarquía de carpetas mostrada en la figura \ref{carpetas_sitem}.

Cada una de las carpetas contiene los ficheros de código fuente del producto:

\begin{itemize}
 \item \textbf{Clases:} Contiene los archivos en PHP que implementan las clases. 
\item \textbf{Funciones:} Grupo de funciones en JavaScript para la validación de información en el nodo de usuario. En la actualidad la fase IV contempla complementar esta aproximación con la utilización de AJAX.
\item \textbf{Configuración:} Alberga el archivo \textit{config.inc.php} que guarda las variables de ingreso a la base de datos. Dichas variables se encuentran codificadas de acuerdo al algoritmo que se seleccione (o implemente) desde la clase \textit{codificar}.
\item \textbf{Bloques:} Agrupa el código fuente de cada bloque desarrollado en el openSITEM. Un bloque se define como una unidad de funcionalidad independiente que puede utilizarse en cualquier página.
\item \textbf{Estilo:} Información acerca de los parámetros generales de estilo - tamaño de fuente, color de bordes, fondos, colores de letras, etc; para diferentes componentes del openSITEM. La modificación o inclusión de parámetros afectará la interfaz global del sistema. Actualmente los estilos en el openSITEM se basan en hojas de estilo CSS.
\item \textbf{Gráficos:} Todos los archivos gráficos usados en el proyecto.
\item \textbf{Documentos:} Carpeta inicialmente vacía que se utiliza para guardar los archivos que los usuarios caragen a través del protocolo HTTP. Por seguridad se recomienda que esta carpeta se encuentre fuera del directorio en donde se encuentra instalada la aplicación.
\item \textbf{Instalar:} Contiene el instalador del producto. Esta carpeta debería ser retirada una vez el sitio se encuentre en producción.
\item \textbf{Desarrollo:} Con varios scripts que facilitan la tarea de desarrollo y adaptación de bloques en el sistema. Estos scripts se han construidos pensando en plataformas de desarrollo y prueba por lo que se supone no debe encontrarse en plataformas de producción.
\end{itemize}

En todo caso, durante el proceso de instalación se puede - y recomienda; asociar nuevos nombres a las carpetas por lo que en teoría ningún desarrollo basado en el openSITEM que esté en etapa de producción debería tener nombres de carpetas conocidos.

\begin{figure}
 \centering
 \includegraphics[width=156mm, height=156mm]{carpetas.png}
 \caption{Carpetas que se distribuyen con el openSITEM}
 \label{carpetas_sitem}
\end{figure}

\subsubsection{Acerca de la Seguridad}
El uso lenguajes de scripts en el lado del servidor puede tener algunos problemas de seguridad cuando son accedidos mediante instrucciones tipo GET - inserción SQL, data binding, etc: por esta razón cuando se usan variables usando este método todas se cifran en una cadena combinada de 256 bits, figura \ref{desenlace}. Todas las peticiones desde el cliente se realizan usando métodos POST sin perjuicio de la funcionalidad. Se prefiere sobre cualquier otro criterio el manejo de cookies pero en el caso de que estas estén deshabilitadas toda la información se validará empleando la dirección IP inicial de conexión.

En cuanto a la integridad de los datos se tiene un modelo de comparación de contenidos que se activa cada periodo de tiempo, el cual es programable; proponiéndose mantener una copia de respaldo verificada y avalada por el administrador. En caso de corrupción o pérdida de datos se mantiene una lista completa de los usuarios del sistema de hasta 1'000.000 de sesiones de tipo desplazamiento en donde el usuario más antiguo es descartado para la inclusión del nuevo cuando el tamaño asignado es completado, asegurándose así el manejo eficiente de disco.

Por ningún motivo se permite el acceso a sitios restringidos a usuarios que no hayan sido plenamente identificados en el sistema. En sitios críticos se hace una revisión de los datos de acceso guardados en cookies o se constata los datos de inicio de sesión. Se han evitado al máximo los usos de carácter comodín y todos los accesos a la base de datos son validados en su sintaxis. El openSITEM actualmente propone el uso de protocolos seguros tales como SSL o SHTML.

Vale la pena destacar el uso de metodologías de autenticación de usuario basado en sesiones y codificación de datos que permiten ofrecer un contenido personalizado de acuerdo al perfil de cada uno de los clientes del sistema.

\begin{figure}
 \centering
 \includegraphics[width=156mm, height=156mm]{desenlace.png}
 \caption{URL encriptada. Con la herramienta \textit{Desenlace} el desarrollador puede descifrar los datos}
 \label{desenlace}
\end{figure}


\subsection{Modelo de Datos}

Dentro del proceso de desarrollo del openSITEM el modelado y elaboración del sistema de bases de datos es una de las partes fundamentales de la propuesta. Se ha venido estructurando el modelo de acuerdo a las necesidades de cada módulo en particular para garantizar la independencia entre ellos en cada una de las capas, incluyendo la de persistencia.

Dependiendo de las caracterísitcas de cada subsistemas se implementan o no políticas transaccionales. El modelo de seguridad en los datos hereda todos los elementos del servidor tales como bloqueos de puertos, ocultación de ventanas, manejo de sockets, etc; además, un esquema lógico de validación por conexiones persistentes complementa estas características.

El anexo \ref{modelo_datos} contiene los diagramas de clases que describen la arquitectura de datos del sistema y cada uno de sus subsistemas asociados. 

En la actualidad el openSITEM acepta bases de datos PostgreSQL, MySQL y ORACLE. La capa de persistencia del hilo principal se despliega sobre un servidor MySQL 5.0.27-standard.


\subsection{Interfaz gráfica.}

De acuerdo a los diagramas conceptuales del portal GITEM y del openSITEM. Se han utilizado para la creación de las páginas los conceptos de diseño web enumerado por (Maldonado,2001), intentando evitar al máximo las páginas sobrecargadas de información. La navegación es guiada mediante enlaces dentro de las mismas páginas las cuales están agrupadas temáticamente logrando una coherencia en el contenido. 

Los gráficos han sido optimizados y su inclusión es necesaria para dar ayuda visual al contenido basado en texto. Teniendo en cuenta que el openSITEM esta diseñado para interactuar permanentemente y por periodos prolongados de tiempo con el cliente, se han evitado deliberadamente la utilización  de componentes dinámicos tales como películas en flash, gif animados o menús desplegables. No obstante, el diseño no pierde atractivo ya que su implementación se fundamenta en estudios técnicos de comportamiento humano cuando navegan por Internet.

Para reducir el tiempo de acceso al portal, sobre todo cuando se trabaja con conexiones lentas, se da la posibilidad en algunos subsistemas de descargar en formato PDF todo el contenido del grupo de páginas en donde se este ubicado.

\subsubsection{Arquitectura de la página}

Independiente del subsistema que nos encontremos las páginas siempre están compuestas por cinco secciones denominadas genéricamente con las letras A, B, C, D y E. En ellas se distribuyen los diferentes bloque que conforman la página en una arquitectura Top - Down. Las páginas que no tienen bloques en todas las secciones colapsan aquellas que no se utilizan para dar una impresión visual consistente. Las figuras \ref{secciones} y \ref{seccion_colapsada} muestran gráficamente el manejo de las secciones en cada página.

\begin{figure}
 \centering
 \includegraphics[width=156mm, height=156mm]{secciones.png}
 \caption{Arquitectura de una página en el openSITEM}
 \label{secciones}
\end{figure}

\begin{figure}
 \centering
 \includegraphics[width=156mm, height=156mm]{seccion_colapsada.png}
 \caption{Página del openSITEM en donde la Sección D se ha colapsado.}
 \label{seccion_colapsada}
\end{figure}

\subsection{Entregables del proyecto}

Los siguientes artefactos – documentos, son generados y utilizados por el proyecto. Una copia de cada uno de ellos puede ser descargada desde Internet. \footnote{En una carpeta especialmente diseñada para esto: http://gitem.udistrital.edu.co/sitem/desarrollo/index.php.} En el anexo \ref{entregables} se tiene extractos importantes de algunos de ellos.

\begin{itemize}
\item \textbf{Plan General de Trabajo}
\item \textbf{Modelo de Casos de Uso}. El cual especifica los requerimientos que debe cumplir el módulo de software y que en últimas constituye los contratos que éste, el módulo, tiene con actores externos. Este artefacto estará hecho en su totalidad usando el Lenguaje de Modelado Unificado. 

Dentro de este modelo se tienen dos vistas claras: la del negocio y la del sistema. El modelo de Casos de Uso del Negocio ilustra el ámbito del negocio que esta siendo modelado. El diagrama contiene actores del negocio y los servicio o funciones que ellos requieren del negocio. El modelo de casos de uso del sistema representa el ámbito de una aplicación. De esto se tiene que un solo modelo de casos de uso del negocio puede tener muchos Modelos de casos de uso asociados, donde cada modelo de casos de uso representa una única aplicación.

\item \textbf{Modelo de Objetos.} El cual describe la forma en que cada requerimiento – o contrato,  es cumplido. Establece las entidades internas, la información que intercambian y los flujos de trabajo que logran el cumplimiento de los requisitos. Los grafos correspondientes podrán incluir diagramas estáticos o interactivos expresados en UML.

\item \textbf{Glosario.} Que es el único artefacto válido de consulta para la terminología usada en el desarrollo del openSITEM. Ver el anexo \ref{glosario}.

\item \textbf{Visión.} Este documento define la visión del openSITEM. Es de todos el que marca las pautas conceptuales. 

\item \textbf{Especificaciones Adicionales.} Este documento capturará todos los requisitos que no han sido incluidos como parte de los casos de uso y se refieren requisitos no-funcionales globales. Dichos requisitos incluyen: requisitos legales o normas, aplicación de estándares, requisitos de calidad del producto, tales como: confiabilidad, desempeño, etc., u otros requisitos de ambiente, tales como: sistema operativo, requisitos de compatibilidad, etc. 

\item \textbf{Prototipos de Interfaces de Usuario.} Se trata de prototipos que permiten al usuario hacerse una idea más o menos precisa de las interfaces que proveerá el sistema y así, conseguir retroalimentación de su parte respecto a los requisitos del sistema. Estos prototipos se realizarán como: dibujos a mano en papel, dibujos con alguna herramienta gráfica o prototipos ejecutables interactivos, siguiendo ese orden de acuerdo al avance del proyecto. Sólo los de este último tipo serán entregados al final de la fase de Elaboración, los otros serán desechados. Asimismo, este artefacto, será desechado en la fase de Construcción en la medida que el resultado de las iteraciones vayan desarrollando el producto final. 

\item \textbf{Modelo de Análisis y Diseño.} Este modelo establece la realización de los casos de uso en clases y pasando desde una representación en términos de análisis (sin incluir aspectos de implementación) hacia una de diseño (incluyendo una orientación hacia el entorno de implementación), de acuerdo al avance del proyecto.  Consultar el anexo \ref{modelo_analisis}.

\item \textbf{Modelo de Datos.} Previendo que la persistencia de la información del sistema será soportada por una base de datos relacional, este modelo describe la representación lógica de los datos persistentes, de acuerdo con el enfoque para modelado relacional de datos. Para expresar este modelo se utiliza un Diagrama de Clases (donde se utiliza un profile UML para Modelado de Datos, para conseguir la representación de tablas, claves, etc.). El anexo \ref{modelo_datos} contiene los artefactos más importantes del modelo de datos. 

\item \textbf{Modelo de Implementación.} Este modelo es una colección de componentes y los subsistemas que los contienen. Estos componentes incluyen: ficheros ejecutables, ficheros de código fuente, y todo otro tipo de ficheros necesarios para la implantación y despliegue del sistema.

\item \textbf{Modelo de Despliegue.} Este modelo muestra el despliegue la configuración de tipos de nodos del sistema, en los cuales se hará el despliegue de los componentes. Ver anexo \ref{modelo_despliegue} 

\item \textbf{Casos de Prueba.} Cada prueba es especificada mediante un documento que establece las condiciones de ejecución, las entradas de la prueba, y los resultados esperados. Estos casos de prueba son aplicados como pruebas de regresión en cada iteración. Cada caso de prueba llevará asociado un procedimiento de prueba con las instrucciones para realizar la prueba, y dependiendo del tipo de prueba dicho procedimiento podrá ser automatizable mediante un script de prueba. 

\item \textbf{Solicitud de Cambio.} Los cambios propuestos para los artefactos se formalizan mediante este documento. Mediante este documento se hace un seguimiento de los defectos detectados, solicitud de mejoras o cambios en los requisitos del producto. Así se provee un registro de decisiones de cambios, de su evaluación e impacto, y se asegura que éstos sean conocidos por el equipo de desarrollo. Los cambios se establecen respecto de la última baseline (el estado del conjunto de los artefactos en un momento determinado del proyecto) establecida. En nuestro caso al final de cada iteración se establecerá una baseline. 

\item \textbf{Plan de Iteración.} El conjunto de actividades y tareas se orden temporalmente y se le asignan recursos a corto plazo. Se realiza para cada iteración y en todas las fases. 

\item \textbf{Lista de Riesgos.} Este documento incluye una lista de los riesgos conocidos y vigentes en el proyecto, ordenados en orden decreciente de importancia y con acciones específicas de contingencia o para su mitigación. El anexo \ref{riesgos} contiene la declaración de los riesgos más importantes detectados en el proyecto así como estrategias para minimizarlos.

\end{itemize}
\begin{thebibliography}{}

\bibitem[Acero y Ariza,2002] {acero2002} Acero, D. Ariza, M.\textit{Requerimientos tecnicos y financieros para la implementacion de una red piloto de Telemedicina en el Hospital Simón Bolivar}.Grupo de Investigación en Telemedicina,  Universidad Distrital Francisco José de Caldas.2002.

\bibitem[Alhir,2003] {alhir2003} \textit{Understanding the Unified Process.} Methods and Tools, Martinig Associates. 2002.

\bibitem[Aparicio-Ramirez,2003] {aparicio2003} Aparicio,L. y Ramírez, J.\textit{Arquitectura de Red de Telemedicina}, Centro de Investigaciones y Desarrollo Científico, Universidad Distrital F.J.C, 2003.

\bibitem[Aparicio,2000] {aparicio2000} Aparicio,L.\textit{Propuesta de Estudio Red de Telemedicina Bogota}, Grupo GITEM, Universidad Distrital F.J.C, 2000.

\bibitem[Ardila,2001] {ardila2001} Ardila,J y Ardila, M.\textit{Evaluación y diagnóstico de los servicios básicos y especializados al servicio de la salud de la clínica San Pedro Claver para el desarrollo de la red de Telemedicina de Bogotá}.Grupo de Investigación en Telemedicina,  Universidad Distrital Francisco José de Caldas.2001.


\bibitem[Bashshur,1977] {bashshur77} Bashshur, R., Lovett J. \textit{Assessment of telemedicine: results of the initial experience.} Aviation, Space and Environmental Medicine 1977.

\bibitem[Bashshur,1995] {bashshur95} Bashshur, R. \textit{On the Definition and Evaluation of Telemedicine. Telemedicine Journal.} Volume 1, Number 1, Mary Ann Liebert, Inc., Publishers. 1995.

\bibitem[Beck,1999] {beck1999} Beck, K. \textit{ Extreme Programming Explained: Embrace Change }. Addison-Wesley. 1999.

\bibitem[Benafia,2016] {benafia2016} Benafia, A., Mazouzi, S. y Maanru, R. \textit{From Linguistic to Conceptual: A Framework Based on a Pipeline for Building Ontologies from Texts}. Journal of Advanced Computational Intelligence and Intelligent Informatics. 2016.

\bibitem[Bergeron,2003] {bergeron2003} Bergeron, B. \textit{Essentials of knowledge management.} Jhon Wiley \& Sons, New Jersey, 2003.

\bibitem[Bos,2004] {bos2004} Bos,B.\textit{A proposal for Webapps}. W3C Consortium,2004.

\bibitem[Boggs,2002] {boggs2002} Boggs,W. Boogs, M.\textit{UML with Rational Rose 2002}. SyBex,2002.

\bibitem[BCE,2003] {bce} Britannica Concise Encyclopedia.\textit{Anarchism.}. Obtenido en julio 18, 2003, desde:    http://search.eb.com/ebc/article?eu=380585.

\bibitem[Brugge, Dutoit, 2000] {objectoriented} Brugge, B. y Dutoit, A.~H.\textit{Object-Oriented Software Engineering}. Prentice Hall, 2000.

\bibitem[McCarty, 2006] {carty} Carty, J. \textit{Dynamics of Software Development}.Microsoft Press, 2006.

\bibitem[Cockburn, 1999] {cockburn1999} Cockburn, A. \textit{The Impact of Object-Orientation on Application Development.} IBM Systems Journal 38. Páginas 308-332, 1999.

\bibitem[Ley 23,1982] {congreso23} Congreso de la República. \textit{Ley 23 de 1982: Sobre Derechos de Autor}. República de Colombia, 1982.

\bibitem[Dec 1360,1989] {congreso1360} Congreso de la República. \textit{Decreto 1360 de 1989: Inscripción del soporte lógico (software) en el Registro Nacional del Derecho de Autor}. República de Colombia, 1989.

\bibitem[Ley 44,1993] {congreso44} Congreso de la República. \textit{Ley 44 de 1993: Modifica y adiciona la Ley 23 de 1982 y modifica la Ley 29 de 1944}. República de Colombia, 1993.

\bibitem[Ley 565,2000] {congreso565} Congreso de la República. \textit{Ley 565 de 2000: adopción del Tratado de la OMPI sobre Derechos de Autor}. República de Colombia, 2000.

\bibitem[CRT,QoS,2006] {crtcondiciones} Comisión de Regulación de Telecomunicaciones. \textit{Condiciones de Calidad en Servicios de Telecomunicaciones.} República de Colombia, 2006. 

\bibitem[CRT,Indicadores,2006] {crtindicadores} Comisión de Regulación de Telecomunicaciones. \textit{Indicadores de Calidad en Servicios de Telecomunicaciones.} República de Colombia, 2006. 

\bibitem[CRT,QoS,2007] {crtqos} Comisión de Regulación de Telecomunicaciones.\textit{Proyecto de Resolución para Indicadores de Calidad de Servicio.} República de Colombia, 2007.

\bibitem[Corte1319,2001] {sentencia1319} Corte Constitucional de Colombia.\textit{Sentencia T-1319 de 2001.} República de Colombia, 2001

\bibitem[Craig, 2005] {craig2005} Craig J, Patterson V. \textit{Introduction to the practice of telemedicine.} Journal of Telemedicine and Telecare. 2005. Págs 3-9.

\bibitem[Davenport,1998] {davenport1998} Davenport, T. y Prusak, L.\textit{Working Knowledge: How Organizations Manage What They Know}, Harvard Business School Press, Boston.1998

\bibitem[Davies,2011] {davies2011}. Davies, M. Knowledge – Explicit, implicit and tacit: Philosophical aspects. , International Encyclopedia of Social and Behavioral Sciences, Elsevier. 2011.

\bibitem[Dueck,2001] {dueck2001} Dueck,G.\textit{Views of knowledge are human views.}IBM Systems Journal, Volumen 40, Número 4, 2001.

\bibitem[Duque \textit{et al},2002] {duque2002} Duque,J. García J, Caicedo, D.\textit{Estudio sobre los requerimientos técnicos y financieros para implementar los servicios telemédicos en el Hospital El Tunal,  para el proyecto telemedicina Bogotá 2000.}.Grupo de Investigación en Telemedicina,  Universidad Distrital Francisco José de Caldas.2002.

\bibitem[UN, 2005] {egovernment} Department of Economic and Social Affairs.\textit{UN Global E-government Readiness Report 2005 From e-government to e-inclusion}. Naciones Unidas, New York, 2005.

\bibitem[Firestone,2001] {firestone2001} Firestone, J.\textit{Key Issues in Knowledge Management. Knowledge and Innovation.} Knowledge Management Consortium International. Volumen 1. 2001.

\bibitem[Gang,2007] {gang2007} Gang, L. y Yi, L. \textit{A Relational Model of Knowledge Share, Knowledge Acquisition and Product Innovation}. Universidad Xi'an Jiaotong. China. 2007.

\bibitem[Girard,2015] {girard2015} Girard, J. \textit{Defining knowledge management: Toward an applied compendium}. Online Journal of Applied Knowledge Management. International Institute for Applied Knowledge Management. 2015.

\bibitem[González y Torres,2002] {gonzalez2002} González, O. Torres, J.\textit{Evaluación y Diagnóstico de los Servicios Básicos y Especializados al Servicio de la Salud del Hospital de San José para el Desarrollo de la Red de Telemedicina de Bogotá.} Grupo de Investigación en Telemedicina, Universidad Distrital Francisco José de Caldas.2002.

\bibitem[Guarin \textit{et al},2003] {guarin2003} Guarin, S. Garcia, M. Torres, L.\textit{Diagnóstico de las Redes Eléctrica, Telefónica y de Datos del Hospital Rafael Uribe Uribe E.S.E.} Grupo de Investigación en Telemedicina,  Universidad Distrital Francisco José de Caldas.2003.

\bibitem[Hurtado,2000] {hurtado2000} Hurtado,J.\textit{Metodología de la Investigación Holística.} Sypal, Caracas, Venezuela, 2000.

\bibitem[Hibbard,1997] {hibbard1997} Hibbard, J. \textit{Knowledge management—knowing what we know.} Information Week, Edición Octubre de 1997.

\bibitem[IEEE, 1990] {softwareengineering} IEEE Institute.\textit{IEEE Standard Glossary of Software Engineering Terminology - IEEE std 610.12-1990}. IEEE, 1990.

\bibitem[Koch, 2005] {koch} Koch, S. \textit{Free Open Source Software Development}. Idea Group, 2005.

\bibitem[Liebowitz,1998] {liebowitz1998} Liebowitz, J. y Beckman, T. \textit{Knowledge Organizations: What Every Manager Should Know}. Boca Raton, St. Luci press, 1998.

\bibitem[ITU-T, 2004] {ITU2004} ITU-T. \textit{Manual Calidad de servicio y calidad de funcionamiento de la red}. Unión Internacional de Telecomunicaciones, 2004.

\bibitem[ITU-OMS, 2012] {ituoms2012} ITU - OMS. \textit{National eHealth Strategy Toolkit}. Unión Internacional de Telecomunicaciones. 2012 


\bibitem[G1000, 2001] {ITUG1000} ITU-T. \textit{Recomendación G.1000 - Calidad de servicio en las comunicaciones: Marco y definiciones}. Unión Internacional de Telecomunicaciones, 2001.

\bibitem[G1010, 2001] {ITUG1010} ITU-T. \textit{Recomendación G.1010 - Categorías de calidad de servicio para los usuarios de extremo de servicios multimedios}. Unión Internacional de Telecomunicaciones, 2001.

\bibitem[Jacobson,2000] {jacobson2000} Jacobson,I. Booch,G y Rumbaugh,J. \textit{El Proceso Unificado de Desarrollo de Software}. Addison-Wesley, Madrid, 2000.

\bibitem[Jacobson,2005] {jacobson2005} Jacobson,I. Booch,G y Rumbaugh,J. \textit{The Unified Modeling Language Reference Manual}, segunda edición. Addison-Wesley, Boston, 2005.

\bibitem[Jackson,2005] {jackson2005} Jackson,D.\textit{The W3C Workshop on Web Applications and Compound Documents}. W3C Consortium,2005.

\bibitem[Larman,2004] {larman2004} Larman,C. \textit{Agile and iterative development: a manager ’s guide.} Addison Wesley, 2004.

\bibitem[Larman,2003] {larman2003} Larman, C. \textit{UML y PATRONES. Una introducción al análisis y diseño orientado a objetos y al Proceso Unificado}. Pearson Educación S.A. Madrid, 2003.

\bibitem[Martínez,1998] {martinez1998} Martínez, R. Martín, F. \textit{LANDSCAPE: A Knowledge-Based System for Visual Landscape Assessment.}. IEA/AIE, Volumen 2. Springer, 1998. Páginas 849-856.

\bibitem[CONPES3072,2000] {mincomunicaciones3072} Ministerio de Comunicaciones.\textit{Documento CONPES 3072 - Agenda de Conectividad}. República de Colombia, 2000.

\bibitem[MINSALUD-4678,2015] {minsalud4678} Ministerio de Salud.\textit{Resolución Número 4678 de 2015}. República de Colombia, 2015.

\bibitem[MINSALUD-1448,2006] {minsalud1448} Ministerio de Salud.\textit{Resolución Número 1448 de 2006}. República de Colombia, 2006.

\bibitem[Barrero,2000] {barrero2000} Muñoz, F. Barrero, J.\textit{Evaluación y Diagnóstico de los Servicios Básicos y Especializados al Servicio de la Salud del Hospital La Victoria para el Desarrollo de la Red de Telemedicina de Bogotá}. Grupo de Investigación en Telemedicina,  Universidad Distrital Francisco José de Caldas.2000.

\bibitem[Nokata,1994] {nokata1994} Nonaka, I.\textit{A dynamic theory of organizational knowledge creation.} Organization Science,5,1994. pp. 14-37.

\bibitem[Nokata,1995] {nokata1995} Nonaka,I. y Takeuchi, H.\textit{The Knowledge Creating Company: How Japanese Companies Create the Dynamics of Innovation.} Oxford University Press,1995.

\bibitem[OMG, 2007] {omg2007} OMG.\textit{Unified Modeling Language: Specification. Versión 2.1.1}. Object Management Group, 2007.

\bibitem[OMG, 2007a] {omg2007a} OMG.\textit{Unified Modeling Language: Superstructure. Versión 2.1.1}. Object Management Group, 2007.

\bibitem[OMS, 2010] {oms2010} Organización Mundial de la salud. \textit{Telemedicine: opportunities and developments in Member States: report on the second global survey on eHealth 2009}. 2010

\bibitem[OMS, 2016] {oms2016} Organización Mundial de la salud. \textit{Global difusion of eHealth: making universal health coverage achievable. Report of the third global survey on eHealth}. 2016

\bibitem[OPS, 2011]{ops2011}. Organización Panamericana de la Salud. \textit{Estrategia y Plan de Acción sobre eSalud}. 51 Consejo Directivo, 2011

\bibitem[Pressman, 2006] {pressman} Pressman, R.~J.\textit{Ingeniería del Software}. Sexta Edición. Mc Graw 
Hill, México, 2006.

\bibitem[Raymond, 1996] {raymond} Raymond,E.\textit{The Cathedral and the Bazaar}, Revision 1.57, 2000.

\bibitem[AIM,1993] {aim} \textit{Research and technology development on telematics systems in health care: AIM 1993; Annual Technical Report on RTD: Health Care.} Comisión Europea: Dirección General XIII, 1993.

\bibitem[Rozo \textit{et al},2002] {rozo2002} Rozo, O. Valencia, S. Barahona, F.\textit{Estudio diagnóstico de las condiciones técnicas y financieras en instrumentos y equipos médicos  y de servicios de la clínica San Pedro Claver para la implentación de los servicios telemédicos.} Grupo de Investigación en Telemedicina,  Universidad Distrital Francisco José de Caldas.2003.

\bibitem[Ruiz y Niño,2002] {ruiz2002} Ruiz, M. Niño, D.\textit{Evaluacion y Diagnostico de los Servicios Basicos y Especializados al Servicio de la Salud del Hospital El Tunal para el Desarrollo de la Red de Telemedicina de Bogotá} Grupo de Investigación en Telemedicina,  Universidad Distrital Francisco José de Caldas.2002.

\bibitem[Salazar,2002] {oas2002} Salazár,J. y Kopec, A.\textit{Aplicaciones de Telecomunicaciones en Salud en la Subregion Andina - Telemedicina}, Organismo Andino de Salud, OPS. 2002.

\bibitem[Sarmento,2005] {sarmento2005} Sarmento,A.\textit{Issues of human computer interaction}.IRM Press, Londres, 2005. 

\bibitem[Sowa,1984] {sowa1984} Sowa, J. \textit{Conceptual Structures: Information Processing in Mind and Machine.} Addison-Wesley,1984.

\bibitem[stallman,2002] {stallman2002} Stallman,R.\textit{Free Software, Free Society:Selected Essays of Richard M. Stallman}. GNU Press, Bosotón, 2002.

\bibitem[Tatnall, 2003] {tatnall2005} Tatnall,A. \textit{Web portals:The New Gateways to Internet Information and Services.} Idea Group, Londres, 2005.

\bibitem[BDT,1999] {itu} \textit{Telemedicine And Developing Countries - Lessons Learned}. Document 2/116-E. ITU-D STUDY GROUPS. Question 14/2: Fostering the application of telecommunication in health care.  Identifying and documenting success factors for implementing telemedicine. 1999.

\bibitem[Bangemann,1994] {bangeman} \textit{The Bangemann Report: Europe and the global Information Society. Recomendaciones al Consejo Europeo.} Bruselas, 1994. Disponible en http://www.cordis.lu.

\bibitem[Turban,1992] {turban1992} Turban, E.\textit{Decision Support and Expert Systems - Management Support Systems.} Collier Macmillan, Sydney,1992.

\bibitem[Wiig,1993] {wiig1993} Wiig,K.\textit{Knowledge Management Foundations}.Schema Press, Arlington, 1993. 

\bibitem[Wielinga \textit{et al}, 1992] {wielinga1992} Wielinga, B. \textit{et al}. \textit{KADS: A Modelling Approach to Knowledge Engineering.}Knowledge Acquisition Journal, 4(1). Páginas 5-53.

\bibitem[11] {yellowlees} Yellowlees PM.\textit{Successfully developing a telemedicine system.} Journal of Telemedicine and Telecare, 2005. Págs:331-335.

\bibitem[Zabala, 2000] {Zavala2000} Zavala R.\textit{Diseño de un Sistema de Información Geográfica sobre internet.} Tesis de Maestría en Ciencias de la Computación. Universidad Autónoma Metropolitana-Azcapotzalco. México, D.F. 2000.

\end{thebibliography}

\chapter*{Glosario}
\label{glosario}

ADSL - (Asymetric Digital Suscriber Line). Línea digital asimétrica del suscriptor. Tecnología que permite la transmisión de información digital a altas velocidades de descarga utilizando medios de transmisión convencionales.

ALGORITMO - Conjunto de reglas o procedimientos que representa la solución a un problema. Un programa de software es la transcripción de un algoritmo.

ANCHO DE BANDA - Cantidad de datos que pueden transmitirse en determinado periodo de tiempo por un canal de transmisión.

APACHE - Servidor Web de distribución libre,  poderoso y flexible. Implementa los últimos protocolos HTTP. Altamente configurable y extensible con módulos de terceros. Fue desarrollado en 1955 y ha llegado a ser el más usado de Internet.

API - (Application Program Interface). Interfaz de programación de Aplicaciones.

APRENDIZAJE POR COMPUTADOR - Hace referencia al uso de computadores como parte clave en la enseñanza y el aprendizaje haciéndolo parte integrante del entorno educativo.

APRENDIZAJE EN LINEA - Hace referencia a aquel tipo de aprendizaje que se lleva a cabo mediante la utilización de una red o INTERNET.

ARTEFACTO - Cualquier tipo de información que es producida o usada en el proceso de desarrollo de software.

BASE DE DATOS (MOTOR) - Aplicación informática que permite la gestión de datos y el manejo de la información.

CALL (Computer-assisted language learning) - Es un pseudo lenguaje que permite aproximar la enseñanza y aprendizaje en el que tecnología computacional es usada ya sea como presentación, refuerzo y acceso a material para aprender.  Usualmente incluye elementos interactivos y contenidos multimedia.

CASO DE USO – Artefacto utilizado para describir la funcionalidad de un sistema desde el punto de vista de los usuarios.

CBR  -  (Constant Bit Rate). Tasa de bits constante.

CGI - (Common Gateway Interface) Estándar para tender interfaces entre aplicaciones externas y servidores de información, tales como los servidores HTTP.

COOKIE -  Pequeño archivo de texto que se almacena en el disco duro al visitar una pagina Web y que sirve para identificar al usuario cuando se conecta de nuevo a dicha página.

CLASE -  En el desarrollo de software es la representación mediante un nombre, atributos y operaciones de un objeto o conjunto de objetos.

CRUD - Caso CRUD, caso que agrupa los casos de crear, leer, actualizar y borrar registros.

CSS -  (Cascading Style Sheet - Hojas en estilo de cascada). Reglas de estilo para documentos HTML.

DAEMON - (Disk And Execution MONitor) Un programa no invocado explícitamente pero que permanece esperando a que una situación especial ocurra  Los sistemas Unix ejecutan muchos daemons, principalmente para manipular solicitudes de servicios desde otro host o desde una red.

DBMS -  (Data Base Management System – Sistema de Gestión de Bases de Datos). Sistema de software que permite a los usuarios guardar y modificar información.

DEPURAR -  Relacionado con el software, detectar, localizar, corregir los problemas en un programa informático.

DESARROLLO ITERATIVO - Método de construcción de productos cuyo ciclo de vida está compuesto por un conjunto de iteraciones, las cuales tienen como objetivo entregar versiones del software.

DISCIPLINA - Conjunto de reglas para mantener el orden y la subordinación entre miembros de un cuerpo.

DOMINIO - Parte del nombre jerárquico con que se conoce a cada entidad conectada a Internet. Un dominio se compone de una serie de etiquetas o nombres separados por puntos.

DSL -  (Línea digital de abonado). Tecnología de red pública que proporciona ancho de banda amplio sobre el cableado tradicional de cobre en distancias limitadas. Hay cuatro tipos de DSL: ADSL, HDSL, SDSLY VDSL. Todos ellos se aprovisionan a través de pares de módems, estando un módem situado en al oficina central y el otro en el domicilio del abonado.

ENTORNO DE APRENDIZAJE - Un entorno de aprendizaje es el conjunto de conocimientos, herramientas de enseñanza y aprendizaje, espacios y personas involucradas en un proceso de aprendizaje dado.

ENTORNO VIRTUAL DE APRENDIZAJE - Un entorno virtual de aprendizaje (virtual learning enviroment) es un paquete de software cuya función es reemplazar o complementar un entorno de aprendizaje formal y tradicional.

eSALUD - De acuerdo con el Instituto de Estándares de Comunicaciones Europeo, es la aplicación de tecnologías de la información y la comunicación en todo el rango de funciones que afecta el sector salud. 

FILTRO - Cualquier función o programa que seleccione información automáticamente con un criterio preestablecido.

GNU – (GNU’s Not Unix). Sistema operativo, compuesto de pequeñas piezas individuales de software totalmente libre.

HIPERTEXTO - Método de preparar y publicar un texto ideal para el computador, con el que los lectores pueden escoger su propia ruta a través del material. La información se descompone en pequeñas unidades y después los hipervínculos se insertan en el texto.

HIPERVINCULO - Enlace. En un sistema de hipertexto, palabra subrayada o destacada que, cuando se pulsa en ella lleva a otro documento.

HTML - (Hyper text markup language – Lenguaje de marcado de hipertexto). Lenguaje de marcado que define la estructura de paginas Web. Utiliza etiquetas para denotar los diferentes objetos que componen una página.

HTTP -  (Hyper Text Transfer Protocol - Protocolo de Transferencia de Archivos). Estándar para la transferencia de mensajes entre navegadores web utilizando texto plano.

INCREMENTAL - Que puede instalarse por fases, cada una de ellas añadiendo nuevas funcionalidades.

INFRAESTRUCTURA - Conjunto de medios necesarios para el desarrollo de una actividad. 

INGENIERÍA DE SOFTWARE - Aplicación de un enfoque sistemático al diseño, desarrollo, operación y mantenimiento de software​.

INGENIERÍA INVERSA – Conjunto de etapas desarrolladas para obtener las bases del diseño y la forma de implementación de algún producto de ingeniería a partir de su estado final.

INTERCONEXIÓN - Unión o conexión entre sí de dos o más elementos.

INTERFAZ - Parte de un sistema en la que se desarrolla la comunicación o interacción entre el programa y el usuario (siendo este humano u otro sistema).

INTEROPERABILIDAD - Según IEEE, es la habilidad de dos o más sistemas para intercambiar mensajes y utilizar la información intercambiada.

IP – (Internet Protocol - Protocolo Internet). Protocolo de la capa de red de la pila TCP / IP que ofrece un servicio de intercambio de paquetes de datos entre host.

ITERACION - Repetición de un conjunto de pasos.

Kbps -  Kilobits por segundo.

KBps -  KiloBytes por segundo.


MÉTODO - Modo ordenado de proceder para llegar a un resultado o fin determinado. 

METODOLOGÍA - Parte de la lógica que estudia los métodos.

MITIGAR - Moderar, suavizar.

MODELADO – Descripción, en un lenguaje de máquina, de la forma o características de un objeto o conjunto de objetos.

MÓDULO - Una aplicación software que presenta una funcionalidad específica dentro de OpenSITEM. En el contexto del proyecto es sinónimo de subsistema.

MULTIPLATAFORMA -  Las aplicaciones pueden ser vistas en cualquier computador del mundo a través de múltiples plataformas (sistemas operativos, navegadores y hardware).

MySQL -  Sistema de administración para bases de datos relacionales (rdbms).

NRT-VBR - (Non-Real-Time Variable Bit Rate). Tasa de bits variable en tiempo no real, Esta categoría de servicio está encaminada para aplicaciones en tiempo no real, las cuales tienen características de tráfico en ráfaga.

NODO - Componente de una red de eSalud. Tiene atributos que definen su tipo y capacidad de interconectarse e interoperar con otros nodos de igual o diferente tipo. Ejemplos de nodos en OpenSITEM: Un electrocardiógrafo (tipo Equipo Médico), un hospital (tipo Entidad de Salud), un radiólogo (tipo Especialista), una toma eléctrica (tipo Componente Eléctrico), un smartphone (tipo Dispositivo de Comunicación), etc.

NTP - (Network Time Protocol – Protocolo de Tiempo de red) - Sistema de sincronización de tiempo para relojes de computadores a través de Internet.

OBJETO -  Un objeto es una instancia, un ejemplar, de una clase. Por extensión del término, en openSITEM se considera que un nodo es un ejemplar de una categoría de nodos.

OMG - Object Management Group. Organización que se encarga de crear y actualizar varios estándares de tecnologías orientadas a objetos incluyendo UML y BPMN.

OSI - Internetworking de sistemas abiertos. Programa internacional de normalización creado por ISO e ITU-T. Para desarrollar estándares para las redes de datos que faciliten la interoperabilidad entre equipos de varios fabricantes.

PASSWORDS - Contraseña. Método de seguridad empleado para identificar a un usuario. Sirve para dar acceso a personas con determinados permisos. En la actualidad se están generando de manera dinámica y distribuyendo vía correo electrónico o mensajes de texto.

PCR - (Peak Cell Rate). Tasa pico de celda.

PHP - (Hypertext Preprocessor"). Es un lenguaje interpretado de alto nivel ejecutado al lado del servidor. Según w3techs.com, para octubre de 2017 cerca del 80\% del total de sitios web del mundo empleaban este lenguaje.

PLANTILLAS -  Serie de archivos que definen la apariencia de una aplicación Web, y permiten cambiar la presentación de la aplicación con sólo modificar unos cuantos archivos sin necesidad de modificar el código que le da funcionalidad a la aplicación.

PLATAFORMA - Base o elemento de apoyo. Se refiere a una combinación específica de hardware y sistema operativo.

PROTOCOLO - Conjunto de instrucciones o lenguaje que utilizan los computadores para comunicarse entre sí.

PROTOTIPO - Original ejemplar o primer molde en que se fabrica una figura u otra cosa. Modelo original que sirve como ejemplo para futuros estados o formas.

QoS – (Quality Of Service - Calidad de servicio). Una medida de rendimiento en el sistema de transmisión que refleja la calidad de la transmisión y la disponibilidad del servicio.

RDSI – (Red digital de servicios integrados). Protocolo de comunicaciones que permite a las redes de las compañías telefónicas transportar datos, voz y otro trafico.

REGISTRO - Pequeña unidad de almacenamiento que contiene cierto tipo de datos. En el ámbito de las bases de datos es cada una de las fichas que forman un conjunto.

RT-VBR - (Real-Time Variable Bit Rate). Está encaminada a aplicaciones que requieren restricción en la variación de retardo (lo mínimo) y variación de retardo, que podrían ser apropiados para voz y vídeo en tiempo real.

SCRIPT -  Pequeños archivos de texto embebidos dentro del documento, que ofrecen una forma de extender los documentos HTML convirtiendo el contenido del documento en una página dinámica, pudiendo ser ejecutados al presentarse un evento como el movimiento del ratón.

SCORM (Sharable Content Object Reference Model) - Es una colección de normas y especificaciones para aprendizaje basado en la Web.  Define como debe ser la comunicación entre el contenido de lado al cliente y un sistema servidor, tratando aspectos como el empaquetamiento de información en archivos ZIP.

SERVIDOR - Dispositivos de un sistema que resuelve las peticiones de información de otros elementos del sistema a los que se denomina clientes.

SISTEMA DE INFORMACIÓN - Conjunto de componentes interrelacionados que permiten capturar, procesar, almacenar y distribuir la información para apoyar la toma de decisiones y el control en una institución.


SQL - (Structured Query Language Lenguaje Estructurado de Consultas). Lenguaje de programación interactivo y estándar para obtener y actualizar información de una base de datos.

TASA DE BITS  - Velocidad en Kilobits por segundo a la que puede transmitir un circuito virtual.

TCP/IP - TCP("Transmisión Control Protocol") e IP("Internet Protocol"). Protocolo para el control de la transmisión / Protocolo Internet. Nombre común que se da al paquete de protocolos desarrollados  por el DoD en los años 70’s con el fin de soportar la construcción de interworks a escala mundial.

TECNOLOGÍA - Conjunto de los conocimientos técnicos. Conjunto de los instrumentos y procedimientos industriales de un determinado producto.

TELEASISTENCIA - Los sistemas de teleasistencia permiten a los pacientes recoger datos acerca de su enfermedad, transmitir la información a un lugar remoto y usar la videoconferencia para discutir sus padecimientos y tratamientos con el profesional de la salud.

TELECONFERENCIA - Es la telecomunicación que se establece entre dos o más especialistas, para tratar diversos temas que en el caso de la Telemedicina, generalmente tienen que ver con los pacientes o  están relacionados con capacitación especializada.

TELEMEDICINA - Herramienta tecnológica para brindar servicios médicos a distancia mediante el intercambio de imágenes, voz, datos y video, haciendo uso de las tecnologías de la información y de las comunicaciones, permitiendo el diagnóstico, la opinión de casos clínicos y la provisión de cuidados de salud y educación médica.

TRANSICIÓN - Tiempo en el cual se cambia de un estado a otro.

UBR - (Unspecified Bit Rate). Esta categoría está encaminada para aplicaciones en tiempo no real, por ejemplo aquellas que no requieren restricciones en la variación de retardo (mínimo) y variaciones de retardo.

UML - (Unified Modeling Language - Lenguaje Unificado de Modelado). Brinda todas las características, tanto sintácticas como semánticas, para lograr caracterizar lógicamente cualquier tipo de software permitiendo ser utilizado en cualquier etapa del diseño y especialmente útil en aquellos desarrollos enfocado a objetos.

UP - Proceso Unificado. El Proceso Unificado puede concebirse como una idea, una plantilla que provee una infraestructura para ejecutar proyectos pero que no tiene en cuenta todos los detalles requeridos en dicha ejecución.

USUARIO - Dicho de una persona: que tiene derecho de usar de una cosa con cierta limitación.

VIDEOCONFERENCIA -  Comunicación simultánea bidireccional de audio y vídeo.

WEB - Sistema de comunicación para el intercambio de mensajes a través de Internet.

WWW - (World Wide Web). Sistema mundial de hipertexto que utiliza Internet como mecanismo de transporte. Conjunto de páginas de información con texto, imágenes sonido, etc., enlazadas entre sí.

XML -  (Extended Markup Language). Lenguaje de marcado extendido. Permite definir estructuras de información empleando etiquetas organizadas de manera jerárquica.
\begin{appendices}
\chapter{Declaración de Riesgos}
\label{anexo_riesgos}
El SITEM, como todo proyecto de objetivos ambiciosos, esta expuesto a una serie de contingencias que de no ser correctamente manejadas atentan la consecución de sus metas. Estos riesgos deben estar de la forma más óptima detectados, evaluados y medido su impacto. Cada vez que se avanza en el ciclo de desarrollo unos se vuelven más importantes que otros. A continuación se declaran los riesgos más importantes a los que se enfrenta el SITEM.

\begin{itemize}
\item \textbf{Baja disponibilidad de Tiempo.} 

La falta de tiempo por parte de los integrantes del proyecto debido al trabajo en paralelo de otros proyectos es un riesgo inminente que de seguro puede desencadenar en la no consecución de los objetivos. El reclutamiento de integrantes cuyo objetivo primario es el lleno de un requisito para grado pone en riesgo la continuidad en el desarrollo de algún hilo.

\textit{Métodos para la reducción del riesgo:} Es aquí cuando los métodos ágiles son de gran ayuda. El grupo de desarrollo fomenta la retroalimentación continua entre los diferentes hilos creando una especie de competencia sana en donde los componentes y prácticas más eficientes son detectadas y propagadas en las sesiones de capacitación. A partir de la versión 3.0, actualmente en desarrollo, los grupos adscritos no se les imponen un objetivo específico sino que las personas interesadas simplemente deciden que tarea es más relevante y su participación solo se tiene en cuenta a través de la producción - posicionamiento basado en el esfuerzo. 


\item \textbf{Pérdida de tiempo debido a procesos administrativos.} 

Las cargas administrativas inherentes a los procesos del proyecto consumen gran cantidad de tiempo y en ciertas circunstancias puede tomar más del esperado obstruyendo el desarrollo de las actividades en el proyecto. La dependencia de algunas actividades al formalismo burocrático desenfoca al grupo de trabajo.

\textit{Métodos para la reducción del riesgo:}  El trabajo se realiza en varias áreas y disciplinas de manera paralela brindando siempre la posibilidad de redistribuir recursos si se requiere esperar a que algún trámite administrativo se concrete.  El jefe del proyecto se encarga de realizar todos los trámites y generación de documentos necesarios con el fin de agilizar los procesos administrativos. 


\item \textbf{Falta de recursos para trabajo.} 

El mayor riesgo para la elaboración del proyecto y la consecución de los objetivos del mismo se encuentra en el no disponer de recursos para dichas tareas. Principalmente un equipo de computo capaz de soportar sistemas operativos Linux así como de  un espacio físico de trabajo dónde se encuentre dicho equipo y en dónde los integrantes del proyecto puedan realizar sus actividades y centralizar información.

\textit{Métodos para la reducción del riesgo:}  Cada uno de los integrantes del proyecto está en capacidad de realizar las tareas que le corresponden por su cuenta fomentando el teletrabajo. El jefe del proyecto se encargo de realizar la centralización del trabajo y la información por medio del préstamo al proyecto de recursos propios. En fases posteriores el grupo de investigación recibe apoyo del Centro de Investigaciones y Desarrollo Científico de la Universidad Distrital con lo que se adquiere un servidor con acceso público a Internet.



\item \textbf{Falta de información y demora en la consecución de la misma.}

El proyecto requiere y se basa en cierta información proveniente de diversas fuentes como publicaciones, manuales y estudios realizados dentro y fuera del grupo GITEM; la demora o imposibilidad de consecución de información calificada se convierte en un riesgo para el desarrollo normal del proyecto.

\textit{Métodos para la reducción de riesgo:} Se realiza desde el comienzo del proyecto un listado de información necesario e indispensable y se inicia la búsqueda y consecución de la misma desde el inicio.  El jefe de proyecto realiza los procedimientos requeridos con el fin de contar con los resultados de estudios previos en el interior del GITEM. Un cúmulo documental se mantiene disponible en todo momento para la comunidad de desarrollo, las jornadas de capacitación y el despliegue de presentaciones focalizadas es también de gran ayuda.

\item \textbf{Falla en la dirección del proyecto.}

Existe la posibilidad de que el director del proyecto o de alguno de sus ayudantes cometa algún error o falla atentando de ésta manera con el proceso natural de desarrollo.

\textit{Métodos para la reducción de riesgo:}  Se realiza un seguimiento continuo a todas las tareas asociadas a la dirección del proyecto. El concepto y objetivos del SITEM plasmados en su documento de Visión se han convertido en un artefacto de consulta continua y recurrente.

\item \textbf{Valoración nula o errónea de los resultados.} 

Al terminar una iteración dentro del proceso de desarrollo es posible que no se obtengan los resultados esperados, el exceso de planificación y la rigidez frente al cambio puede generar que se subvaloren los resultados obtenidos y, con la excusa de no cambiar un documento impreso y empastado, sacrificar nuevas ideas y mejoras.

\textit{Métodos para la reducción de riesgo:} Se ha difundido claramente en el grupo de desarrollo el hecho de que todos los artefactos están en construcción; nunca se valora uno de ellos como la versión definitiva. El uso de términos correctos como las versiones estables permite que los entes de control al interior del grupo puedan realizar métricas a corto plaza que en sumatoria pueden arrojar datos valiosos en cuanto al avance del proyecto. Los objetivos de los trabajos de grado pueden ser modificados respecto a los anteproyectos siempre y cuando se conserve el espíritu de los mismos. En la práctica se utiliza Cervisia, un sistema de CVS, que permite el seguimiento del proyecto, sus resultados y sus errores.

\item \textbf{Falta de motivación de los integrantes del proyecto.} 

Debido a la posibilidad de que los integrantes del grupo en ciertos momentos y etapas del desarrollo del proyecto carezcan de la motivación necesaria, se corre el riesgo de que las actividades asignadas no se realicen o se realicen de una forma no satisfactoria. Usualmente esto es fruto de reclutamientos forzosos de personal.

\textit{Métodos para la reducción de riesgo:} A integrantes de fases preliminares se les dicta capacitaciones en los métodos de desarrollo ágiles y se les descarga en gran medida la responsabilidad de documentar el desarrollo - enfoque 100\% centrado en el código. Para el cumplimiento de las tareas y actividades se asigna un tiempo adicional con el objetivo de que los integrantes del grupo resolvieran sus dudas y preocupaciones sobre el proyecto y así atacar una determinada tarea entre todos los integrantes.

En la actualidad el proyecto es de carácter abierto y cualquier persona puede; sin una intermediación directa del grupo de investigación, realizar trabajos sobre el mismo. Los desarrollos que deseen ser validados podrán solicitar dicha certificación al grupo de desarrollo principal el cuál sopesará los resultados y en algunos casos concretos el proceso de desarrollo. Es decir, a partir de la versión 3.0 del SITEM ningún nuevo integrante es regulado por las políticas del grupo principal de desarrollo, es una decisión individual el seguimiento o no de los lineamientos.

\item \textbf{Imposibilidad total de trabajo.} 

Posiblemente por motivos externos al proyecto como paros, vacaciones o cierres, es posible que el trabajo se detenga totalmente.

\textit{Métodos para la reducción de riesgo:} Teletrabajo, depreciación de la comunicación persona a persona y uso intensivo de herramientas de trabajo en grupo dentro de la plataforma del SITEM. Especialmente las de primer nivel como correo electrónico, canales IRC y Foros. Las de segundo nivel tales como chat y videoconferencia en tiempo real han sido poco utilizadas para minimizar el riesgo colateral de segregación tecnológica. 

\item \textbf{Capacitación deficiente.} 

Es posible que algún miembro del proyecto asignado a alguna tarea no posea todos los conocimientos necesarios para realizarla.

\textit{Métodos para la reducción de riesgo:} Al interior del SITEM se han creado cursos en línea con contenidos actualizados a los que los integrantes podrán acceder de forma asíncrona a través de Internet. En los foros se guarda información relevante a los problemas en el desarrollo de tareas con el objetivo de que todo el equipo de trabajo pueda encontrar soluciones basadas en la colaboración. El jefe del proyecto puede asignar recursos enfocados en la capacitación de ciertas áreas específicas.

\item \textbf{Mal funcionamiento de herramientas de software y hardware.} 

La posibilidad de fallo en los equipos de trabajo y de los recursos utilizados puede causar la para y perdida de tiempo.

\textit{Métodos para la reducción de riesgo:} En el grupo de desarrollo principal existe un plan de mantenimiento a corto plazo que contempla actividades para el aseguramiento de la integridad de la plataforma de hardware y de los datos asociados al proyecto. Con la autorización para desplegar el hilo principal en servidores de aplicaciones se tendrá un mayor respaldo.

\item \textit{No seguimiento del plan de trabajo.} 

Las libertades propias de cada integrante del grupo además del conjunto formado por todos los otros riesgos pueden causar un desarrollo no planeado. En sí este riesgo solo es evidente cuando se captan recursos que van enfocados al cumplimiento de objetivos específicos, a partir de la versión 3.0 el grupo principal será el encargado de dictaminar los alcances de los nuevos releases.

\textit{Método para la reducción de riesgo:}  Uno o más de los integrantes debe actuar como ingeniero de procesos y se encargará de verificar el cumplimiento del plan de trabajo.

\item \textbf{Desacuerdo entre los integrantes del grupo.} 

La posibilidad de desacuerdos de trabajo o personales puede darse al interior del proyecto. 

\textit{Método para la reducción de riesgo:} En el caso de desarrollo novedoso los integrantes en desacuerdo deberán seguir las reglas de contribución concurrente. Para todos los demás casos - procedimiento depreciado - el jefe de proyecto y el ingeniero de procesos deberán encargarse de tomar cualquier decisión con miras al cumplimiento de los objetivos.

\item \textbf{Desviación de los recursos de trabajo.} 

Es posible que los recursos obtenidos para el desarrollo del proyecto sean utilizados para otras actividades que no involucran la obtención de objetivos específicos.

\textit{Método para la reducción de riesgo:}  El ingeniero de procesos verifica el correcto cumplimiento del plan de trabajo así como la correcta utilización de los recursos.  Se realiza un historial de utilización de los recursos.

\item \textbf{Otros riesgos.} 

Existen otros riesgos como la renuncia al proyecto de algún miembro, el cambió de los objetivos del proyecto, cambió en las características del mercado y eventos externos que afecten el normal desarrollo del proyecto.

\textit{Método para la reducción de riesgo:}  El proceso de desarrollo ágil - con elementos claves del proceso unificado, la utilización de artefactos, el desarrollo centrado en la persona y el compromiso con la filosofía general de la libertad y el derecho a la información.
\end{itemize}


\include{licencia}
\chapter{Modelo de Requisitos}
\label{modelo_requisitos}

En el SITEM el modelo de requisitos - junto con el de datos, es el que más importancia reviste ya que es un pilar importante para garantizar que la funcionalidad implementada es de interés para un actor específico.

\section{Modelo de Casos de Uso}
\subsection{Características Generales de los Actores}

El SITEM ofrece funcionalidad que es de interés para alguno de los siguientes actores los cuales se representan en la figura \ref{actores}:

\begin{itemize}
\item \textbf{Administrador}: Se encarga de la gestión de usuarios, el registro de nuevos subsistemas y tareas de administración tales como copias de seguridad, corrección de errores, edición de páginas, etc.
\item \textbf{Entidad de Salud}: Para la gestión especifica de información de una institución en el subsistema de Entidades de Salud. Por política general un usuario de este tipo esta confinado a la institución que crea. Para poder gestionar información de otra entidad deberá solicitar autorización a la instancia de actor dueña del registro.
\item \textbf{Servicios Médicos:} Gestiona la información de las diferentes secciones del subsistema Servicios Médicos.
\item \textbf{Especialista:} Con acceso exclusivo a su ambiente de trabajo que en la actualidad incluye registro de la hoja de vida, blog y calendario. El presente modelo de requisitos especifica algunos casos de uso para el manejo de historia clínica de pacientes y tele-consulta que están en etapa preliminar de diseño.
\item \textbf{Profesional TIC:} Actor abstracto que agrupa los actores que gestionan información en los subsistemas de: Equipos Médicos, Operadores de Telecomunicaciones, Proyectos de Telemedicina y Tecnologías de Interconexión.
\item \textbf{Usuario General:} Usuario general de consulta tiene acceso a todos los subsistemas del SITEM en modo de consulta. Puede generar informes pero no se le permite modificar ningún registro.
\item \textbf{Consultor:} Especificación del usuario general en donde además de consultar información tiene acceso a los módulo de tablas de análisis, agenda y blog.
\end{itemize}

\begin{figure}
 \centering
 \includegraphics[width=156mm, height=182mm]{actores.png}
 \caption{Conjunto de Actores del SITEM}
 \label{actores}
\end{figure}

\subsection{Casos de Usos}
Los artefactos correspondientes a los casos de uso se han empaquetado de acuerdo al subsistema en el cual se desarrollan. En el presente documento se incluyen solo los casos que se consideran nucleares dentro de cada subsistema del SITEM. De la misma forma solo se incluyen las especificaciones de mayor relevancia e impacto.

\begin{figure}
 \centering
 \includegraphics[width=156mm, height=182mm]{casos_admin.png}
 \caption{Casos de uso principales del Actor administrador}
 \label{casos_admin}
\end{figure}

\begin{figure}
 \centering
 \includegraphics[width=156mm, height=182mm]{casos_consultor.png}
 \caption{Casos de uso principales del Actor Consultor/Usuario General}
 \label{casos_consultor}
\end{figure}

\begin{figure}
 \centering
 \includegraphics[width=156mm, height=182mm]{casos_entidad.png}
 \caption{Casos de uso principales del Actor Entidad de Salud}
 \label{casos_entidad}
\end{figure}

\begin{figure}
 \centering
 \includegraphics[width=156mm, height=182mm]{casos_servicios.png}
 \caption{Casos de uso principales del Actor Servicios Médicos}
 \label{casos_servicios}
\end{figure}

\begin{figure}
 \centering
 \includegraphics[width=156mm, height=182mm]{casos_tecnologias.png}
 \caption{Casos de uso principales del Actor Tecnologías de Interconexión}
 \label{casos_tecnologia}
\end{figure}

\begin{figure}
 \centering
 \includegraphics[width=156mm, height=182mm]{casos_equipos.png}
 \caption{Casos de uso principales en el Subsistema Equipos de Medicina}
 \label{casos_equipos}
\end{figure}

\begin{figure}
 \centering
 \includegraphics[width=156mm, height=182mm]{casos_equipos.png}
 \caption{Casos de uso principales en el Subsistema de Operadores de Servicios de Telecomunicaciones}
 \label{casos_telecomunicaciones}
\end{figure}

\chapter{Modelo de Análisis y Diseño}
\label{modelo_analisis}
\section{Especificaciones de Casos de Usos}

Para analizar en detalle los requerimientos del sistema se especifica mediante plantillas aquellos casos de uso que se consideran de importancia para el desarrollo base - conocidos como \textit{Casos de Uso Nucleares}. La tabla \ref{tabla_plantilla} muestra las secciones básicas que contiene un caso de uso. En aquellos casos de uso que una o varias secciones carezcan de contenido relevante se omiten por completo su declaración.

\begin{table}
\begin{center}
\begin{tabular}{|l|p{10cm}|}
\hline
\textbf{Caso de Uso}&\\
\hline
Nombre & Nombre que identifica el caso de Uso usualmente es el mismo utilizado en el diagrama de Casos de Uso\\
\hline
Objetivo & Beneficio que obtiene el actor con la ejecución de este caso de uso.\\
\hline
Código Interno & Código único que identifica al Caso de Uso dentro del repositorio de artefactos.\\
\hline
Actores & Usuarios que intervienen en el caso de uso\\
\hline
Precondiciones & Estado en que debe encontrarse el SITEM antes de ejecutarse el caso de uso.\\
\hline
Flujo Básico & Flujo principal de actividades. Ambiente ideal.\\
\hline
Flujo Alternativo y de error & Actividades que bifurcan el flujo básico. Si existe más de un flujo alternativo este debe colocarse en una nueva fila.\\
\hline
Postcondiciones & Estado en que queda el SITEM después de ejecutado el caso de uso.\\
\hline
Puntos de Extensión & Secuencias de acciones que extienden el flujo del caso de uso.\\
\hline
\end{tabular}
\caption{Plantilla Genérica para la Especificación de los Casos de Uso.}
\label{tabla_plantilla} 
\end{center}
\end{table}


\begin{table}
\begin{center}
\begin{tabular}{|l|p{10cm}|}
\hline
\textbf{Caso de Uso}&\\
\hline
Nombre & Registrarse en el SITEM\\
\hline
Objetivo & El actor logra crear una cuenta en el SITEM con un rol específico para poder trabajar en un subsistema dado.\\
\hline
Código Interno & UC-GENERAL-001 \\
\hline
Actores & Usuario General\\
\hline
Precondiciones & Ninguna.\\
\hline
Flujo Básico & 1. El usuario general selecciona la opción de nuevo usuario desde la página principal del SITEM.\\
& 2. El SITEM muestra un formulario con los campos:\\
& Nombres\\
& Apellidos\\
& Correo Electrónico\\
& Teléfono\\
& Nombre de Usuario\\
& Clave\\
& Reescriba la clave\\
& Acceso Requerido\\
& 3. El usuario diligencia uno a uno los campos requeridos y opcionales.\\
& 4. El usuario envía los datos al SITEM.\\
& 5. El SITEM verifica que los datos tengan los formatos esperados.\\
& 6. El SITEM ingresa el registro a la base de datos colocando el campo de estado en 1 - registrado sin autorización.\\
& 7. El SITEM redirecciona a la página de registro exitoso.\\
& 8. El usuario acepta el mensaje.\\
\hline
Postcondiciones & Se agregó un registro en la base de datos con el campo de estado en 1.\\
\hline
Casos de uso relacionados&Seleccionar Rol en el SITEM\\
\hline
\end{tabular}
\caption{Caso de Uso Registrarse en el SITEM}
\label{casouso1} 
\end{center}
\end{table}

\begin{table}
\begin{center}
\begin{tabular}{|l|p{10cm}|}
\hline
\textbf{Caso de Uso}&\\
\hline
Nombre & Administrar autorizaciones de Usuario\\
\hline
Objetivo & Proveer un mecanismo eficaz para que el administrador general del SITEM pueda gestionar el estado de autorización de los usuarios en los diferentes subsistemas.\\
\hline
Código Interno & UC-GENERAL-002 \\
\hline
Actores & Administrador\\
\hline
Precondiciones & Debe existir por lo menos un usuario registrado en el sistema diferente al administrador.\\
& El administrador se encuentra correctamente autenticado y autorizado en el subsistema de administrador.\\
\hline
Flujo Básico & 1. El Administrador selecciona la opción Usuarios desde el menú principal del subsistema Administrador.\\
& 2. El SITEM muestra un listado con los datos básicos de diferentes usuarios registrados:\\
& Nombres\\
& Apellidos\\
& Correo Electrónico\\
& Acceso Requerido\\
& 3. \textit{Punto de Extensión} 1.\\
& 4. \textit{Punto de Extensión} 2.\\
& 5. El administrador verifica que los datos del usuario son reales.\\
& 6. El administrador seleccionada la casilla de verificación y acepta el trámite.\\
& 7. El SITEM procesa el formulario colocando el estado del usuario en valor 2 - Registrado y Autorizado.\\
& 8. El SITEM indica con un mensaje el éxito en la operación de autorización.\\
& 9. El SITEM envía un mensaje de texto al usuario indicando que ha sido autorizado.\\
& 8. El Administrador acepta el mensaje de éxito.\\
\hline
Postcondiciones & Se cambia el valor en el campo estado del registro correspondiente al usuario.\\
\hline
Puntos de Extensión & 1:direccion=“avanzar” o direccion=“retroceder” extend Navegar en listado. \\
& 2:Opción=“buscar” extend Busqueda condicional de registro. \\
\hline
\end{tabular}
\caption{Caso de Uso Autorizar un usuario para acceder al SITEM}
\label{casouso2} 
\end{center}
\end{table}

\begin{table}
\begin{center}
\begin{tabular}{|l|p{10cm}|}
\hline
\textbf{Caso de Uso}&\\
\hline
Nombre & Generar Copia de Seguridad\\
\hline
Objetivo & Generar una copia de seguridad de los datos contenidos en la base de datos asociada al SITEM.\\
\hline
Código Interno & UC-GENERAL-003 \\
\hline
Actores & Administrador\\
\hline
Precondiciones & El administrador se encuentra correctamente autenticado y autorizado en el subsistema de administrador.\\
\hline
Flujo Básico & 1. El Administrador selecciona la opción Copia de Seguridad el menú principal del subsistema Administrador.\\
& 2. El SITEM muestra un listado con las tablas opcionales para la copia de seguridad.\\
& 3. El administrador selecciona la casilla de verificación de las tablas que desea sean copiadas.\\
& 4. \textit{Punto de Extensión} 1.\\
& 5. El administrador acepta la selección.\\
& 6. El SITEM muestra un listado con las tablas que serán copiadas y un formulario con los campos:\\
& Nombre del Archivo.\\
& Ruta de Descarga.\\
& 7. El usuario diligencia uno a uno los campos requeridos.\\
& 8. El usuario envía los datos al SITEM.\\
& 9. El SITEM realiza una copia de los registros escribiéndolos uno a uno en los archivos de destino.\\
& 10. El SITEM redirecciona a la pagina de operación exitosa.\\
& 11. El usuario acepta el mensaje.\\
\hline
Postcondiciones & Se crea n archivos en la carpeta de destino con el contenido de las n tablas seleccionadas para copia de seguridad.\\
\hline
Puntos de Extensión & 1:opcion=“todo” extend Seleccionar todos los cuadros.\\
\hline
\end{tabular}
\caption{Caso de Uso realizar Copia de Seguridad}
\label{casouso3} 
\end{center}
\end{table}

\begin{table}
\begin{center}
\begin{tabular}{|l|p{10cm}|}
\hline
\textbf{Caso de Uso}&\\
\hline
Nombre & Elaborar Tablas de Análisis\\
\hline
Objetivo & Obtener un repositorio de análisis de algún componente del modelo jerárquico de seguimiento a proyectos.\\
\hline
Código Interno & UC-GENERAL-004 \\
\hline
Actores & Consultor\\
\hline
Precondiciones & Existe en el SITEM un modelo jerárquico de seguimiento a proyectos.\\
\hline
Flujo Básico & 1. El Consultor selecciona la opción Seguimiento desde el menú principal del subsistema Consultor.\\
& 2. El SITEM muestra el modelo de seguimiento a proyectos con sus componentes de primer nivel.\\
& 3. \textit{Punto de Extensión} 1.\\
& 4. El Consultor selecciona la opción de Analizar para un componente.\\
& 5. El SITEM muestra un formulario con los campos:\\
& Valoración Cualitativa.\\
& Valoración Cuantitativa.\\
& Juicio.\\
& Diagnóstico.\\
& Fortalezas.\\
& Oportunidades.\\
& Debilidades.\\
& Amenazas.\\
& Directrices de Mejoramiento.\\
& Directrices de Acción.\\
& 6. El usuario diligencia uno a uno los campos requeridos y opcionales.\\
& 8. El usuario envía los datos al SITEM.\\
& 5. El SITEM verifica que los datos tengan los formatos esperados.\\
& 6. El SITEM ingresa el registro a la base de datos.\\
& 7. El SITEM redirecciona a la página de registro exitoso.\\
& 8. El usuario acepta el mensaje.\\
\hline
Postcondiciones & Existe un registro de análisis asociado a un componente y un consultor.\\
\hline
Puntos de Extensión & 1:opcion=“mas” extend Mostrar Componentes de Nivel Inferior\\
\hline
\end{tabular}
\caption{Caso de Uso Elaborar Tablas de Análisis}
\label{casouso4} 
\end{center}
\end{table}

\begin{table}
\begin{center}
\begin{tabular}{|l|p{10cm}|}
\hline
\textbf{Caso de Uso}&\\
\hline
Nombre & Ingresar una Entidad\\
\hline
Objetivo & Registrar una nueva entidad de Salud en el SITEM.\\
\hline
Código Interno & UC-GENERAL-005 \\
\hline
Actores & Entidad Salud\\
\hline
Precondiciones & El usuario Entidad Salud se encuentra autorizado y autenticado en el subsistema Entidades de Salud.\\
\hline
Flujo Básico & 1. Entidad Salud selecciona la opción Nueva Entidad desde el menú principal del subsistema Entidades de Salud.\\
& 2. El SITEM muestra un formulario con los campos:\\
& Nombre de la Entidad.\\
&Nombre Corto.\\
&Logosímbolo.\\
&NIT.\\
&Fecha de Fundación.\\
&Dirección Principal.\\
&Teléfono Principal (PBX).\\
&Misión.\\
&Visión.\\
&Descripción.\\
&Comentarios.\\
& 3. El usuario diligencia uno a uno los campos requeridos y opcionales.\\
& 4. El usuario envía los datos al SITEM.\\
& 5. El SITEM verifica que los datos tengan los formatos esperados.\\
& 6. El SITEM comprueba que no exista otra Entidad registrada con el mismo NIT.\\
& 7. El SITEM ingresa el registro a la base de datos.\\
& 8. El SITEM redirecciona a la página de registro exitoso.\\
& 9. El usuario acepta el mensaje.\\
\hline
Flujo Alternativo & 5.A. Los datos no tienen el formato adecuado. \\
& 6.A. El SITEM informa el error.\\
& 7.A. \textit{Punto de Extensión} 1.\\
\hline
Flujo Alternativo & 6.A. Existe una entidad registrada con el mismo NIT. \\
& 7.A. El SITEM informa el error.\\
& 8.A. \textit{Punto de Extensión} 1.\\
\hline
Postcondiciones & Existe un registro de una entidad de salud.\\
\hline
Puntos de Extensión & 1:opcion=“corregir” extend Mostrar Formulario Corrección.\\
\hline
\end{tabular}
\caption{Caso de Uso Ingresar una Nueva Entidad de Salud.}
\label{casouso5} 
\end{center}
\end{table}

\begin{table}
\begin{center}
\begin{tabular}{|l|p{10cm}|}
\hline
\textbf{Caso de Uso}&\\
\hline
Nombre & Consultar información básica de una entidad de Salud.\\
\hline
Objetivo & Obtener en pantalla los datos básicos de una entidad de salud.\\
\hline
Código Interno & UC-GENERAL-006 \\
\hline
Actores & Entidad Salud, usuario general\\
\hline
Precondiciones & El usuario se encuentra autorizado y autenticado en el subsistema Entidades de Salud.\\
\hline
Flujo Básico & 1. Entidad Salud selecciona la opción Entidades desde el menú principal del subsistema Entidades de Salud.\\
& 2. El SITEM muestra un listado de 25 entidades ordenadas alfabéticamente por nombre.\\
& 3. \textit{Punto de Extensión} 1.\\
& 4. El usuario selecciona una entidad de salud desde el listado.\\
& 5. El SITEM realiza una búsqueda con el id de la entidad.\\
& 6. El SITEM muestra en pantalla el menú secundario para solicitar edición y los datos de la entidad:\\
& Nombre de la Entidad.\\
&Nombre Corto.\\
&Logosímbolo.\\
&NIT.\\
&Fecha de Fundación.\\
&Dirección Principal.\\
&Teléfono Principal (PBX).\\
&Misión.\\
&Visión.\\
&Descripción.\\
& 7. El usuario acepta los datos.\\
\hline
Puntos de Extensión & 1:direccion=“avanzar” o direccion=“retroceder” extend Navegar en listado. \\
\hline
\end{tabular}
\caption{Caso de Uso Consultar información básica de una entidad de Salud.}
\label{casouso6} 
\end{center}
\end{table}

\begin{table}
\begin{center}
\begin{tabular}{|l|p{10cm}|}
\hline
\textbf{Caso de Uso}&\\
\hline
Nombre & Editar un registro en el SITEM.\\
\hline
Objetivo & Editar la información que se encuentra en un registro del SITEM. La actualización puede involucrar más de una entidad en la capa de persistencia\\
\hline
Código Interno & UC-GENERAL-007\\
\hline
Actores & Profesional TIC, entidad salud, administrador, usuario general\\
\hline
Precondiciones & El usuario se encuentra autorizado y autenticado en el subsistema.\\
\hline
Flujo Básico & 1. El usuario selecciona la opción Editar Registro desde el menú secundario del subsistema.\\
& 2. El SITEM muestra un formulario rellenado con los datos del registro correspondiente.\\
& 3. El usuario editada los valores dentro de los controles del formulario.\\
& 4. El usuario envia el formulaenvíal SITEM.\\
& 5. El SITEM verifica que los datos editados no violen alguna regla de integridad referencial.\\
& 6. El SITEM actualiza los registros en la capa de persistencia.\\
& 7. El SITEM muestra al usuairo un mensausuarioxito.\\
& 8. El usuario acepta el mensaje.\\
\hline
Flujo Alternativo & 5.A. Existe un error de integridad referencial. \\
& 7.A. El SITEM informa el error.\\
& 8.A. \textit{Punto de Extensión} 1.\\
\hline
Puntos de Extensión & 1:opcion=“corregir” extend Mostrar Formulario Corrección.\\
\hline
Postcondiciones & Se actualizan los registros correspondientes.\\
\hline
\end{tabular}
\caption{Caso de Uso Editar un registro en el SITEM}
\label{casouso7} 
\end{center}
\end{table}

\begin{table}
\begin{center}
\begin{tabular}{|l|p{10cm}|}
\hline
\textbf{Caso de Uso}&\\
\hline
Nombre & Asociar un protocolo de comunicaciones al modelo OSI.\\
\hline
Objetivo & Asociar un protocolo de comunicaciones al modelo de referencia OSI.\\
\hline
Código Interno & UC-GENERAL-008\\
\hline
Actores & Profesional TIC, Tecnologías.\\
\hline
Precondiciones & El usuario se encuentra autorizado y autenticado en el subsistema.\\
& Existe registrado por lo menos un protocolo de comunicaciones en el SITEM.\\
\hline
Flujo Básico & 1. El usuario selecciona la opción \textbf{Más Información} desde el menú secundario del subsistema.\\
& 2. El SITEM muestra la información del protocolo asociada por áreas temáticas.\\
& 3. El usuario selecciona la opción Clasificar OSI.\\
& 4. El SITEM muestra el grafico de siete capgráficomodelo OSI.\\
& 5. El usuario selecciona una o varias capas del modelo.\\
& 6. El SITEM asocia el id del protocolo a cada una de las capas del modelo OSI seleccionadas por el usuario.\\
& 7. El SITEM muestra el modelo OSI extendido con los demás protocolos registrados en cada capa.\\
& 8. El usuario acepta el registro.\\
\hline
Flujo Alternativo & 5.A. El usuario no selecciona ninguna capa. \\
& 7.A. El SITEM regresa al punto 2 del caso de uso.\\
\hline
Postcondiciones & El conjunto de protocolos asociados a una capa del modelo OSI se incrementa.\\
\hline
\end{tabular}
\caption{Caso de Uso Asociar un protocolo de comunicaciones al modelo OSI.}
\label{casouso8} 
\end{center}
\end{table}

\begin{table}
\begin{center}
\begin{tabular}{|l|p{10cm}|}
\hline
\textbf{Caso de Uso}&\\
\hline
Nombre & Borrar un registro del SITEM.\\
\hline
Objetivo & Eliminar un registro en algún subsistema del SITEM garantizando que solo el experto en información lo realice y se mantenga la integridad referencial en los datos.\\
\hline
Código Interno & UC-GENERAL-009\\
\hline
Actores & Profesional TIC, Entidad Salud, Consultor, Administrador.\\
\hline
Precondiciones & El usuario se encuentra autorizado y autenticado en el subsistema.\\
& Existe un registro en el SITEM.\\
& El usuario a creado el registro y este no tiene información asociada.\\
\hline
Flujo Básico & 1. El usuario selecciona la opción \textbf{Eliminar Registro} desde el menú secundario del subsistema.\\
& 2. El SITEM muestra un mensaje de conformación de eliminación con los datos básicos del registro.\\
& 3. El usuario selecciona la opción de \textbf{Confirmar Borrado}.\\
& 4. El SITEM elimina el registro cumpliento las restrcumpliendoe claves foraneas.\\
& 5. El foráneasestra un mensaje indicando que el registro se ha borrado del sistema.\\
& 6. El usuario acepta el mensaje.\\
& 7. El SITEM redirecciona a la página en donde se encontraba el usuario antes del proceso de borrado.\\
\hline
Flujo Alternativo & 3.A. El usuario no acepta borrar el registro. \\
& 4.A. Continua en el punto 7 del flujo principal.\\
\hline
Postcondiciones & El registro se borra del sistema.\\
\hline
\end{tabular}
\caption{Caso de Uso para Borrar un registro en el SITEM.}
\label{casouso9} 
\end{center}
\end{table}

\begin{table}
\begin{center}
\begin{tabular}{|l|p{10cm}|}
\hline
\textbf{Caso de Uso}&\\
\hline
Nombre & Acceder a una página del SITEM.\\
\hline
Objetivo & Ingresar a una página específica dentro del sistema.\\
\hline
Código Interno & UC-GENERAL-010\\
\hline
Actores & Profesional TIC, Entidad Salud, Consultor, Administrador.\\
\hline
Precondiciones & El usuario está autorizado para acceder a la página.\\
& La página se encuentra registrada en el SITEM.\\
& La página tiene uno o más bloques asociados.\\
\hline
Flujo Básico & 1. El usuario elige un enlace a una página dentro del SITEM.\\
& 2. El SITEM verifica que el usuario tiene una sesión  válida\\
& 3. El SITEM rescata los valores de la página desde la base de datos.\\
& 4. El SITEM verifica que el usuario tenga los privilegios necesarios para ingresar a la página.\\
& 5. El SITEM consulta la estructura de la página desde la base de datos.\\
& 6. El SITEM envía secuencialmente el código HTML correspondiente a cada una de los bloques que constituyen la página.\\
& 7. El SITEM registra el acceso del usuario en la base de datos.\\
& 8. El SITEM actualiza la información de sesión.\\
\hline
Flujo Alternativo & 4.A. El usuario no tiene los privilegios para ver la página. \\
& 5.A. El SITEM registra un atento de ingreso ilegal.\\
& 6.A. El SITEM muestra un mensaje informando el error.\\
\hline
Postcondiciones & El usuario ingresa a una página dentro del SITEM.\\
\hline
\end{tabular}
\caption{Caso de Uso para Acceder a una página del SITEM.}
\label{casouso10} 
\end{center}
\end{table}
\chapter{Modelo de Implementación}
\label{modelo_implementacion}

El modelo de implementación del SITEM esta compuesto básicamente por los siguientes artefactos que en su conjunto representan lo que comunmente se denomina el aplicativo:
\begin{itemize}
\item \textbf{Bloques:} Unidad Básica de funcionalidad. Se pueden pensar en ellos como “instancias” persistentes de clases abstractas. En especial se tienen tres clases:
\begin{itemize}
\item Administración: Con atributos y métodos para mostrar directorios de datos.
\item Menú: Con métodos especializados en la administración de enlaces dentro del SITEM.
\item Registro: Para la realización de casos CRUD.
\end{itemize}
\item \textbf{Clases}: Descriptores para varios tipos de objetos entre las cuales se tiene:
\begin{itemize}
\item DBMS: Interaccción con la bases de datos.
\item Página: Describe objetos que realizan la construcción en tiempo de ejecución de las páginas.
\item Encriptar: Descriptor para objetos que se encargan de codificar y decodificar los datos en el SITEM. El conjunto de operaciones debe ser manipulado en cada implementación del SITEM para garantizar un alto nivel de seguridad.
\item Autenticacion: Con descripción de atributos y operaciones que controlan las rutinas de AAA en el SITEM.
\item Config: Clasificador de objetos que manipulan las variables de configuración globales.
\item Html: Descriptor de controles de formularios en el SITEM.
\item Sesión: Operaciones y atributos para el control de sesiones en el SITEM luego del proceso de AAA.
\item Mensaje: Para objetos que administran mensajes de interacción con los actores.
\item Sql: Clase para describir objetos especializados en gestionar archivos con extensión SQL.
\item Navegacion: Con operaciones específicas para el control de desplazamiento entre conjuntos de registros.
\end{itemize}
\item \textbf{Función:} Métodos JavaScript para la validación y control de navegación en el lado del cliente.
\item \textbf{Estilo:} Para el control de la capa de Interfaz Gráfica.
\end{itemize}

\begin{figure}
 \centering
 \includegraphics[width=156mm, height=182mm]{mapa_navegacion.png}
 \caption{Modelo General de Navegación}
 \label{mapa_navegacion}
\end{figure}

\section{Código Fuente del SITEM}

La siguiente porción de código fuente representa el formato general que se encuentra en el SITEM. Por patrón general se recomienda a todos los grupos que participan en el desarrollo que mantengan un esquema de codificación claro y documentación \textit{in situ} suficiente para aclarar secciones.

\subsection{Clase Página}

La clase Página tiene las operaciones:
\begin{itemize}
\item pagina: Constructor.
\item especificar pagina: Inicializar variables privadas.
\item procesar pagina: Controlar redireccionamientos.
\item ancho seccion: Implementa control de secciones colapsadas.
\item armar seccion: Para seleccionar los bloques que contiene cada página.
\end{itemize}

Algunos métodos instancian la clase DBMS y HTML. Estas por lo tanto deben ser visibles.
\begin{verbatim*}

class pagina
{

	//Metodo constructor
	function pagina(id_pagina,configuracion)
	{
		//Declaracion de variable parta controlar accesos indebidos
		GLOBALS["autorizado"]=TRUE;
		this->especificar_pagina(id_pagina);
		
		if(!isset(_POST['registro_compuesta']))
		{
			if(!isset(_REQUEST['action']))
			{
				
				this->mostrar_pagina(configuracion);
			}
			else
			{
				//echo 'Procesamiento de la pagina';
				this->procesar_pagina(configuracion);
			}
		}
		else
		{
			this->mostrar_pagina(configuracion);		
		}
	}
	//Fin del metodo constructor
	
	
	
	//Metodo especificar_pagina
	function especificar_pagina(id_pagina)
	{
	
		this->id_pagina=id_pagina;
	
	}
	//Fin del metodo especificar_pagina
.
.
.
} \end{verbatim*}
\chapter{Modelo de Datos}
\label{modelo_datos}

La capa de persistencia en el SITEM esta soportada en una estructura de base de datos de modelo Relacional. La forma normal básica - 3NF, se garantiza mientras que otras formas normales pueden ser omitidas en casos puntuales donde se demuestre, por pruebas de desempeño, que no seguirlas redunda en una mejora significativa de la velocidad en el acceso a los datos sin detrimento de la calidad e integridad de los mismos. A continuación se listan las formas normales tenidas en cuenta como patrón de diseño en el modelo de datos del SITEM:

\begin{itemize}
\item Primera Forma Normal: Cada renglón-columna contiene valores atómicos.
\item Segunda Forma Normal: 1NF y todo campo que no sea clave primaria depende de los campos clave.
\item Tercera Forma Normal: 2NF y no hay dependencias transitivas.
\item Forma Normal de Boyce-Codd: Todos los determinantes de la tabla son clave candidata.
\item Cuarta Forma Normal:  Una fila no debe contener dos o más campos multi-valorados.
\item Quinta Forma Normal: Una tabla puede almacenar atributos dependientes a la clave sólo por unión.
\end{itemize}

\begin{figure}
 \centering
 \includegraphics[width=156mm, height=182mm]{datos_entidad.png}
 \caption{Arquitectura de datos Subsistema Entidades de Salud}
 \label{mapanavegacion}
\end{figure}
\chapter{Modelo de Despliegue}
\label{modelo_despliegue}

\begin{figure}
 \centering
 \includegraphics[width=141mm, height=122mm]{despliegue.png}
 \caption{Modelo de Despliegue}
 \label{despliegue}
\end{figure}

El SITEM se despliega sobre un servidor que tenga instalado un programa que acepte peticiones usando el protocolo HTTP - Servidor Web, tenga soporte para el interprete de PHP y un servidor de bases de datos MySQL o PostgreSQL. La figura \ref{despliegue} muestra el diagrama de componentes correspondiente a la arquitectura propuesta.

La arquitectura mostrada se reproduce en tres instancias del servidor web, cada una correspondiendo a los niveles de:

\begin{itemize}
\item \textbf{Desarrollo:} Con las versiones más recientes del sistema que se consideran en versión alfa o inestables. Dentro de la codificación son aquellas que muestran incrementos en el cuarto segmento: AA.XX.YY.\textbf{ZZ}
\item \textbf{Prueba:} Versiones que tienen desarrollos estables pero que aún no han pasado la etapa de revisión de casos de prueba. En la codificación son aquellas que muestran un incremento en el tercer segmento: AA.XX.\textbf{YY}.ZZ 
\item \textbf{Producción:} Versiones que han sido probadas y no tienen errores detectados. De acuerdo a la naturaleza de la funcionalidad alcanzada estas versiones representan un incremento en los segmentos primero o segundo dentro de la codificación. 

Si es el primer segmento el que incrementa se alcanza un ciclo de desarrollo y la versión es bautizada con un nombre específico. Para el SITEM se usan los nombre de los Dioses maya que participaron en alguno de los tres intentos por crear la humanidad.
\end{itemize}

\include{encuesta_preliminar}
\chapter{Otros Entregables Nucleares del SITEM}
\label{entregables}

\section{Visión - Resumen Ejecutivo}

\subsection{Propósito}

\textbf{SITEM }es un \textit{Portal Web} especializado en la gestión de datos e información de diferentes componentes estructurales de los sistemas de telemedicina. Provee un ambiente de apoyo a las tareas de las comunidades de práctica involucradas en la investigación, el diseño, mantenimiento, desarrollo e implementación de redes de Telemedicina. Tuvo su génesis conceptual en el año 2000, en la primera fase del Proyecto Telemedicina Bogotá, como solución a la necesidad de administrar los resultados del estudio de campo realizado a las entidades e instituciones de salud y los operadores de Telecomunicaciones en la ciudad de Bogotá.

Su principal objetivo es apoyar las actividades básicas de los denominados \textit{trabajadores del conocimiento} en el área de la telemedicina ofreciendoles, además de un repositorio de datos, herramientas que facilitan las tareas de capturar, extraer, organizar, analizar, encontrar, sintetizar, distribuir y compartir información y conocimiento. El ideal es actualizar el estado de ciertos nodos interesantes del Sistema de Salud de Bogotá Distrito Capital, haciendo énfasis en la posibilidad de interacción a nivel nacional e internacional y en los requerimientos que en Telemedicina tengan las diferentes entidades que participan o no en el proyecto de Telemedicina auspiciado por el grupo GITEM.

\subsection{Alcance}

El SITEM es principalmente \textit{un concepto}, su estado actual es una representación del potencial real del sistema que debe ser socializado y entregado a la comunidad. La base de desarrollo principal es el grupo GITEM y será responsable de la versión oficial del producto. Sin embargo, dada la dinámica en el mundo del software libre, el grupo GITEM no limitará el trabajo independiente que sobre su desarrollo realice cualquier persona o grupo de personas. En este sentido la funcionalidad original del sistema podrá ser modificada pero no avalada directamente por el grupo.\footnote{Salvo en casos en que no se trasgredan directamente los objetivos primarios del desarrollo. En tales casos las contribuciones serán asociadas al hilo oficial de desarrollo.}. 

El SITEM ha sido creado con el fin de apoyar a los grupos de trabajo que realizan labores en el área de proyección de sistemas de Telemedicina. La información que en él se encuentra debe ser ingresada por personas autorizadas para asegurar en un alto grado la veracidad e idoneidad de la misma. Sin embargo no se puede garantizar, y no se garantiza, la exactitud, disponibilidad, integridad y oportunidad de dicha información: LA INFORMACIÓN CONTENIDA EN EL SITEM NO ES UNA FUENTE OFICIAL DE DATOS. El uso de la misma es responsabilidad de quien lo realiza. La información que se encuentre en el SITEM no ha sido necesariamente revisada por expertos profesionales. Todos los contenidos que se ingresen al SITEM deben ser de licencia pública o de libre uso; los contenidos que no cumplan estos criterios serán eliminados.

\subsection{Posicionamiento}

\begin{itemize}
\item \textbf{Definición del problema}

La mayoría de los estudios base de conocimiento se encuentran disgregados y en idiomas diferentes al español por lo cual su consulta es compleja y no existe un mapa seguro de navegación que guíe al investigador hacia las fuentes confiables de información.

\item \textbf{Afecta a}

Investigadores, consultores, usuarios y proveedores de servicios en el área de la salud.

\item \textbf{El impacto asociado es}

Estudios abandonados, e inconclusos, junto con la complejidad innecesaria del proceso de determinación del estado del arte, están abocando a los grupos universitarios a competir codo a codo - a pesar de todas sus limitaciones - contra grandes empresas multinacionales interesadas en “sacar del camino” a estos facilitadores de procesos.

\item \textbf{Una solución parcial adecuada sería}

Un Sistema Informático que en un ambiente integrado ofrezca posibilidades a los usuarios para la administración de información sobre varios componentes tecnológicos de las redes de telemedicina así como la posibilidad de realizar seguimiento al cumplimiento de ciertos indicadores en los proyectos de Telemedicina.

Un sistema que sea fácilmente adaptable a las necesidades novedosas y que este basado en software libre para concentrar la inversión en su desarrollo y no en le pago de licencias de uso o de compra de herramientas de programación.

\item \textbf{Para}

Investigadores, estudiantes, usuarios, prestadores de servicios de salud, prestadores de servicios de telecomunicaciones, programadores.

\item \textbf{Quienes}

Son los beneficiarios directos del despliegue de servicios médicos por la modalidad de Telemedicina.

\item \textbf{Nuestro producto}

Sistema de Información para el Apoyo de Grupos de Trabajo en Proyectos de Telemedicina. SITEM

Disminuye el tiempo de adquisición, análisis y despliegue de la información. Es construido guiado por adaptaciones de procesos de desarrollo ampliamente conocidos y siguiendo el paradigma de la orientación a objetos con lo que se garantiza su facilidad de mantenimiento, escalabilidad e indirectamente su permanencia en el medio.

Contiene módulos para la generación de estadísticas e informes pormenorizados de cada uno de los componentes y logra obtener en unos pocos segundos los datos necesarios para apoyar la labor de análisis, diseño e implementación de proyectos telemédicos o de telesalud. Usa un esquema modular de crecimiento a la medida en donde el esfuerzo para la creación de instrumentos nuevos de consulta se minimiza por el uso de plantillas prediseñadas. En lugar de ser un Sistema estático, SITEM contiene características de adaptación dinámica para cubrir las necesidades que tengan los próximos proyectos emanados del GITEM y otras entidades que hagan uso del sistema.
\end{itemize}

\subsection{Participantes en el Proyecto y Usuarios}

Perfil de los participantes del SITEM. En nuestro desarrollo nos unimos al manifiesto de los metodólogos ágiles manteniendo ciertas pautas del Proceso Unificado para poder dar fe de la calidad en el proceso y el producto:

\begin{table}
\begin{center}
\begin{tabular}{|p{4cm}|p{5cm}|p{5cm}|}
\hline
\textbf{Nombre} & \textbf{Descripción} & \textbf{Responsabilidades}\\
\hline
Director de Proyecto. & Directora Grupo GITEM & Garantiza el flujo de recursos para el desarrollo del proyecto.\\
&&Seguimiento del desarrollo del proyecto.\\
&&Aprueba requisitos y funcionalidades Generales\\
\hline
Arquitectos del Sistema & Se encarga de Definición, modelado del Problema – Arquitectura del Sistema Solución Engloba las funciones de los antiguos analistas, diseñadores e ingenieros de Proceso &Caso de desarrollo aplicando en parte el Proceso Unificado.\\
&&Determinar las necesidades de los usuarios del Sistemas.\\
&&Generar los niveles más altos de requerimientos del sistema.\\
&&Asegurar los criterios de consistencia, pertinencia y completitud del modelo de requisitos.\\
&&Particionar el SITEM en subsistemas y componentes.\\
&&Generar artefactos del modelo de requisitos, análisis y diseño.\\
\hline
Ingenieros de Prueba & Se encargan de desplegar los casos de prueba para garantizar que los ejecutables cumplen con los requisitos de los usuarios.& Diseñar Casos de Prueba\\
&&Realizar pruebas.\\
&&Proponer modificaciones en los componentes.\\
&&Depurar componentes.\\
\hline
Programador& En el SITEM representa el integrante de mayor jerarquía dentro del proceso de desarrollo. Engloba las funciones asociadas a los demás participantes. & Desarrollar componentes\\
\hline
\end{tabular}
\caption{Perfil de los participantes del SITEM.}
\label{participantes_sitem} 
\end{center}
\end{table}

\subsection{Entorno de usuario}

El usuario opera una interfaz web a través de un navegador HTTP, con soporte para HTML 4.0, XML 2.0, javascript, XSL y Cascada Style Sheet 1.0. 

Para acceder a las diferentes secciones del SITEM se requiere que el usuario ejecute un proceso de Autenticación, Autorización y Registro (AAA) – asociados a una sesión. Hasta la versión 3.0 se mantendrá un entorno gráfico básico centrado especialmente en hipervínculos y diseño gráfico mínimo.

\subsection{Suposiciones y dependencias}
\begin{itemize}
\item La plataforma tecnológica sobre la que se implementa el módulo tiene una disponibilidad superior al 99 por ciento del tiempo.
\item Las herramientas de desarrollo son Software Libre.
\item El SITEM podrá integrar en su arquitectura otras aplicaciones de Software Libre o Público.
\item El hilo principal de desarrollo estará en la Universidad Distrital pero no se restringirá la distribución del producto a usuarios interesados.
\item El grupo de participantes en el SITEM es indefinido. Los procesos se potenciaran en la medida que se produzcan “explosiones” de desarrollo fomentadas por usuarios interesados.
\end{itemize}

\subsection{Descripción Global del SITEM}

SITEM es implementado sobre una arquitectura multicapa que distribuye los diferentes componentes en tres capas principales: Presentación, aplicación y datos, estando presente una capa transversal tácita de seguridad. A nivel de usuario el SITEM está compuesto por siete subsistemas autónomos que prestan servicios a sus pares. Estos agrupan seis componentes claves en todo proyecto de telemedicina: entidades de salud, operadores de telecomunicaciones, tecnologías de interconexión, equipos y tecnologías de captura de datos, proyectos e instituciones relacionadas con la telemedicina y servicios médicos - incluyendo módulos de vademécum, consulta de procedimientos, enfermedades y especialidades médicas.

\subsection{Otros Requisitos del Producto}

Estándares Aplicables

\begin{itemize}
\item Unified Process
\item Unified Modeling Language versión 2.0
\item Extensible Markup Language Versión
\item SOAP
\item OWL
\end{itemize}


El sistema debe ser:

\begin{itemize}
\item Multiplataforma.
\item Multiusuario.             
\end{itemize}

Requisitos de Desempeño

\begin{itemize}
\item Velocidad de acceso promedio interior a 10 s.
\item Ayudas contextuales y contenidos autoexplicativos.
\item Disponibilidad superior al 99 por ciento.
\item Manejo de conexiones concurrentes.
\item Integridad referencial en la capa de persistencia.
\end{itemize}

\subsection{Lineamientos de codificación para la organización de los módulos}

Para asegurar una codificación eficiente. que permita realizar búsquedas rápidas dentro de la organización documental, se tienen las siguientes reglas de obligatorio cumplimiento en todos los artefactos:

El nombre del artefacto deberá estar antecedido de un identificador del tipo:

\begin{center}
\textbf{aaa-bbb-ccc}
inicialesmodulo-tipoartefacto-versiónartefacto
\end{center}

Así, para la primera versión del documento de especificaciones de casos de uso, del módulo de administración de instrumentos para la recolección de información ha de tenerse una codificación similar a:

\begin{center}
\textbf{MAI-ECU-001}
\end{center}

Siendo MAI y ECU los identificadores únicos tomados del artefacto Códigos
\chapter{Manual Básico de Usuario}
\label{manual_usuario}
\end{appendices}
\end{document}
