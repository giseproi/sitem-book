\begin{thebibliography}{}

\bibitem[Acero y Ariza,2002] {acero2002} Acero, D. Ariza, M. (2002)\textit{Requerimientos tecnicos y financieros para la implementacion de una red piloto de Telemedicina en el Hospital Simón Bolivar}.Grupo de Investigación en Telemedicina,  Universidad Distrital Francisco José de Caldas.

\bibitem[Alhir,2003] {alhir2003} Alhir, D. (2003) \textit{Understanding the Unified Process.} Methods and Tools, Martinig Associates.

\bibitem[Anderson,2016] {anderson2016} Anderson, D. y Carmichael, A. (2016). \textit{Essential Kanban Condensed}. Seattle, WA: Lean Kanban University Press.

\bibitem[Aparicio-Ramirez,2003] {aparicio2003} Aparicio,L. y Ramírez, J.\textit{Arquitectura de Red de Telemedicina}, Centro de Investigaciones y Desarrollo Científico, Universidad Distrital F.J.C, 2003.

\bibitem[Aparicio,2000] {aparicio2000} Aparicio,L.\textit{Propuesta de Estudio Red de Telemedicina Bogota}, Grupo GITEM, Universidad Distrital F.J.C, 2000.

\bibitem[Ardila,2001] {ardila2001} Ardila,J y Ardila, M.\textit{Evaluación y diagnóstico de los servicios básicos y especializados al servicio de la salud de la clínica San Pedro Claver para el desarrollo de la red de Telemedicina de Bogotá}.Grupo de Investigación en Telemedicina,  Universidad Distrital Francisco José de Caldas.2001.

\bibitem[Balduino,2010] {balduino2010} Balduino, R.\textit{Introduction to OpenUP (Open Unified Process)}.Fundación Eclipse, 2010. Disponible en: https://www.eclipse.org/epf/general/OpenUP.pdf [Accedido el 20 de junio de 2017].

\bibitem[Bashshur,1977] {bashshur77} Bashshur, R., Lovett J. \textit{Assessment of telemedicine: results of the initial experience.} Aviation, Space and Environmental Medicine 1977.

\bibitem[Bashshur,1995] {bashshur95} Bashshur, R. \textit{On the Definition and Evaluation of Telemedicine. Telemedicine Journal.} Volume 1, Number 1, Mary Ann Liebert, Inc., Publishers. 1995.

\bibitem[Beck,1999] {beck1999} Beck, K. \textit{ Extreme Programming Explained: Embrace Change }. Addison-Wesley. 1999.

\bibitem[Benafia,2016] {benafia2016} Benafia, A., Mazouzi, S. y Maanru, R. \textit{From Linguistic to Conceptual: A Framework Based on a Pipeline for Building Ontologies from Texts}. Journal of Advanced Computational Intelligence and Intelligent Informatics. 2016.

\bibitem[Bergeron,2003] {bergeron2003} Bergeron, B. \textit{Essentials of knowledge management.} Jhon Wiley \& Sons, New Jersey, 2003.

\bibitem[Bos,2004] {bos2004} Bos,B.\textit{A proposal for Webapps}. W3C Consortium,2004.

\bibitem[Boggs,2002] {boggs2002} Boggs,W. Boogs, M.\textit{UML with Rational Rose 2002}. SyBex,2002.

\bibitem[BCE,2003] {bce} Britannica Concise Encyclopedia.\textit{Anarchism.}. Obtenido en julio 18, 2003, desde:    http://search.eb.com/ebc/article?eu=380585.

\bibitem[Brugge, Dutoit, 2000] {objectoriented} Brugge, B. y Dutoit, A.~H.\textit{Object-Oriented Software Engineering}. Prentice Hall, 2000.

\bibitem[McCarty, 2006] {carty} Carty, J. \textit{Dynamics of Software Development}.Microsoft Press, 2006.

\bibitem[Cockburn, 1999] {cockburn1999} Cockburn, A. \textit{The Impact of Object-Orientation on Application Development.} IBM Systems Journal 38. Páginas 308-332, 1999.

\bibitem[Ley 23,1982] {congreso23} Congreso de la República. \textit{Ley 23 de 1982: Sobre Derechos de Autor}. República de Colombia, 1982.

\bibitem[Dec 1360,1989] {congreso1360} Congreso de la República. \textit{Decreto 1360 de 1989: Inscripción del soporte lógico (software) en el Registro Nacional del Derecho de Autor}. República de Colombia, 1989.

\bibitem[Ley 44,1993] {congreso44} Congreso de la República. \textit{Ley 44 de 1993: Modifica y adiciona la Ley 23 de 1982 y modifica la Ley 29 de 1944}. República de Colombia, 1993.

\bibitem[Ley 565,2000] {congreso565} Congreso de la República. \textit{Ley 565 de 2000: adopción del Tratado de la OMPI sobre Derechos de Autor}. República de Colombia, 2000.

\bibitem[CRT,QoS,2006] {crtcondiciones} Comisión de Regulación de Telecomunicaciones. \textit{Condiciones de Calidad en Servicios de Telecomunicaciones.} República de Colombia, 2006. 

\bibitem[CRT,Indicadores,2006] {crtindicadores} Comisión de Regulación de Telecomunicaciones. \textit{Indicadores de Calidad en Servicios de Telecomunicaciones.} República de Colombia, 2006. 

\bibitem[CRT,QoS,2007] {crtqos} Comisión de Regulación de Telecomunicaciones.\textit{Proyecto de Resolución para Indicadores de Calidad de Servicio.} República de Colombia, 2007.

\bibitem[Corte1319,2001] {sentencia1319} Corte Constitucional de Colombia.\textit{Sentencia T-1319 de 2001.} República de Colombia, 2001

\bibitem[Craig, 2005] {craig2005} Craig J, Patterson V. \textit{Introduction to the practice of telemedicine.} Journal of Telemedicine and Telecare. 2005. Págs 3-9.

\bibitem[Currie y otros,2014] {currie2014} Currie W. y Seddon J. A cross-national analysis of eHealth in the European Union: some policy and research directions. Inf Manage. 2014.

\bibitem[Davenport,1998] {davenport1998} Davenport, T. y Prusak, L.\textit{Working Knowledge: How Organizations Manage What They Know}, Harvard Business School Press, Boston.1998

\bibitem[Davies,2011] {davies2011}. Davies, M. Knowledge – Explicit, implicit and tacit: Philosophical aspects. , International Encyclopedia of Social and Behavioral Sciences, Elsevier. 2011.

\bibitem[Dueck,2001] {dueck2001} Dueck,G.\textit{Views of knowledge are human views.}IBM Systems Journal, Volumen 40, Número 4, 2001.

\bibitem[Duque \textit{et al},2002] {duque2002} Duque,J. García J, Caicedo, D.\textit{Estudio sobre los requerimientos técnicos y financieros para implementar los servicios telemédicos en el Hospital El Tunal,  para el proyecto telemedicina Bogotá 2000.}.Grupo de Investigación en Telemedicina,  Universidad Distrital Francisco José de Caldas.2002.

\bibitem[UN, 2005] {egovernment} Department of Economic and Social Affairs.\textit{UN Global E-government Readiness Report 2005 From e-government to e-inclusion}. Naciones Unidas, New York, 2005.

\bibitem[erl, 2017] {earl2017} Earl, T. (2017).\textit{Service-Oriented Architecture: Analysis and Design for Services and Microservices}. 2da Edición. Prentice Hall.

\bibitem[Firestone,2001] {firestone2001} Firestone, J.\textit{Key Issues in Knowledge Management. Knowledge and Innovation.} Knowledge Management Consortium International. Volumen 1. 2001.

\bibitem[Gang,2007] {gang2007} Gang, L. y Yi, L. \textit{A Relational Model of Knowledge Share, Knowledge Acquisition and Product Innovation}. Universidad Xi'an Jiaotong. China. 2007.

\bibitem[Girard,2015] {girard2015} Girard, J. \textit{Defining knowledge management: Toward an applied compendium}. Online Journal of Applied Knowledge Management. International Institute for Applied Knowledge Management. 2015.

\bibitem[González y Torres,2002] {gonzalez2002} González, O. Torres, J.\textit{Evaluación y Diagnóstico de los Servicios Básicos y Especializados al Servicio de la Salud del Hospital de San José para el Desarrollo de la Red de Telemedicina de Bogotá.} Grupo de Investigación en Telemedicina, Universidad Distrital Francisco José de Caldas.2002.

\bibitem[Guarin \textit{et al},2003] {guarin2003} Guarin, S. Garcia, M. Torres, L.\textit{Diagnóstico de las Redes Eléctrica, Telefónica y de Datos del Hospital Rafael Uribe Uribe E.S.E.} Grupo de Investigación en Telemedicina,  Universidad Distrital Francisco José de Caldas.2003.

\bibitem[Hurtado,2000] {hurtado2000} Hurtado,J.\textit{Metodología de la Investigación Holística.} Sypal, Caracas, Venezuela, 2000.

\bibitem[Hibbard,1997] {hibbard1997} Hibbard, J. \textit{Knowledge management—knowing what we know.} Information Week, Edición Octubre de 1997.

\bibitem[IEEE, 1990] {softwareengineering} IEEE Institute.\textit{IEEE Standard Glossary of Software Engineering Terminology - IEEE std 610.12-1990}. IEEE, 1990.

\bibitem[Koch, 2005] {koch} Koch, S. \textit{Free Open Source Software Development}. Idea Group, 2005.

\bibitem[Liebowitz,1998] {liebowitz1998} Liebowitz, J. y Beckman, T. \textit{Knowledge Organizations: What Every Manager Should Know}. Boca Raton, St. Luci press, 1998.

\bibitem[ISO,2011] {ISO42010} International Organization for Standardization. (2011) \textit{ISO/IEC/IEEE 42010:2011 Systems and software engineering -- Architecture description}. 

\bibitem[ITU-T, 2004] {ITU2004} ITU-T. \textit{Manual Calidad de servicio y calidad de funcionamiento de la red}. Unión Internacional de Telecomunicaciones, 2004.

\bibitem[ITU-OMS, 2012] {ituoms2012} ITU - OMS. \textit{National eHealth Strategy Toolkit}. Unión Internacional de Telecomunicaciones. 2012 

\bibitem[G1000, 2001] {ITUG1000} ITU-T. \textit{Recomendación G.1000 - Calidad de servicio en las comunicaciones: Marco y definiciones}. Unión Internacional de Telecomunicaciones, 2001.

\bibitem[G1010, 2001] {ITUG1010} ITU-T. \textit{Recomendación G.1010 - Categorías de calidad de servicio para los usuarios de extremo de servicios multimedios}. Unión Internacional de Telecomunicaciones, 2001.

\bibitem[Jacobson,2000] {jacobson2000} Jacobson,I. Booch,G y Rumbaugh,J. \textit{El Proceso Unificado de Desarrollo de Software}. Addison-Wesley, Madrid, 2000.

\bibitem[Jacobson,2005] {jacobson2005} Jacobson,I. Booch,G y Rumbaugh,J. \textit{The Unified Modeling Language Reference Manual}, segunda edición. Addison-Wesley, Boston, 2005.

\bibitem[Jackson,2005] {jackson2005} Jackson,D.\textit{The W3C Workshop on Web Applications and Compound Documents}. W3C Consortium,2005.

\bibitem[Larman,2004] {larman2004} Larman,C. \textit{Agile and iterative development: a manager ’s guide.} Addison Wesley, 2004.

\bibitem[Larman,2003] {larman2003} Larman, C. \textit{UML y PATRONES. Una introducción al análisis y diseño orientado a objetos y al Proceso Unificado}. Pearson Educación S.A. Madrid, 2003.

\bibitem[Lewis, 2012] {lewis2012} Lewis T, Synowiec C, Lagomarsino G y Schweitzer J. E-health in low- and middle-income countries: findings from the Center for Health Market Innovations. Bull World Health Organ. 2012

\bibitem[Martínez,1998] {martinez1998} Martínez, R. Martín, F. \textit{LANDSCAPE: A Knowledge-Based System for Visual Landscape Assessment.}. IEA/AIE, Volumen 2. Springer, 1998. Páginas 849-856.

\bibitem[CONPES3072,2000] {mincomunicaciones3072} Ministerio de Comunicaciones.\textit{Documento CONPES 3072 - Agenda de Conectividad}. República de Colombia, 2000.

\bibitem[MINSALUD,2016] {minsalud2016} Ministerio de Salud y Protección Social.\textit{Estudio Exploratorio de la Situación de la Telemedicina en Municipios Priorizados - Colombia}. República de Colombia, 2016.

\bibitem[MINSALUD-4678,2015] {minsalud4678} Ministerio de Salud.\textit{Resolución Número 4678 de 2015}. República de Colombia, 2015.

\bibitem[MINSALUD-1448,2006] {minsalud1448} Ministerio de Salud.\textit{Resolución Número 1448 de 2006}. República de Colombia, 2006.

\bibitem[Barrero,2000] {barrero2000} Muñoz, F. Barrero, J.\textit{Evaluación y Diagnóstico de los Servicios Básicos y Especializados al Servicio de la Salud del Hospital La Victoria para el Desarrollo de la Red de Telemedicina de Bogotá}. Grupo de Investigación en Telemedicina,  Universidad Distrital Francisco José de Caldas.2000.

\bibitem[Nokata,1994] {nokata1994} Nonaka, I.\textit{A dynamic theory of organizational knowledge creation.} Organization Science,5,1994. pp. 14-37.

\bibitem[Nokata,1995] {nokata1995} Nonaka,I. y Takeuchi, H.\textit{The Knowledge Creating Company: How Japanese Companies Create the Dynamics of Innovation.} Oxford University Press,1995.

\bibitem[OAVES, 2017] {oaves2017} Observatorio Así Vamos en Salud. \textit{Tendencias de la Salud en Colombia. Informe Anual 2016}. 2017

\bibitem[OCDE, 2015] {ocde2015} Organización para la Cooperación y el Desarrollo Económico. \textit{OECD Reviews of Health Systems: Colombia 2016}. OECD Publishing, Paris. 2015.

\bibitem[OMG, 2007] {omg2007} OMG.\textit{Unified Modeling Language: Specification. Versión 2.1.1}. Object Management Group, 2007.

\bibitem[OMG, 2007a] {omg2007a} OMG.\textit{Unified Modeling Language: Superstructure. Versión 2.1.1}. Object Management Group, 2007.

\bibitem[OMS, 2010] {oms2010} Organización Mundial de la salud. \textit{Telemedicine: opportunities and developments in Member States: report on the second global survey on eHealth 2009}. 2010

\bibitem[OMS, 2016] {oms2016} Organización Mundial de la salud. \textit{Global difusion of eHealth: making universal health coverage achievable. Report of the third global survey on eHealth}. 2016

\bibitem[OPS, 2011]{ops2011}. Organización Panamericana de la Salud. \textit{Estrategia y Plan de Acción sobre eSalud}. 51 Consejo Directivo, 2011

\bibitem[Pressman, 2006] {pressman} Pressman, R.~J.\textit{Ingeniería del Software}. Sexta Edición. Mc Graw 
Hill, México, 2006.

\bibitem[Raymond, 1996] {raymond} Raymond,E.\textit{The Cathedral and the Bazaar}, Revision 1.57, 2000.

\bibitem[AIM,1993] {aim} \textit{Research and technology development on telematics systems in health care: AIM 1993; Annual Technical Report on RTD: Health Care.} Comisión Europea: Dirección General XIII, 1993.

\bibitem[Ross y Otros,2016] {ross2016} \textit{Factors that influence the implementation of e-health: a systematic review of systematic reviews.} Research Department of Primary Care and Population Health, University College London. 2016.

\bibitem[Rozo \textit{et al},2002] {rozo2002} Rozo, O. Valencia, S. Barahona, F.\textit{Estudio diagnóstico de las condiciones técnicas y financieras en instrumentos y equipos médicos  y de servicios de la clínica San Pedro Claver para la implentación de los servicios telemédicos.} Grupo de Investigación en Telemedicina,  Universidad Distrital Francisco José de Caldas.2003.

\bibitem[Ruiz y Niño,2002] {ruiz2002} Ruiz, M. Niño, D.\textit{Evaluacion y Diagnostico de los Servicios Basicos y Especializados al Servicio de la Salud del Hospital El Tunal para el Desarrollo de la Red de Telemedicina de Bogotá} Grupo de Investigación en Telemedicina,  Universidad Distrital Francisco José de Caldas.2002.

\bibitem[Salazar,2002] {oas2002} Salazár,J. y Kopec, A.\textit{Aplicaciones de Telecomunicaciones en Salud en la Subregion Andina - Telemedicina}, Organismo Andino de Salud, OPS. 2002.

\bibitem[Sarmento,2005] {sarmento2005} Sarmento,A.\textit{Issues of human computer interaction}.IRM Press, Londres, 2005. 

\bibitem[Sowa,1984] {sowa1984} Sowa, J. \textit{Conceptual Structures: Information Processing in Mind and Machine.} Addison-Wesley,1984.

\bibitem[stallman,2002] {stallman2002} Stallman,R.\textit{Free Software, Free Society:Selected Essays of Richard M. Stallman}. GNU Press, Bosotón, 2002.

\bibitem[Tatnall, 2003] {tatnall2005} Tatnall,A. \textit{Web portals:The New Gateways to Internet Information and Services.} Idea Group, Londres, 2005.

\bibitem[BDT,1999] {itu} \textit{Telemedicine And Developing Countries - Lessons Learned}. Document 2/116-E. ITU-D STUDY GROUPS. Question 14/2: Fostering the application of telecommunication in health care.  Identifying and documenting success factors for implementing telemedicine. 1999.

\bibitem[Bangemann,1994] {bangeman} \textit{The Bangemann Report: Europe and the global Information Society. Recomendaciones al Consejo Europeo.} Bruselas, 1994. Disponible en http://www.cordis.lu.

\bibitem[theopengroup,2016] {theopengroup2016} The Open Group (2016). \textit{ArchiMate 3.0.1 Specification}, Van Haren Publishing.

\bibitem[Turban,1992] {turban1992} Turban, E.\textit{Decision Support and Expert Systems - Management Support Systems.} Collier Macmillan, Sydney,1992.

\bibitem[Wiig,1993] {wiig1993} Wiig,K.\textit{Knowledge Management Foundations}.Schema Press, Arlington, 1993. 

\bibitem[Wielinga \textit{et al}, 1992] {wielinga1992} Wielinga, B. \textit{et al}. \textit{KADS: A Modelling Approach to Knowledge Engineering.}Knowledge Acquisition Journal, 4(1). Páginas 5-53.

\bibitem[Wilson, 2017] {wilson2017} Wilson, R., (2017). \textit{Successful Digital Health Systems: Guidelines for Healthcare Leaders and Clinicians}. Memorias de e-Health 2017 Virtual Meeting. Canada.

\bibitem[11] {yellowlees} Yellowlees PM.\textit{Successfully developing a telemedicine system.} Journal of Telemedicine and Telecare, 2005. Págs:331-335.

\bibitem[Zabala, 2000] {Zavala2000} Zavala R.\textit{Diseño de un Sistema de Información Geográfica sobre internet.} Tesis de Maestría en Ciencias de la Computación. Universidad Autónoma Metropolitana-Azcapotzalco. México, D.F. 2000.

\end{thebibliography}
