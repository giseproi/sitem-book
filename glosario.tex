\chapter*{Glosario}
\label{glosario}
ABR - (Available Bit Rate – Rata de Bits Variable). Característica de servicio ATM mediante el cual la tasa de transferencia puede variar dependiendo de las condiciones de la conexión.

ADMINISTRAR - Dirigir, Disponer y Organizar el uso de algo, para obtener mayor rendimiento de ello o para que produzca mejor efecto

ADSL - (Asymetric Digital Suscriber Line). Línea digital asimétrica del suscriptor. Tecnología que permite la transmisión de información digital a altas velocidades de descarga utilizando medios de transmisión convencionales.

ALGORITMO - Conjunto de reglas o procedimientos para la resolución de un problema. Un programa de software es la transcripción de un algoritmo.

ANCHO DE BANDA - Cantidad de datos que pueden transmitirse en determinado periodo de tiempo por un canal de transmisión.

ANTÍTESIS - Oposición de sentido entre dos proposiciones.

APACHE - Servidor Web de distribución libre,  poderoso y flexible. Implementa los últimos protocolos HTTP. Altamente configurable y extensible con módulos de terceros. Fue desarrollado en 1955 y ha llegado a ser el más usado de Internet.

API - (Application Program Interface). Interfaz de programación de Aplicaciones.

APRENDIZAJE POR COMPUTADOR - Hace referencia al uso de computadores como parte clave en la enseñanza y el aprendizaje haciéndolo parte integrante del entorno educativo.

APRENDIZAJE EN LINEA - Hace referencia a aquel tipo de aprendizaje que se lleva a cabo mediante la utilización de una red o INTERNET.

ARTEFACTO - Cualquier tipo de información que es producida o usada en el proceso de desarrollo de software.

ASIGNAR - Nombrar, designar a alguien o algo para un oficio o función.

ATM - (Asynchronous Transfer Mode - Modo de Transferencia Asincrónica ). Estándar internacional para la retransmisión de celdas en la que múltiples tipos de servicios (como la voz, el video o los datos) son transportados en celdas de longitud fija (53 bytes). Las celdas de longitud fija permiten que tenga lugar el procesamiento de celdas en el hardware reduciendo así los retrasos de transito.

BASE DE DATOS - Aplicación informática que permite la gestión de datos y el manejo de la información.

BORRAR - Quitar, hacer que desaparezca algo.

CALL (Computer-assisted language learning) - Es un pseudo lenguaje que permite aproximar la enseñanza y aprendizaje en el que tecnología computacional es usada ya sea como presentación, refuerzo y acceso a material para aprender.  Usualmente incluye elementos interactivos y contenidos multimedia.


CASO DE USO – Elemento el cual define la funcionalidad de un sistema.

CBR  -  (Constant Bit Rate). Tasa de bits constante.

CGI - (Common Gateway Interface) Estándar para tender interfaces entre aplicaciones externas y servidores de información, tales como los servidores HTTP.

CONCEPCIÓN – Idea, formación de un pensamiento. Dar existencia a un proyecto.

COOKIE -  Pequeño archivo de texto que se almacena en el disco duro al visitar una pagina Web y que sirve para identificar al usuario cuando se conecta de nuevo a dicha página.

CLASE -  En el desarrollo de software es la representación mediante un nombre, atributos y operaciones de un clase o conjunto de objetos que pueden ser reales o abstractos.

CREAR - Producir o instituir por vez primera algo.

CRUD - Caso CRUD, caso utilizado en lenguajes de modulado de software que agrupa los casos de crear, leer, actualizar y borrar.

CSS -  (Cascading Style Sheet - Hojas en estilo de cascada). Permiten adjuntar estilos personalizados a los documentos HTML como colores, tamaños y tipos de fuente, clase espaciado o colores de fondo.

DAEMON - (Disk And Execution MONitor) Un programa no invocado explícitamente pero que permanece esperando a que una situación especial ocurra  Los sistemas Unix ejecutan muchos daemons, principalmente para manipular solicitudes de servicios desde otro host o desde una red.

DBMS -  (Data Base Management System – Sistema de Gestión de Bases de Datos). Sistema de software que permite a los usuarios guardar y modificar información.

DEPURAR -  Relacionado con el software, detectar, localizar, corregir los problemas en un programa informático.

DESARROLLO ITERATIVO - Método de construcción de productos cuyo ciclo de vida está compuesto por un conjunto de iteraciones, las cuales tienen como objetivo entregar versiones del software.

DISCIPLINA - Conjunto de reglas para mantener el orden y la subordinación entre miembros de un cuerpo.

DOMINIO - Parte del nombre jerárquico con que se conoce a cada entidad conectada a Internet. Un dominio se compone de una serie de etiquetas o nombres separados por puntos.

DSL -  (Línea digital de abonado). Tecnología de red pública que proporciona ancho de banda amplio sobre el cableado tradicional de cobre en distancias limitadas. Hay cuatro tipos de DSL: ADSL, HDSL, SDSLY VDSL. Todos ellos se aprovisionan a través de pares de módems, estando un módem situado en al oficina central y el otro en el domicilio del abonado.

EDITAR - Modificar datos existentes.

ELML (ELESSON MARKUP LANGUAGE) - Es una estructura XML de fuente abierta para la creación de cursos electrónicos, fue lanzado en el 2004 y contiene herramientas como CVS, foros y bugtraker.

ENTORNO DE APRENDIZAJE - Un entorno de aprendizaje es el conjunto de conocimientos, herramientas de enseñanza y aprendizaje, espacios y personas involucradas en un proceso de aprendizaje dado.

ENTORNO VIRTUAL DE APRENDIZAJE - Un entorno virtual de aprendizaje (virtual learning enviroment) es un paquete de software cuya función es reemplazar o complementar un entorno de aprendizaje formal y tradicional.


FILTRO - Cualquier función o programa que seleccione información automáticamente con un criterio preestablecido.

FTP - (File Transfer Protocol  - Protocolo de transferencia de archivos). Sistema, servicio o protocolo que transfiere programas o archivos entre servidor y cliente.

GENERAR - Producir, causar algo.

GESTIONAR - Hacer diligencias para el logro de un negocio o de un deseo cualquiera.

GNU – (GNU’s Not Unix). Sistema operativo, compuesto de pequeñas piezas individuales de software totalmente libre.

HIPERTEXTO - Método de preparar y publicar un texto ideal para el computador, con el que los lectores pueden escoger su propia ruta a través del material. La información se descompone en pequeñas unidades y después los hipervínculos se insertan en el texto.

HIPERVINCULO - Enlace. En un sistema de hipertexto, palabra subrayada o destacada que, cuando se pulsa en ella lleva a otro documento.

HIPÓTESIS - Entiéndase por hipótesis un enunciado (o un conjunto de enunciados) que precede a otros enunciados y constituye su fundamento.

HOME PAGE -  Pagina principal, sitio o pantallazo de entrada que se puede ver cuando se llega a una dirección o sitio Web.

HTML - (Hyper text markup language – Lenguaje de marcado de hipertexto). Lenguaje especial con el que se construyen las paginas Web, permite saltar de un documento a otro o de una parte del documento a otra, únicamente haciendo clic sobre las letras resaltadas, iconos, imágenes o figuras.

HTTP -  (Hyper Text Transfer Protocol - Protocolo de Transferencia de Archivos). Permite interactuar con el usuario, tal como si fuera un programa de computadora que se ejecutara en red.

INCREMENTAL - Expandible que puede instalarse por fases, cada una de ellas añadiendo nuevas facilidades.

INFRAESTRUCTURA - Conjunto de medios necesarios para el desarrollo de una actividad. 

INGENIERÍA DE SOFTWARE - Conjunto de etapas parcialmente ordenadas con la intención de lograr un objetivo, en este caso, la obtención de un producto de software de calidad.

INGENIERÍA INVERSA – Conjunto de etapas desarrolladas para obtener las bases del diseño y la forma de implementación de algún producto de ingeniería a partir de su estado final.

INGRESAR - Ser admitido o tener entrada en alguna parte o algún proceso..

INTERCONEXIÓN - Unión o conexión entre sí de dos o más elementos.

INTERFAZ - Parte de un sistema interactivo en la que se desarrolla la comunicación o interacción entre el usuario y el programa.
INTEROPERABILIDAD - Característica de los computadores que hace compatible su interconexión y funcionamiento.

IP – (Internet Protocol - Protocolo Internet). Protocolo de la capa de red de la pila TCP / IP que ofrece un servicio para intertrabajo sin conexiones.
ITERACION - Repetición de un hecho con el fin de mejorarlo.

Kbps -  Kilobits por segundo.

KBps -  KiloBytes por segundo.

LEER - Descifrar un código de signos.

MANEJO DE SESIONES - Permiten a la aplicación conocer el lugar exacto donde se encuentra el usuario, luego que ha ingresado al sitio Web de la aplicación y evitarle la molestia de realizar este proceso repetidas veces.

MÉTODO - Modo ordenado de proceder para llegar a un resultado o fin determinado. 
METODOLOGÍA - Parte de la lógica que estudia los métodos.

MITIGAR - Moderar, suavizar.

MODELADO – Descripción, en un lenguaje de máquina, de la forma o características de un objeto o conjunto de objetos.

MÓDULO - Una aplicación software que presenta una funcionalidad específica dentro de SITEM. En el contexto del proyecto es sinónimo de subsistema.

MUDS - Metodología Unificada para el Desarrollo de Software.

MULTIPLATAFORMA -  Las aplicaciones pueden ser vistas en cualquier computador del mundo a través de múltiples plataformas (sistemas operativos, navegadores y hardware).

MySQL -  Sistema de administración para bases de datos relacionales (rdbms) que provee una solución robusta a los usuarios con poderosas herramientas multi-usuario, soluciones de base de datos SQL. Es rápido, robusto y fácil de utilizar.

NRT-VBR - (Non-Real-Time Variable Bit Rate). Tasa de bits variable en tiempo no real, Esta categoría de servicio está encaminada para aplicaciones en tiempo no real, las cuales tienen características de tráfico en ráfaga.

NTP - (Network Time Protocol – Protocolo de Tiempo de red) - Sistema de sincronización de tiempo para relojes de computadores a través de Internet.

OBJETO -  Un objeto es una instancia, un elemento único que puede representar o nacer de una o varias clases.

OMG - Object Management Group.

OSI - Internetworking de sistemas abiertos. Programa internacional de normalización creado por ISO e ITU-T. Para desarrollar estándares para las redes de datos que faciliten la interoperabilidad entre equipos de varios fabricantes.

PÁGINAS DINÁMICAS - Paginas Web que incluyen además de texto y gráficos, archivos de sonido y de video animaciones o elementos interactivos.

PASSWORDS - Contraseña. Método de seguridad empleado para identificar a un usuario. Sirve para dar acceso a personas con determinados permisos.

PCR - (Peak Cell Rate). Tasa pico de celda.

PHP - (Hypertext Preprocessor"). Es un lenguaje interpretado de alto nivel embebido en páginas HTML y ejecutado en el servidor. La meta del lenguaje es permitir rápidamente a los desarrolladores la generación dinámica de páginas.

PLANTILLAS -  Son una serie de archivos que definen la apariencia de una aplicación Web, y que permiten cambiar la presentación de la aplicación con sólo modificar unos cuantos archivos, sin necesidad de modificar el código que le da funcionalidad a la aplicación.

PLATAFORMA - Base o elemento de apoyo. Se refiere a una combinación específica de hardware y sistema operativo y/o compilador.

PRI – (Primary-Rate Interface - Interfaz de acceso principal). Interfaz RDSI de acceso principal. En acceso consta de un solo canal D de 64 Kbps, mas 23 (T1) o 30 (E1) canales B para voz o datos.

PROCESO - Pluralidad de actos encadenados entre sí, de modo que los actos anteriores justifican y son requisitos de validez de los posteriores.

PROTOCOLO - Conjunto de instrucciones o lenguaje que utilizan los computadores para comunicarse entre sí.

PROTOTIPO - Original ejemplar o primer molde en que se fabrica una figura u otra cosa. Modelo original que sirve como ejemplo para futuros estados o formas.

QoS – (Quality Of Service - Calidad de servicio). Una medida de rendimiento en el sistema de transmisión que refleja la calidad de la transmisión y la disponibilidad del servicio.

RDSI – (Red digital de servicios integrados). Protocolo de comunicaciones que permite a las redes de las compañías telefónicas transportar datos, voz y otro trafico.

REGISTRO - Pequeña unidad de almacenamiento que contiene cierto tipo de datos. En el ámbito de las bases de datos es cada una de las fichas que forman un conjunto.

RT-VBR - (Real-Time Variable Bit Rate). Está encaminada a aplicaciones que requieren restricción en la variación de retardo (lo mínimo) y variación de retardo, que podrían ser apropiados para voz y vídeo en tiempo real.

SCRIPT -  Pequeños archivos de texto embebidos dentro del documento, que ofrecen una forma de extender los documentos HTML convirtiendo el contenido del documento en una página dinámica, pudiendo ser ejecutados al presentarse un evento como el movimiento del ratón.

SCORM (Sharable Content Object Reference Model) - Es una colección de normas y especificaciones para aprendizaje basado en la Web.  Define como debe ser la comunicación entre el contenido de lado al cliente y un sistema servidor, tratando aspectos como el empaquetamiento de información en archivos ZIP.

SERVIDOR - Dispositivos de un sistema que resuelve las peticiones de información de otros elementos del sistema a los que se denomina clientes.

SÍNTESIS - Composición de un todo por la reunión de sus partes.

SISTEMA DE INFORMACIÓN - Conjunto de componentes interrelacionados que permiten capturar, procesar, almacenar y distribuir la información para apoyar la toma de decisiones y el control en una institución.

SITEM - Sistema de Información en Telemedicina, basado en aplicaciones Web, su fin es el de gestionar eficientemente la información referente al proyecto de Telemedicina Bogotá.

SQL - (Structured Query Language Lenguaje Estructurado de Consultas). Lenguaje de programación interactivo y estándar para obtener y actualizar información de una base de datos.

TASA DE BITS  - Velocidad en Kilobits por segundo a la que puede transmitir un circuito virtual.

TCP/IP - TCP("Transmisión Control Protocol") e IP("Internet Protocol"). Protocolo para el control de la transmisión / Protocolo Internet. Nombre común que se da al paquete de protocolos desarrollados  por el DoD en los años 70’s con el fin de soportar la construcción de interworks a escala mundial.

TECNOLOGÍA - Conjunto de los conocimientos técnicos. Conjunto de los instrumentos y procedimientos industriales de un determinado producto.

TELEASISTENCIA - Los sistemas de teleasistencia permiten a los pacientes recoger datos acerca de su enfermedad, transmitir la información a un lugar remoto y usar la videoconferencia para discutir sus padecimientos y tratamientos con el profesional de la salud.

TELECONFERENCIA - Es la telecomunicación que se establece entre dos o más especialistas, para tratar diversos temas que en el caso de la Telemedicina, generalmente tienen que ver con los pacientes o  están relacionados con capacitación especializada.


TELEMEDICINA - Herramienta tecnológica para brindar servicios médicos a distancia mediante el intercambio de imágenes, voz, datos y video, haciendo uso de las tecnologías de la información y de las comunicaciones, permitiendo el diagnóstico, la opinión de casos clínicos y la provisión de cuidados de salud y educación médica.

TRANSICIÓN - Tiempo en el cual se cambia de un estado a otro.

UBR - (Unspecified Bit Rate). Esta categoría está encaminada para aplicaciones en tiempo no real, por ejemplo aquellas que no requieren restricciones en la variación de retardo (mínimo) y variaciones de retardo.

UML - (Unified Modeling Language - Lenguaje Unificado de Modelado). Brinda todas las características, tanto sintácticas como semánticas, para lograr caracterizar lógicamente cualquier tipo de software permitiendo ser utilizado en cualquier etapa del diseño y especialmente útil en aquellos desarrollos enfocado a objetos.

UP - Proceso Unificado. El Proceso Unificado puede concebirse como una idea, una plantilla que provee una infraestructura para ejecutar proyectos pero que no tiene en cuenta todos los detalles requeridos en dicha ejecución.

USUARIO - Dicho de una persona: Que tiene derecho de usar de una cosa con cierta limitación.

VIABILIDAD - Que puede llevarse a cabo.

VIDEOCONFERENCIA - Conferencia efectuada utilizando el teléfono y la televisión combinados en un aparato que transmite a la a vez señales acústicas o visuales.

WEB - Sistema de computación que ejecuta un servidor en red y que se ha configurado para publicar documentos en Internet.
WWW - (World Wide Web). Sistema mundial de hipertexto que utiliza Internet como mecanismote transporte. Conjunto de páginas de información con texto, imágenes sonido, etc., enlazadas entre sí.

XML -  (Extended Markup Language). Lenguaje de marcado extendido. Es una
versión extendida del HTML menos rígida, en la cual el desarrollador de software coloca sus condiciones para mostrar el texto.
