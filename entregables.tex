\chapter{Otros Entregables Nucleares del SITEM}
\label{entregables}

\section{Visión - Resumen Ejecutivo}

\subsection{Propósito}

\textbf{SITEM }es un \textit{Portal Web} especializado en la gestión de datos e información de diferentes componentes estructurales de los sistemas de telemedicina. Provee un ambiente de apoyo a las tareas de las comunidades de práctica involucradas en la investigación, el diseño, mantenimiento, desarrollo e implementación de redes de Telemedicina. Tuvo su génesis conceptual en el año 2000, en la primera fase del Proyecto Telemedicina Bogotá, como solución a la necesidad de administrar los resultados del estudio de campo realizado a las entidades e instituciones de salud y los operadores de Telecomunicaciones en la ciudad de Bogotá.

Su principal objetivo es apoyar las actividades básicas de los denominados \textit{trabajadores del conocimiento} en el área de la telemedicina ofreciendoles, además de un repositorio de datos, herramientas que facilitan las tareas de capturar, extraer, organizar, analizar, encontrar, sintetizar, distribuir y compartir información y conocimiento. El ideal es actualizar el estado de ciertos nodos interesantes del Sistema de Salud de Bogotá Distrito Capital, haciendo énfasis en la posibilidad de interacción a nivel nacional e internacional y en los requerimientos que en Telemedicina tengan las diferentes entidades que participan o no en el proyecto de Telemedicina auspiciado por el grupo GITEM.

\subsection{Alcance}

El SITEM es principalmente \textit{un concepto}, su estado actual es una representación del potencial real del sistema que debe ser socializado y entregado a la comunidad. La base de desarrollo principal es el grupo GITEM y será responsable de la versión oficial del producto. Sin embargo, dada la dinámica en el mundo del software libre, el grupo GITEM no limitará el trabajo independiente que sobre su desarrollo realice cualquier persona o grupo de personas. En este sentido la funcionalidad original del sistema podrá ser modificada pero no avalada directamente por el grupo.\footnote{Salvo en casos en que no se trasgredan directamente los objetivos primarios del desarrollo. En tales casos las contribuciones serán asociadas al hilo oficial de desarrollo.}. 

El SITEM ha sido creado con el fin de apoyar a los grupos de trabajo que realizan labores en el área de proyección de sistemas de Telemedicina. La información que en él se encuentra debe ser ingresada por personas autorizadas para asegurar en un alto grado la veracidad e idoneidad de la misma. Sin embargo no se puede garantizar, y no se garantiza, la exactitud, disponibilidad, integridad y oportunidad de dicha información: LA INFORMACIÓN CONTENIDA EN EL SITEM NO ES UNA FUENTE OFICIAL DE DATOS. El uso de la misma es responsabilidad de quien lo realiza. La información que se encuentre en el SITEM no ha sido necesariamente revisada por expertos profesionales. Todos los contenidos que se ingresen al SITEM deben ser de licencia pública o de libre uso; los contenidos que no cumplan estos criterios serán eliminados.

\subsection{Posicionamiento}

\begin{itemize}
\item \textbf{Definición del problema}

La mayoría de los estudios base de conocimiento se encuentran disgregados y en idiomas diferentes al español por lo cual su consulta es compleja y no existe un mapa seguro de navegación que guíe al investigador hacia las fuentes confiables de información.

\item \textbf{Afecta a}

Investigadores, consultores, usuarios y proveedores de servicios en el área de la salud.

\item \textbf{El impacto asociado es}

Estudios abandonados, e inconclusos, junto con la complejidad innecesaria del proceso de determinación del estado del arte, están abocando a los grupos universitarios a competir codo a codo - a pesar de todas sus limitaciones - contra grandes empresas multinacionales interesadas en “sacar del camino” a estos facilitadores de procesos.

\item \textbf{Una solución parcial adecuada sería}

Un Sistema Informático que en un ambiente integrado ofrezca posibilidades a los usuarios para la administración de información sobre varios componentes tecnológicos de las redes de telemedicina así como la posibilidad de realizar seguimiento al cumplimiento de ciertos indicadores en los proyectos de Telemedicina.

Un sistema que sea fácilmente adaptable a las necesidades novedosas y que este basado en software libre para concentrar la inversión en su desarrollo y no en le pago de licencias de uso o de compra de herramientas de programación.

\item \textbf{Para}

Investigadores, estudiantes, usuarios, prestadores de servicios de salud, prestadores de servicios de telecomunicaciones, programadores.

\item \textbf{Quienes}

Son los beneficiarios directos del despliegue de servicios médicos por la modalidad de Telemedicina.

\item \textbf{Nuestro producto}

Sistema de Información para el Apoyo de Grupos de Trabajo en Proyectos de Telemedicina. SITEM

Disminuye el tiempo de adquisición, análisis y despliegue de la información. Es construido guiado por adaptaciones de procesos de desarrollo ampliamente conocidos y siguiendo el paradigma de la orientación a objetos con lo que se garantiza su facilidad de mantenimiento, escalabilidad e indirectamente su permanencia en el medio.

Contiene módulos para la generación de estadísticas e informes pormenorizados de cada uno de los componentes y logra obtener en unos pocos segundos los datos necesarios para apoyar la labor de análisis, diseño e implementación de proyectos telemédicos o de telesalud. Usa un esquema modular de crecimiento a la medida en donde el esfuerzo para la creación de instrumentos nuevos de consulta se minimiza por el uso de plantillas prediseñadas. En lugar de ser un Sistema estático, SITEM contiene características de adaptación dinámica para cubrir las necesidades que tengan los próximos proyectos emanados del GITEM y otras entidades que hagan uso del sistema.
\end{itemize}

\subsection{Participantes en el Proyecto y Usuarios}

Perfil de los participantes del SITEM. En nuestro desarrollo nos unimos al manifiesto de los metodólogos ágiles manteniendo ciertas pautas del Proceso Unificado para poder dar fe de la calidad en el proceso y el producto:

\begin{table}
\begin{center}
\begin{tabular}{|p{4cm}|p{5cm}|p{5cm}|}
\hline
\textbf{Nombre} & \textbf{Descripción} & \textbf{Responsabilidades}\\
\hline
Director de Proyecto. & Directora Grupo GITEM & Garantiza el flujo de recursos para el desarrollo del proyecto.\\
&&Seguimiento del desarrollo del proyecto.\\
&&Aprueba requisitos y funcionalidades Generales\\
\hline
Arquitectos del Sistema & Se encarga de Definición, modelado del Problema – Arquitectura del Sistema Solución Engloba las funciones de los antiguos analistas, diseñadores e ingenieros de Proceso &Caso de desarrollo aplicando en parte el Proceso Unificado.\\
&&Determinar las necesidades de los usuarios del Sistemas.\\
&&Generar los niveles más altos de requerimientos del sistema.\\
&&Asegurar los criterios de consistencia, pertinencia y completitud del modelo de requisitos.\\
&&Particionar el SITEM en subsistemas y componentes.\\
&&Generar artefactos del modelo de requisitos, análisis y diseño.\\
\hline
Ingenieros de Prueba & Se encargan de desplegar los casos de prueba para garantizar que los ejecutables cumplen con los requisitos de los usuarios.& Diseñar Casos de Prueba\\
&&Realizar pruebas.\\
&&Proponer modificaciones en los componentes.\\
&&Depurar componentes.\\
\hline
Programador& En el SITEM representa el integrante de mayor jerarquía dentro del proceso de desarrollo. Engloba las funciones asociadas a los demás participantes. & Desarrollar componentes\\
\hline
\end{tabular}
\caption{Perfil de los participantes del SITEM.}
\label{participantes_sitem} 
\end{center}
\end{table}

\subsection{Entorno de usuario}

El usuario opera una interfaz web a través de un navegador HTTP, con soporte para HTML 4.0, XML 2.0, javascript, XSL y Cascada Style Sheet 1.0. 

Para acceder a las diferentes secciones del SITEM se requiere que el usuario ejecute un proceso de Autenticación, Autorización y Registro (AAA) – asociados a una sesión. Hasta la versión 3.0 se mantendrá un entorno gráfico básico centrado especialmente en hipervínculos y diseño gráfico mínimo.

\subsection{Suposiciones y dependencias}
\begin{itemize}
\item La plataforma tecnológica sobre la que se implementa el módulo tiene una disponibilidad superior al 99 por ciento del tiempo.
\item Las herramientas de desarrollo son Software Libre.
\item El SITEM podrá integrar en su arquitectura otras aplicaciones de Software Libre o Público.
\item El hilo principal de desarrollo estará en la Universidad Distrital pero no se restringirá la distribución del producto a usuarios interesados.
\item El grupo de participantes en el SITEM es indefinido. Los procesos se potenciaran en la medida que se produzcan “explosiones” de desarrollo fomentadas por usuarios interesados.
\end{itemize}

\subsection{Descripción Global del SITEM}

SITEM es implementado sobre una arquitectura multicapa que distribuye los diferentes componentes en tres capas principales: Presentación, aplicación y datos, estando presente una capa transversal tácita de seguridad. A nivel de usuario el SITEM está compuesto por siete subsistemas autónomos que prestan servicios a sus pares. Estos agrupan seis componentes claves en todo proyecto de telemedicina: entidades de salud, operadores de telecomunicaciones, tecnologías de interconexión, equipos y tecnologías de captura de datos, proyectos e instituciones relacionadas con la telemedicina y servicios médicos - incluyendo módulos de vademécum, consulta de procedimientos, enfermedades y especialidades médicas.

\subsection{Otros Requisitos del Producto}

Estándares Aplicables

\begin{itemize}
\item Unified Process
\item Unified Modeling Language versión 2.0
\item Extensible Markup Language Versión
\item SOAP
\item OWL
\end{itemize}


El sistema debe ser:

\begin{itemize}
\item Multiplataforma.
\item Multiusuario.             
\end{itemize}

Requisitos de Desempeño

\begin{itemize}
\item Velocidad de acceso promedio interior a 10 s.
\item Ayudas contextuales y contenidos autoexplicativos.
\item Disponibilidad superior al 99 por ciento.
\item Manejo de conexiones concurrentes.
\item Integridad referencial en la capa de persistencia.
\end{itemize}

\subsection{Lineamientos de codificación para la organización de los módulos}

Para asegurar una codificación eficiente. que permita realizar búsquedas rápidas dentro de la organización documental, se tienen las siguientes reglas de obligatorio cumplimiento en todos los artefactos:

El nombre del artefacto deberá estar antecedido de un identificador del tipo:

\begin{center}
\textbf{aaa-bbb-ccc}
inicialesmodulo-tipoartefacto-versiónartefacto
\end{center}

Así, para la primera versión del documento de especificaciones de casos de uso, del módulo de administración de instrumentos para la recolección de información ha de tenerse una codificación similar a:

\begin{center}
\textbf{MAI-ECU-001}
\end{center}

Siendo MAI y ECU los identificadores únicos tomados del artefacto Códigos