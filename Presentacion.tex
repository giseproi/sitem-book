\chapter*{\begin{Large}PRESENTACIÓN\end{Large}}
\addcontentsline{toc}{chapter}{PRESENTACIÓN}

En el marco del proyecto\textit{ Telemedicina Bogotá 2K}, el grupo de investigación en telemedicina de la Universidad Distrital Francisco José de Caldas (GITEM), realizó un estudio de campo para caracterizar las entidades de la Red Distrital de Salud. Este estudio permitió dar un diagnóstico del estado de la eSalud en la ciudad y refrendar que las acciones de investigación del grupo estaban alineadas con la propuesta que la Organización Mundial de la salud (OMS) y la Unión Internacional de Telecomunicaciones presentaron en las \textit{Herramientas para el Desarrollo de Estrategias Nacionales de eSalud} \cite{ituoms2012}. 

Como consecuencia de los hallazgos presentados en los informes del Ministerio de Salud y Seguridad Social (MINSALUD) \cite{minsalud2016}, de la Organización para la Cooperación y el Desarrollo Económico (OCDE) \cite{ocde2015} y el Observatorio Así Vamos en Salud \cite{oaves2017}; el grupo GITEM desarrolló la fase III del \textbf{Sistema de Información para la Caracterización de Nodos Potenciales de Redes de eSalud} con el que se propone un Portal de inteligencia analítica que apoye los procesos de definición de capacidad en proyectos de eSalud. 

OpenSITEM, como se ha denominado el sistema, es fruto del esfuerzo que realiza el grupo para sistematizar su experiencia de diagnóstico. Busca ofrecer una plataforma alternativa, independiente y emergente que colabore en la superación de las limitaciones que en materia de gestión y análisis de información tienen los equipos de trabajo encargados del diseño y desarrollo de redes de e-salud. 

En el dominio técnico, OpenSITEM es un sistema federado de aplicaciones de software libre o de código abierto que provee herramientas para analizar datos e información de los siguientes elementos que son de interés para la descripción - y definición de capacidad de, nodos potenciales de redes de e-salud: entidades de salud, servicios médicos, tecnologías de interconexión, operadores de telecomunicaciones, equipos médicos, organizaciones, profesionales, estándares, pacientes, enfermedades, medicamentos y proyectos. OpenSITEM \textit{no es un sistema de eSalud} sino de una plataforma para apoyar la definición de la capacidad que tiene un nodo para potencialmente hacer parte de una red de eSalud.

OpenSITEM propone un mecanismo para la integración de actores en el área de \textit{caracterización y análisis de capacidad} de nodos en redes de eSalud. Provee un escenario ubicuo, basado en tecnologías de la información y un modelo de trabajo colaborativo en red que propende por la construcción evolutiva de una base de información y conocimiento. Siguiendo los lineamientos de apertura en la información y el conocimiento (open philosophy), busca elaborar un \textit{recurso público} para ayudar al diseño de proyectos de e-Salud en nuestra ciudad. 

La arquitectura general del sistema y la experiencia en el desarrollo se presenta en este documento, organizado en las siguientes secciones:

\begin{itemize}
 \item \textbf{Antecedentes}: Enumera los aspectos que motivaron el proyecto y un resumen de las fases anteriores.
 \item \textbf{Estructura del Proyecto}: Muestra un resumen de los aspectos, que a consideración de los autores, son relevantes para medir el impacto presente y la potencialidad del proyecto.
 \item \textbf{Contexto Teórico}: Recopilación de aspectos teóricos que fueron de interés para abordar el desarrollo del trabajo de investigación . 
 \item \textbf{OpenSITEM}: Descripción de la arquitectura del sistema resultante haciendo énfasis en sus conceptos y propiedades fundamentales. El producto software que de manera emergente se está creando está centrado en esta descripción y guiado por los casos de uso que se presentan en este informe. Esta sección se complementa con los anexos de los modelos de requisitos, análisis y diseño, implementación, despliegue y datos.
 \item \textbf{Experiencia de Desarrollo}: Teniendo en cuenta que la plataforma de software que soporta OpenSITEM es de código abierto, se presenta una definición del método de trabajo empleado para que los grupos interesados en participar puedan integrarse a la comunidad de desarrollo. Esta sección hace especial énfasis en los aportes originales del grupo y describe las demás herramientas que se integran al modelo.
 \item \textbf{Conclusiones y Recomendaciones}: Aspectos y hallazgos que se deben considerar para la consolidación de OpenSITEM así como el desarrollo de nuevos proyectos de investigación.
\end{itemize}

\begin{figure}[!htpb]
\begin{minipage}[r]{0.8\textwidth}
\begin{flushright}
\textbf{Lilia Edith Aparicio Pico. PhD}
\\Directora GITEM+\\Grupo de Investigación en Telemedicina\\Universidad Distrital Francisco José de Caldas
\end{flushright}
\end{minipage}
\begin{minipage}[r]{0.15\textwidth}
 \includegraphics[width=20mm, height=26mm]{edith.png}
\end{minipage}
\end{figure}