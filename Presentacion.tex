\chapter*{\begin{Large}PRESENTACIÓN\end{Large}}
\addcontentsline{toc}{chapter}{PRESENTACIÓN}

En el marco del proyecto\textit{ Telemedicina Bogotá 2K}, el grupo de investigación en telemedicina de la Universidad Distrital Francisco José de Caldas (GITEM), realizó un estudio de campo para caracterizar las entidades de la Red Distrital de Salud que permitió dar un diagnóstico del estado de la eSalud en la ciudad. Observando que las acciones de investigación estaban alineadas con la propuesta que la Organización Mundial de la salud (OMS) y la Unión Internacional de Telecomunicaciones presentaron en las \textit{Herramientas para el Desarrollo de Estrategias Nacionales de eSalud} \cite{ituoms2012}; y motivados por los hallazgos presentados en los informes del Ministerio de Salud y Seguridad Social (MINSALUD) \cite{minsalud2016}, de la Organización para la Cooperación y el Desarrollo Económico (OCDE) \cite{ocde2015} y el Observatorio Así Vamos en Salud \cite{oaves2017}; el grupo GITEM desarrolló la tercera fase del \textbf{Sistema de Información para la Caracterización de Proyectos de Telesalud} con el que se propone un Portal de inteligencia analítica que apoye los procesos de definición de capacidad en instituciones que deseen emprender proyectos de eSalud. En general, SITEM, como se ha denominado el sistema, es fruto del esfuerzo que realiza el grupo para sistematizar su experiencia de diagnóstico y ofrecer un mecanismo que permita resolver las limitaciones que en materia de captura y análisis de información han expresado las organizaciones antes mencionadas.

En el dominio técnico, SITEM es un sistema federado de aplicaciones de software libre o de código abierto que provee herramientas para analizar datos e información de los siguientes componentes - que son de interés tanto en la definición de capacidad como en el seguimiento de proyectos de eSalud: profesionales, estándares, pacientes, enfermedades, medicamentos, entidades de salud, servicios médicos, tecnologías de interconexión, operadores de telecomunicaciones, equipos médicos, organizaciones y proyectos. SITEM es una alternativa independiente y emergente para apoyar la descripción y evaluación de redes de atención en salud. En este punto vale la pena aclarar que no se trata de un sistema de eSalud sino de una plataforma para apoyar la definición de la capacidad que tiene una institución para el emprendimiento de proyectos de eSalud.

El proyecto recoge las experiencias que el grupo ha acumulado a través de más de doce años de trabajo continuo en el área de la Telesalud y las aplica a modelos de gestión de conocimiento para formular una arquitectura conceptual y de software que soporta ciclos de integración, creación, reproducción y distribución de conocimiento a través de un portal especializado. En su más reciente versión integra \textit{Agentes notificadores y de Recomendación} que analizan constantemente los repositorios de información del sistema para, a partir del uso de inteligencia artificial, desplegar información referenciada y adaptada a las necesidades y perfiles de cada uno de los usuarios registrados dentro del sistema.

SITEM propone un mecanismo para la integración de actores en el área de \textit{análisis de capacidad} para el despliegue de soluciones de Telesalud, proveyendo un escenario ubicuo, basado en tecnologías de la información y un modelo de trabajo colaborativo en red que propende por la construcción evolutiva de una base de información y conocimiento. Esto, asociado a los conceptos de libertad en el uso de la información y el conocimiento (open philosophy), busca elaborar un \textit{recurso público} esencial para el desarrollo de Telesalud en nuestra ciudad. El modelo conceptual del SITEM aborda la plataforma para Telesalud con un enfoque holístico que trasciende la dimensión técnica y tecnológica y abarca aspectos médicos, culturales y de gestión. 

La arquitectura general del sistema, así como la experiencia en el desarrollo, es presentado en este documento que, para facilitar su consulta, se ha organizado en las siguientes secciones:

\begin{itemize}
 \item \textbf{Antecedentes} Se presentan los aspectos que motivaron el proyecto y un resumen de las fases anteriores.
 \item \textbf{Contexto Teórico} Recopilación de los aspectos teóricos que soportan el proyecto. 
 \item \textbf{SITEM} Describe la arquitectura del sistema, haciendo énfasis en sus conceptos y propiedades fundamentales.
 \item \textbf{Método de Desarrollo} Teniendo en cuenta que la plataforma de software que soporta SITEM es de código abierto, se presenta una definición del método de trabajo empleado para que los grupos interesados en participar puedan integrarse a la comunidad de desarrollo. Esta sección hace especial énfasis en los aportes originales del grupo y describe las demás herramientas que se integran al modelo.
 \item \textbf{Método de Estudio de Campo}: A partir de la experiencia del equipo de campo y con el objetivo de potenciar los aspectos positivos y gestionar de manera correcta el riesgo asociado a los aspectos negativos, en esta sección se define el modelo para la articulación Universidad - Distrito en estudios de campo en entidades de salud del distrito.
 \item \textbf{Conclusiones y Recomendaciones}: Aspectos y hallazgos que se deben considerar para la consolidación del SITEM así como el desarrollo de proyectos de investigación.
\end{itemize}

\begin{figure}[!htpb]
\begin{minipage}[r]{0.8\textwidth}
\begin{flushright}
\textbf{Lilia Edith Aparicio Pico}

Grupo de Investigación en Telemedicina
Universidad Distrital Francisco José de Caldas
\end{flushright}
\end{minipage}
\begin{minipage}[r]{0.15\textwidth}
 \includegraphics[width=20mm, height=26mm]{edith.png}
\end{minipage}
\end{figure}